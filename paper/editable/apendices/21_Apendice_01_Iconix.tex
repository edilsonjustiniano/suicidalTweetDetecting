\chapter*{Apêndice 1: Iconix}
\label{ap1:iconix}

Neste Apêndice serão apresentados os artefatos gerados ao longo do processo de desenvolvimento deste trabalho.

\section*{Fluxos de eventos}

Após a criação do diagrama de casos de uso, foram gerados os fluxos de eventos para cada caso de uso, como serão apresentados a seguir. O primeiro fluxo de eventos gerado é referente ao caso de uso ''Aceitar parceria'', conforme apresenta o Quadro~\ref{ap1:quad:fluxo_evento_aceitar_parceria}.

\captionsetup[quadro]{list=no}
\begin{quadro}[h!]
	\input{./fluxos/fluxo-evento-aceitar-parceria}
	\caption[Fluxo de eventos para o caso de uso ''Aceitar parceria''.]
	{Fluxo de eventos para o caso de uso ''Aceitar parceria''. \textbf{Fonte:} Elaborado pelos autores}
	\label{ap1:quad:fluxo_evento_aceitar_parceria}
\end{quadro}

O próximo fluxo de eventos gerado é referente ao caso de uso ''Localizar parceiro'', conforme apresenta o Quadro~\ref{ap1:quad:fluxo_evento_localizar_parceiro}.

\captionsetup[quadro]{list=no}
\begin{quadro}[h!]
	\input{./fluxos/fluxo-evento-localizar-parceiro}
	\caption[Fluxo de eventos para o caso de uso ''Localizar parceiro''.]
	{Fluxo de eventos para o caso de uso ''Localizar parceiro''. \textbf{Fonte:} Elaborado pelos autores}
	\label{ap1:quad:fluxo_evento_localizar_parceiro}
\end{quadro}

O próximo fluxo de eventos gerado é referente ao caso de uso ''Adicionar parceiro'', conforme apresenta o Quadro~\ref{ap1:quad:fluxo_evento_adicionar_parceiro}.

\newpage
\captionsetup[quadro]{list=no}
\begin{quadro}[h!]
	\input{./fluxos/fluxo-evento-adicionar-parceiro}
	\caption[Fluxo de eventos para o caso de uso ''Adicionar parceiro''.]
	{Fluxo de eventos para o caso de uso ''Adicionar parceiro''. \textbf{Fonte:} Elaborado elos autores}
	\label{ap1:quad:fluxo_evento_adicionar_parceiro}
\end{quadro}

O próximo fluxo de eventos gerado é referente ao caso de uso ''Localizar mão de obra'', conforme apresenta o Quadro~\ref{ap1:quad:fluxo_evento_localizar_mao_de_obra}.

\newpage
\captionsetup[quadro]{list=no}
\begin{quadro}[h!]
	\input{./fluxos/fluxo-evento-localizar-mao-de-obra}
	\caption[Fluxo de eventos para o caso de uso ''Localizar mão de obra'']
	{Fluxo de eventos para o caso de uso ''Localizar mão de obra''. \textbf{Fonte:} Elaborado pelos autores}
	\label{ap1:quad:fluxo_evento_localizar_mao_de_obra}
\end{quadro}

O próximo fluxo de eventos gerado é referente ao caso de uso ''Avaliar mão de obra'', conforme apresenta o Quadro~\ref{ap1:quad:fluxo_evento_avaliar_mao_de_obra}.

\newpage
\captionsetup[quadro]{list=no}
\begin{quadro}[h!]
	\input{./fluxos/fluxo-evento-avaliar-mao-de-obra}
	\caption[Fluxo de eventos para o caso de uso ''Avaliar mão de obra''.]
	{Fluxo de eventos para o caso de uso ''Avaliar mão de obra''. \textbf{Fonte:} Elaborado pelos autores}
	\label{ap1:quad:fluxo_evento_avaliar_mao_de_obra}
\end{quadro}

O próximo fluxo de eventos gerado é referente ao caso de uso ''Gerenciar serviços'', conforme apresenta o Quadro~\ref{ap1:quad:fluxo_evento_gerenciar_servicos}.

\newpage
\captionsetup[quadro]{list=no}
\begin{quadro}[h!]
	\input{./fluxos/fluxo-evento-gerenciar-servicos}
	\caption[Fluxo de eventos para o caso de uso ''Gerenciar serviços''.]
	{Fluxo de eventos para o caso de uso ''Gerenciar serviços''. \textbf{Fonte:} Elaborado pelos autores}
	\label{ap1:quad:fluxo_evento_gerenciar_servicos}
\end{quadro}

O último fluxo de eventos gerado é referente ao caso de uso ''Criar conta'', conforme apresenta o Quadro~\ref{ap1:quad:fluxo_evento_criar_conta}.

\newpage
\begin{quadro}[h!]
	\input{./fluxos/fluxo-evento-criar-conta}
	\caption[Fluxo de eventos para o caso de uso ''Criar conta''.]
	{Fluxo de eventos para o caso de uso ''Criar conta''. \textbf{Fonte:} Elaborado pelos autores}
	\label{ap1:quad:fluxo_evento_criar_conta}
\end{quadro}

Após a criação dos fluxos de evento, foram criados os diagramas de robustez que serão apresentados a seguir.

\section*{Diagramas de Robustez}

Os diagramas de robustez foram criados baseado-se no diagrama de caso de uso, somado aos fluxos de eventos demonstrados na seção anterior. Seguindo a mesma ordem dos fluxos de eventos, o primeiro diagrama de robustez apresentado na Figura~\ref{fig:ap1:diagrama_robustez_aceitar_parceria} é referente ao caso de uso ''Aceitar parceria''.

\newpage
\captionsetup[figure]{list=no}
\begin{figure}[h!]
	\centerline{\includegraphics[scale=0.38]{./imagens/apendices/diagrama-robustez-aceitar-parceria.png}}
	\caption[Diagrama de robustez referente ao caso de uso ''Aceitar parceria''.]
	{Diagrama de robustez referente ao caso de uso ''Aceitar parceria''. \textbf{Fonte:} Elaborado pelos autores.}
	\label{fig:ap1:diagrama_robustez_aceitar_parceria}
\end{figure}

O próximo diagrama de robustez apresentado na Figura~\ref{fig:ap1:diagrama_robustez_localizar_parceiro} é referente ao caso de uso ''Localizar parceiro''.

\captionsetup[figure]{list=no}
\begin{figure}[h!]
	\centerline{\includegraphics[scale=0.37]{./imagens/apendices/diagrama-robustez-localizar-parceiros.png}}
	\caption[Diagrama de robustez referente ao caso de uso ''Localizar parceiro''.]
	{Diagrama de robustez referente ao caso de uso ''Localizar parceiro''. \textbf{Fonte:} Elaborado pelos autores.}
	\label{fig:ap1:diagrama_robustez_localizar_parceiro}
\end{figure}

O próximo diagrama de robustez apresentado na Figura~\ref{fig:ap1:diagrama_robustez_adicionar_parceiro} é referente ao caso de uso ''Adicionar parceiro''.

\newpage
\captionsetup[figure]{list=no}
\begin{figure}[h!]
	\centerline{\includegraphics[scale=0.35]{./imagens/apendices/diagrama-robustez-adicionar-parceiro.png}}
	\caption[Diagrama de robustez referente ao caso de uso ''Adicionar parceiro''.]
	{Diagrama de robustez referente ao caso de uso ''Adicionar parceiro''. \textbf{Fonte:} Elaborado pelos autores.}
	\label{fig:ap1:diagrama_robustez_adicionar_parceiro}
\end{figure}

O próximo diagrama de robustez apresentado na Figura~\ref{fig:ap1:diagrama_robustez_localizar_mao_de_obra} é referente ao caso de uso ''Localizar mão de obra''.

\captionsetup[figure]{list=no}
\begin{figure}[h!]
	\centerline{\includegraphics[scale=0.35]{./imagens/apendices/diagrama-robustez-localizar-mao-de-obra.png}}
	\caption[Diagrama de robustez referente ao caso de uso ''Localizar mão de obra''.]
	{Diagrama de robustez referente ao caso de uso ''Localizar mão de obra''. \textbf{Fonte:} Elaborado pelos autores.}
	\label{fig:ap1:diagrama_robustez_localizar_mao_de_obra}
\end{figure}

O próximo diagrama de robustez apresentado na Figura~\ref{fig:ap1:diagrama_robustez_avaliar_mao_de_obra} é referente ao caso de uso ''Avaliar mão de obra''.

\newpage
\captionsetup[figure]{list=no}
\begin{figure}[h!]
	\centerline{\includegraphics[scale=0.43]{./imagens/apendices/diagrama-robustez-avaliar-mao-de-obra.png}}
	\caption[Diagrama de robustez referente ao caso de uso ''Avaliar mão de obra''.]
	{Diagrama de robustez referente ao caso de uso ''Avaliar mão de obra''. \textbf{Fonte:} Elaborado pelos autores.}
	\label{fig:ap1:diagrama_robustez_avaliar_mao_de_obra}
\end{figure}

O próximo diagrama de robustez apresentado na Figura~\ref{fig:ap1:diagrama_robustez_gerenciar_servicos} é referente ao caso de uso ''Gerenciar serviços''.

\captionsetup[figure]{list=no}
\begin{figure}[h!]
	\centerline{\includegraphics[scale=0.38]{./imagens/apendices/diagrama-robustez-adicionar-servicos.png}}
	\caption[Diagrama de robustez referente ao caso de uso ''Gerenciar serviços''.]
	{Diagrama de robustez referente ao caso de uso ''Gerenciar serviços''. \textbf{Fonte:} Elaborado pelos autores.}
	\label{fig:ap1:diagrama_robustez_gerenciar_servicos}
\end{figure}

O último diagrama de robustez apresentado na Figura~\ref{fig:ap1:diagrama_robustez_criar_conta} é referente ao caso de uso ''Criar conta''.

\newpage
\captionsetup[figure]{list=no}
\begin{figure}[h!]
	\centerline{\includegraphics[scale=0.4]{./imagens/apendices/diagrama-robustez-criar-conta.png}}
	\caption[Diagrama de robustez referente ao caso de uso ''Criar conta''.]
	{Diagrama de robustez referente ao caso de uso ''Criar conta''. \textbf{Fonte:} Elaborado pelos autores.}
	\label{fig:ap1:diagrama_robustez_criar_conta}
\end{figure}

Após a criação dos diagramas de robustez, o diagrama de modelo de domínio foi atualizado, adicionando os atributos identificados pelos diagramas de caso de robustez, passando-se a trabalhar nos diagramas de sequência, que seão apresentados a seguir.

\section*{Diagramas de Sequência}

Os diagramas de sequência foram criados baseado-se no diagramas apresentados anteriormente. Seguindo a mesma ordem anteriormente definida, o primeiro diagrama de sequência apresentado na Figura~\ref{fig:ap1:diagrama_sequencia_aceitar_parceria} é referente ao caso de uso ''Aceitar parceria''.

\newpage
\captionsetup[figure]{list=no}
\begin{figure}[h!]
	\centerline{\includegraphics[angle=90,scale=0.42]{./imagens/apendices/diagrama-sequencia-aceitar-parceria.png}}
	\caption[Diagrama de sequência referente ao caso de uso ''Aceitar parceria''.]
	{Diagrama de sequência referente ao caso de uso ''Aceitar parceria''. \textbf{Fonte:} Elaborado pelos autores.}
	\label{fig:ap1:diagrama_sequencia_aceitar_parceria}
\end{figure}

O próximo diagrama de sequência apresentado na Figura~\ref{fig:ap1:diagrama_sequencia_localizar_parceiro} é referente ao caso de uso ''Localizar parceiro''.

\newpage
\captionsetup[figure]{list=no}
\begin{figure}[h!]
	\centerline{\includegraphics[angle=90,scale=0.42]{./imagens/apendices/diagrama-sequencia-localizar-parceiros.png}}
	\caption[Diagrama de sequência referente ao caso de uso ''Localizar parceiro''.]
	{Diagrama de sequência referente ao caso de uso ''Localizar parceiro''. \textbf{Fonte:} Elaborado pelos autores.}
	\label{fig:ap1:diagrama_sequencia_localizar_parceiro}
\end{figure}

O próximo diagrama de sequência apresentado na Figura~\ref{fig:ap1:diagrama_sequencia_adicionar_parceiro} é referente ao caso de uso ''Adicionar parceiro''.

\begin{landscape}
\newpage
\captionsetup[figure]{list=no}
\begin{figure}[h!]
	\centerline{\includegraphics[scale=0.3]{./imagens/apendices/diagrama-sequencia-adicionar-parceiros.png}}
	\caption[Diagrama de sequência referente ao caso de uso ''Adicionar parceiro''.]
	{Diagrama de sequência referente ao caso de uso ''Adicionar parceiro''. \textbf{Fonte:} Elaborado pelos autores.}
	\label{fig:ap1:diagrama_sequencia_adicionar_parceiro}
\end{figure}

O próximo diagrama de sequência apresentado na Figura~\ref{fig:ap1:diagrama_sequencia_localizar_mao_de_obra} é referente ao caso de uso ''Localizar mão de obra''.

\newpage
\captionsetup[figure]{list=no}
\begin{figure}[h!]
	\centerline{\includegraphics[scale=0.4]{./imagens/apendices/diagrama-sequencia-localizar-mao-de-obra.png}}
	\caption[Diagrama de sequência referente ao caso de uso ''Localizar mão de obra''.]
	{Diagrama de sequência referente ao caso de uso ''Localizar mão de obra''. \textbf{Fonte:} Elaborado pelos autores.}
	\label{fig:ap1:diagrama_sequencia_localizar_mao_de_obra}
\end{figure}

O próximo diagrama de sequência apresentado na Figura~\ref{fig:ap1:diagrama_sequencia_avaliar_mao_de_obra} é referente ao caso de uso ''Avaliar mão de obra''.

\newpage
\captionsetup[figure]{list=no}
\begin{figure}[h!]
	\centerline{\includegraphics[scale=0.2]{./imagens/apendices/diagrama-sequencia-avaliar-mao-de-obra.png}}
	\caption[Diagrama de sequência referente ao caso de uso ''Avaliar mão de obra''.]
	{Diagrama de sequência referente ao caso de uso ''Avaliar mão de obra''. \textbf{Fonte:} Elaborado pelos autores.}
	\label{fig:ap1:diagrama_sequencia_avaliar_mao_de_obra}
\end{figure}

O próximo diagrama de sequência apresentado na Figura~\ref{fig:ap1:diagrama_sequencia_gerenciar_servicos} é referente ao caso de uso ''Gerenciar serviços''.

\newpage
\captionsetup[figure]{list=no}
\begin{figure}[h!]
	\centerline{\includegraphics[scale=0.4]{./imagens/apendices/diagrama-sequencia-gerenciar-servicos.png}}
	\caption[Diagrama de sequência referente ao caso de uso ''Gerenciar serviços''.]
	{Diagrama de sequência referente ao caso de uso ''Gerenciar serviços''. \textbf{Fonte:} Elaborado pelos autores.}
	\label{fig:ap1:diagrama_sequencia_gerenciar_servicos}
\end{figure}

O último diagrama de sequência apresentado na Figura~\ref{fig:ap1:diagrama_sequencia_criar_conta} é referente ao caso de uso ''Criar conta''.

\captionsetup[figure]{list=no}
\begin{figure}[h!]
	\centerline{\includegraphics[scale=0.35]{./imagens/apendices/diagrama-sequencia-criar-conta.png}}
	\caption[Diagrama de sequência referente ao caso de uso ''Criar conta''.]
	{Diagrama de sequência referente ao caso de uso ''Criar conta''. \textbf{Fonte:} Elaborado pelos autores.}
	\label{fig:ap1:diagrama_sequencia_criar_conta}
\end{figure}
\end{landscape}

Após a criação dos diagramas de sequência, o diagrama de modelo de domínio foi atualizado, adicionando os atributos identificados pelos diagramas de sequência, gerando o diagrama de classes final.

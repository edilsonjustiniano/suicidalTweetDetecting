%\chapter*{Introdução}
\begin{flushleft}
	\vspace{1.2em}
	\textbf{\large INTRODUÇÃO}
	\vspace{2.9em}
\end{flushleft}
\thispagestyle{empty}

\addcontentsline{toc}{chapter}{INTRODUÇÃO}
\stepcounter{chapter} %incrementa o número do capítulo

Atualmente, estamos vivendo em meio a um mundo totalmente globalizado e extremamente conectado, isso afeta diretamente as nossas vidas, visto que as pessoas
anceiam por estar conectadas cada vez mais. Basta olharmos para o mercado de Telecomunicações e suas ofertas, que possuem como enfoque planos de dados e não mais
minutos de chamadas ou \textit{Short Messages} - SMS. Essas ofertas refletem a mudança de estilo de vida das pessoas nos últimos anos, demonstrando claramente
esse desejo e/ou necessidade de se manter conectado na grande rede de computadores, a internet. Toda essa conectividade impacta diretamente a vida social
das pessoas, pois muitas delas, deixam de se comunicar com seus amigos, familiares cara a cara, para se comunicar com eles ou outras pessoas por meio de mídias
sociais ou outras ferramentas \textit{online}.

Essa comunicação com pessoas em tempo real fora facilitado pela acessibilidade a tecnologia, segundo \citeonline{lima_mariana_number_of_smartphones_in_brazil} a projeção é que no Brasil haja mais de um \textit{smartphone} ativo por pessoa, esse valor corresponderia a mais de trezentos milhões de aparelhos. Além dos smartphones outras tecnologias que fazem parte do conceito de \textit{Internet of Things} - IOT (Internet das coisas)
veem se popularizando, aumentado ainda mais o tempo de conexão das pessoas a internet. Não é nada incomum hoje, encontrarmos casas que possuem um controle de seus
dispositivos eletrônicos realizados de forma totalmente remota.

Uma das ferramentas que ajudou na disseminação ou necessidade do aumento da conectividade dos usuários com a internet foram as mídias socias. Criadas a pouco mais de uma dezena de anos, as mídias sociais teve como ideal inicial prover um mecanismo para conectar pessoas estreitando os laços de amizades entre elas, removendo o impecílio da distância física entre elas. Porém, com o passar dos anos essas mídias socias se tornaram uma grande vitrine, e, algumas pessoas a utilizam como uma
ferramenta para gerar lucros. Visto que por meio delas as pessoas podem disponibilizar conteúdos de todos os tipos e, é claro dar sua opnião a respeito de um determinado assunto, sem nenhuma restrição. Portanto, muitas pessoas passaram a utilizar as mídias sociais para descrever seus dia-a-dia, e, até mesmo transcrever seus sentimentos, sejam eles bons ou ruins, assim como um antigo diário, deixando assim sua vida social totalmente exposta para as pessoas que se interessam pela sua vida pessoal. 

Atualmente as mídias socias como \textit{Facebook} e \textit{Twitter}, entre outras somam mais de três bilhões de usuários ativos, conforme \citeonline{kemp_simon_number_of_social_media_users}, esse número só tende a crescer nos próximos anos. Todo esse sucesso das redes sociais somado ao fato da facilidade de se expressar e comunicar com os demais usuários abriram oportunidades para as pessoas falarem, revelarem situações corriqueiras ou não que as incomodavam em sua vida, compartilhando assim experiências de vida. 

Isso se deve a facilidade e a capacidade de usuários de mídias socias falarem e serem ouvidos, esta por sua vez se tornou uma válvula de escape para problemas comportamentais, pessoais, financeiros e outros. Principalmente entre os jovens que são os principais alvos de \textit{Bullying}, seja ele na escola, na família ou na sociedade como um todo. \textit{Bullying} que se tornou um dos principais problemas enfrentados em escolas de todo o mundo, como apresentado por \citeonline{the_guardian_ten_years_of_bulyng_data}. 

Infelizmente  o \textit{Bullyng} veem se alastrando perante a sociedade e, de certa forma, acaba contribuindo para catástrofes, como atentados em casos extremos. Esse problema, é enfretado por milhares de pessoas, principalmente jovens no dia a dia, em muitos desses casos as próprias vítimas se sentem tão constrangidas que não conseguem falar a respeito disso com seus amigos próximos, familiares ou com algum tipo profissional.

Devido a essa dificuldade em conversar a respeito desses problemas e somado ao fato de redes sociais serem abertas em alguns casos, jovens costumam publicar mensagens ou pensamentos que poderiam ser rotulados como uma mudança de humor, como uma tristeza em alguns casos mais sérios, poderiam sugerir algum tipo de depressão e em casos extremos alguma tendência a
tirar a sua própria vida. Entre as redes sociais, hoje uma das mais utilizadas, e, com um grande número de usuários ativos é o \textit{Twitter}, isso se deve ao fato dele ser uma rede social aberta, onde o que é escrito por um usuário pode ser visto por qualquer outra pessoa conforme \citeonline{bridanne_stephen_twitter_research}.

Visto que esse problema é extramente sério e que, os casos são mais detectados em jovens estudantes esse trabalho pretende criar uma ferramenta utilizando algoritmos de \textit{Machine Learning} que será capaz de classificar um determinado \textit{tweet} entre três tipos possíveis, são eles: alto risco de ser real; risco de ser real; seguro para ser ignorado; ajudando assim escolas e instituições de ensinos, bem como os pais e familiares de alunos a identiticarem desvios de humor a fim de evitar finais trágicos e melhorar assim a vida social dos envolvidos.

%\par \citeonline[p. 2]{neo4j_team_manual} afirma que:

%\begin{citacao}
%	\textit{A single server instance can handle a graph of billions of nodes and relationships. When data throughput is insufficient, the graph database can be distributed among multiple servers in a high availability configuration.}\footnotemark[11]
%\end{citacao}
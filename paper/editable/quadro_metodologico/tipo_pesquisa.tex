\section{Tipo de pesquisa}

\par Para \citeonline[p. 31]{padua_metodologia_pesquisa}, pesquisa é:

\begin{citacao}
	Toda atividade voltada para a solução de problemas; como atividade de busca, indagação, investigação, inquirição da realidade, e a atividade que visa nos permitir, no âmbito da ciência, elaborar um  conhecimento, ou um conjunto de conhecimentos, que nos auxilie na compreensão desta realidade e nos oriente em nossas ações.
\end{citacao}

% Comentário, pois estava repetindo demais o fato de utilizar  apesquisa aplicada no desenvolvimento do trabalho. 
%\par O tipo de pesquisa aplicada foi escolhido para o desenvolvimento da pesquisa, pois conforme \citeonline[p. 32]{cooper_schindler_metodos_pesquisa_administracao}, ela ``tem uma ênfase prática na solução de problemas, embora a solução de problemas nem sempre seja gerada por uma circunstância negativa.''

\par De forma objetiva, a pesquisa é o meio utilizado para buscar respostas aos mais diversos tipos de indagações, tendo por base procedimentos racionais e sistemáticos. A pesquisa é realizada quando se tem um problema e não se tem informações suficientes para solucioná-lo. Esta seção tem como objetivo explicar o tipo de pesquisa que norteou o desenvolvimento deste trabalho, justificando também como ele se enquadra no tipo escolhido.

\par Pesquisar é um trabalho que envolve planejamento, para que ela seja satisfatória, o pesquisador precisa estar envolvido e  desenvolver habilidades técnicas que o levem a escolher o melhor caminho em busca da obtenção dos resultados.

\par Segundo \citeonline{fonseca_metodologia_da_pesquisa}, ter um método de pesquisa envolve o estudo dos fatores que compõem o contexto da mesma, tais como, a escolha do caminho e o planejamento do percurso. Essa escolha inicia-se com a definição do tipo de pesquisa utilizada. Para este trabalho, é utilizada a pesquisa aplicada, que é aquela cujo o pesquisador tem como objetivo aplicar os conhecimentos obtidos durante o período da pesquisa, em um projeto real, a fim de conhecer os seus resultados. \citeonline{gil_como_elaborar_projeto_de_pesquisa} afirma que este tipo de pesquisa dirige-se à solução de problemas específicos, de interesses locais.

%\par Conforme \citeonline[p. 32]{cooper_schindler_metodos_pesquisa_administracao}, a pesquisa aplicada tem uma ênfase prática na solução de problemas. 
\par Nesta pesquisa foram estudados os conceitos de banco de dados orientado a grafos e a sua aplicabilidade, desenvolvendo, por meio dos conhecimentos obtidos, uma solução prática, disponibilizada por meio de um sistema \textit{web}, que auxilia na busca por mão de obra temporária, que não caracterize vínculo empregatício.

\par Seguindo o enquadramento desta pesquisa, ela deve ser aplicada a um contexto específico, conforme será abordado a seguir. 



%\par Seguindo esta ideia, este tipo de pesquisa será utilizado no desenvolvimento deste projeto, pois, o objetivo é gerar uma solução para o problema de localização de mão de obra, por meio de um sistema \textit{web}. Este projeto adequa-se perfeitamente ao tipo de pesquisa aplicada, uma vez que o mesmo busca analisar e gerar uma possível solução para o problema em destaque.

%Parágrafo utilizado no pré-projeto. Foi corrigido por Edilson no dia 02/04/15 e criado o parágrafo acima
%\par Este tipo de pesquisa será utilizada no desenvolvimento deste projeto, pois a mesma busca analisar o problema e gerar uma solução para o mesmo através de um aplicativo ou serviço.  Neste caso, será o desenvolvimento de um sistema \textit{web} que auxilia na busca por profissionais temporários.

\section{Contexto de pesquisa}

\par O trabalho informal é um elemento estrutural da economia no Brasil e nos países em desenvolvimento. Ele faz parte do cenário atual, crescente a cada dia, e contribui ativamente para a geração de renda. É considerado como um desdobramento do excesso de mão de obra, definido a partir de pessoas que criam sua própria forma de trabalho como estratégia de sobrevivência ou como forma alternativa de recolocação no mercado de trabalho. O fortalecimento deste tipo de trabalho ocorre a partir da construção de redes, formadas por parentes e amigos, criando laços de confiança que são fundamentais para o desempenho da atividade. No entanto, há uma grande dificuldade de se encontrar estes profissionais, uma vez que não há um lugar centralizado para divulgar o seu perfil profissional.

\par O desenvolvimento deste trabalho se propôs a atuar sobre essa limitação. Uma pesquisa informal, realizada por meio de um questionário, na região do sul de Minas Gerais, com pessoas de diferentes perfis sociais, constatou que uma aplicação capaz de centralizar a busca por estes profissionais seria muito bem aceita. A partir deste resultado, validou-se a ideia de construir um ambiente \textit{web} onde o trabalhador informal tenha espaço para centralizar suas habilidades e manter um perfil visível aos possíveis contratantes. Qualquer prestador de serviço informal pode ter acesso a este ambiente, desde que possua um dispositivo eletrônico capaz de se conectar à internet. 

\par O ambiente desenvolvido também visa facilitar ao contratante a busca por estes profissionais, uma vez que não é fácil localizá-los por meio dos mecanismos de busca tradicionais. Desta forma, existe um benefício mútuo, em que contratados e contratantes dispõem da praticidade.

\par Enfim, o contexto ao qual esta pesquisa se destina busca ser bem abrangente, com o intuito de contribuir de forma relevante, proporcionando uma boa experiência aos envolvidos.
 
%Exemplo de contexto que a Joelma deu na sala de aula. Usar como exemplo
%\par Este sistema volta-se para implantação e execução em todas as empresas de pequeno porte do ramo varejista no sul de minas gerais que tenha apresentado a necessidade de um controle mais rigoroso da sua entrada e saída de mercadorias

%Segundo a Joelma e o Márcio este seria o contexto da pesquisa: As pessoas que buscam mão de obra para determinados tipos de trabalho (contratantes) e as pessoas que disponibilizam tais mãos de obra (contratados ou prestadores de serviços).

%Antigo Contexto que a Joelma disse que estava muito geral na correção do pré-projeto
%\par Esta pesquisa terá como foco todas as pessoas que necessitam de mão de obra temporária para realizar tarefas domésticas e rotineiras, além daquelas que não ocorrem com tanta intensidade. Uma vez que o sistema será desenvolvido em uma plataforma \textit{web}, todas as pessoas terão fácil acesso ao serviço, sendo necessário apenas possuir uma comunicação com a internet.

%\par Esta pesquisa terá como foco todas as pessoas da região do sul de Minas Gerais, que necessitam de determinados tipos de mão de obra (contratantes), bem como para aqueles que necessitam de um espaço para divulgar esta oferta (contratados ou prestadores de serviços). Tomando por base que encontrar este tipo de profissional tem se tornado uma tarefa cada vez mais complicada, que acaba gerando transtornos na vida da população, pois, em muitos casos, a falta de opções leva a contratações equivocadas e infelizes.

%\par Como o sistema será desenvolvido em uma plataforma \textit{web}, todas as pessoas neste contexto terão fácil acesso ao serviço, permitindo a disseminação desta ferramenta, sendo necessário apenas possuir uma comunicação com a internet.

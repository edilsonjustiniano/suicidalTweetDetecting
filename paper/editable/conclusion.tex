\chapter{CONCLUSÃO} 


\par O número de pessoas que utiliza a internet para realizar tarefas rotineiras vem crescendo nos últimos tempos e isso tem gerado um grande impacto social. Por meio de toda essa metamorfose, viu-se a necessidade de adaptar o tempo de execução de algumas tarefas, a fim de otimizá-lo.  Este trabalho propôs atuar nessa limitação, produzindo um serviço capaz de atender a necessidade na busca por profissionais temporários, sem vínculo empregatício formalizado. A escolha do tema justifica-se pela escassez de aplicações que ofereçam este tipo de auxílio aos usuários.

\par Visando propor uma possível alternativa para esta situação foram realizadas pesquisas e constatou-se que o modelo de negócios mais difundido seria o equiparado aos das redes sociais, tão comuns atualmente. Este modelo consiste em relacionamentos que podem ser facilmente resolvido aplicando a teoria dos grafos, onde cada usuário pode possuir vínculos de amizade com base em vários aspectos, seja por afinidades, cidade em que vive, empresa onde trabalha, entre tantos outros. Por se tratar de um trabalho que possui características de relacionamento parecidas com as mencionadas acima, este modelo foi escolhido para a realização deste trabalho, além de se tratar de um modelo bem aceito e conhecido, facilitando a sua aceitação no mercado.

\par Com o desenvolvimento deste trabalho foi possível observar a dificuldade encontrada pelas pessoas que buscam por estes tipos de profissionais, sendo que, muitos alegam que para efetuar a contratação é necessário ter alguma referência sobre o profissional que desempenhará o serviço. Desta forma, foi observado que as pessoas ainda são conservadoras nestes aspectos, escolhendo sempre profissionais que possuam boas referências, principalmente de pessoas próximas a elas. Em contra partida, foi observado também, que muitas vezes o profissional temporário não possui um espaço para ofertar o seu trabalho, o que os leva a ficar no anonimato e muitas vezes não conseguirem trabalho. Estes profissionais podem ser tanto os que preferem trabalhar de maneira informal, quanto os que não conseguem uma colocação ou recolocação no mercado de trabalho e precisam trabalhar.

\par  Por meio deste trabalho foi desenvolvido um ambiente \textit{web} que oferece suporte tanto ao contrante quanto o profissional temporário. Foram desenvolvidas funcionalidades específicas para cada tipo de usuário, sendo que a tecnologia escolhida para a realização foi de suma importância. Para fazer o armazenamento das informações optou-se por utilizar o banco de dados Neo4j, que como já mencionado no decorrer deste projeto, oferece várias vantagens, inclusive possuir uma base de dados orientada a grafos, o que permite várias formas de relacionamentos, se enquadrando na proposta de desenvolvimento deste trabalho. Para realizar as operações, foi utilizada a API \textit{Cypher}, que possibilitou a escrita das diversas consultas necessárias para retorno de informações e também para a inserção de dados no banco, de uma forma simples, clara e objetiva. 

\par De forma breve, foi observado, também, por meio deste trabalho, o desempenho do \textit{framework} Angular JS, que tornou o desenvolvimento muito mais ágil, uma vez que sua biblioteca traz várias funções pré moldadas.

\par Com a utilização destas tecnologias foram desenvolvidas funcionalidades como a possibilidade de criar uma rede de parcerias, realizando buscas por possíveis parceiros, sendo estes sugeridos pelo sistema com base em fatores comuns ou ainda buscando diretamente pelo nome. Também foi implementada a opção de buscar serviços, avaliá-los e comparar sua qualidade. Por meio da implementação dessas funcionalidades, observou-se que esse conjunto  atenderia não somente a justificativa inicial deste trabalho, como também daria a oportunidade de criação para novas ferramentas.

\par Visto que o tempo hábil para o desenvolvimento deste trabalho foi limitado, algumas funcionalidades ficaram por serem desenvolvidas. Um exemplo seria a opção de troca de mensagens entre os prestadores de serviço e contratantes. Esta funcionalidade permitiria que ambos pudessem se comunicar de forma rápida, a fim de agilizar o processo de contratação e definição das tarefas a ser realizadas. Com um tempo maior de estudos e testes, este ambiente poderia ter uma versão \textit{mobile}, o que facilitaria sua visualização em dispositivos móveis. 

\par Uma outra melhoria prevista é a de criar planos de contas. Estes planos visam separar em níveis os usuários cadastrados no sistema. A ideia é oferecer benefícios aos usuários que possuírem um nível maior que os demais. Estes benefícios serão obtidos mediante a alguma forma de pagamento ainda não definida.

\par Posterior ao desenvolvimento deste trabalho notou-se que esta aplicação causaria impacto social positivo, uma vez que oferece uma alternativa ao modo tradicional de se procurar por profissionais temporários. Com a utilização deste ambiente, há uma otimização do tempo de busca, pois o contato deles estão centralizados em um único lugar, sendo que eles são apresentados levando em consideração o nível de avaliação realizado por pessoas que fazem parte do círculo de parcerias do usuário que está realizando a busca. Outro impacto percebido foi que ao utilizar o sistema, o prestador de serviços teria uma maneira mais segura e confiável de divulgar os trabalhos exercidos.  

\par Por meio deste trabalho observou-se também a possibilidade de melhoria contínua, pois o mesmo poderá ser utilizado como base para futuros acadêmicos que tenham como objetivo realizar o estudo do banco de dados orientado a grafos, aplicado à solução de algum problema social. Este trabalho também estará disponível, juntamente com todos os códigos fontes, o que permite que novas funcionalidades sejam desenvolvidas para ele. Por se tratar de uma solução de baixo custo, a maioria da população poderá ter acesso a essa ferramenta de busca, o que irá disseminá-la.

\par Assim conclui-se que o desenvolvimento deste trabalho foi de grande valor, pois ofereceu auxílio a um problema tão comum nos dias atuais, além de proporcionar uma base maior de conhecimento aos seus pesquisadores se tratando da utilização de banco de dados orientado a grafos, tecnologia não pertencente a grade curricular.
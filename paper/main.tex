%Documento principal do latex para uso no TCC da Univás.

\documentclass[a4paper, 12pt, chapter=TITLE, oneside, english, brazil]{abntex2}

\usepackage{styles/CodeStyle}      %Formatação de códigos e listagens
\usepackage{styles/EventFlowStyle} %Estilo para o quadro de fluxo de eventos
\usepackage{styles/NuapaStyle}     %Estilo exigido pela universidade
\usepackage[table,xcdraw]{xcolor}  %Pacote para gerar as tabelas
\usepackage{array,multirow,graphicx} %Pacote para escrever o nome dos meses da tabela em vertical
\definecolor{lightgray}{gray}{0.9} %Pacote para montar a tabela zebrada do orçamento
\usepackage{ragged2e} %Usado para justificar as referências basta colocar no arquivo Principal.bbl o \justify antes das referências. Exemplo abaixo
%\bibitem[W3C 2015]{w3c_css_definition}
%\abntrefinfo{W3C}{W3C}{2015b}
%\justify{W3C. \textbf{What is CSS?} 2015. \url{http:/www.w3.org/Style/CSS/}.
%	Acesso em: 24 de julho de 2015.}

\usepackage{pdflscape}
\usepackage{float}
\restylefloat{table}
%Início do documento
\begin{document}

\pretextual %Início dos Elementos Pré-Textuais c


\makeatletter
\renewcommand{\imprimircapa}{%
\begin{capa}%
\begin{center}%
  {\ABNTEXchapterfont\large\MakeUppercase\imprimirautor}\\
  \vspace*{\fill}%
  \vspace*{\fill}%
  \vspace*{\fill}%
  {\ABNTEXchapterfont\bfseries\Large\MakeUppercase\imprimirtitulo}\\
  \vspace*{\fill}%
  {\color{white}%
\abntex@ifnotempty{\imprimirpreambulo}{%
  \hspace{.45\textwidth}
  \begin{minipage}{.5\textwidth}
    \SingleSpacing
    \imprimirpreambulo
  \vspace{.1\textwidth}\\%
  {\imprimirorientadorRotulo:~\imprimirorientador}
  \abntex@ifnotempty{\imprimircoorientador}{%
    \hspace{.45\textwidth}
    {\large\imprimircoorientadorRotulo~\imprimircoorientador}%
  }%
  \end{minipage}%
}%
\color{black}}%
  \vspace*{\fill}\\%
  {\ABNTEXchapterfont\large\bfseries\MakeUppercase\imprimirinstituicao}\\
  {\large\bfseries\MakeUppercase\imprimirlocal}\\
  {\large\bfseries\MakeUppercase\imprimirdata}
\end{center}
\end{capa}
}
\makeatother

\imprimircapa

% folha de rosto 
%\folhaderostoname{Folha de rosto}
%http://code.google.com/p/abntex2/wiki/ComoCustomizar

\makeatletter
\renewcommand{\folhaderostocontent}{
\begin{center}%
  {\ABNTEXchapterfont\large\MakeUppercase\imprimirautor}\\
  \vspace*{\fill}%
  \vspace*{\fill}%
  \vspace*{\fill}%
  {\ABNTEXchapterfont\bfseries\Large\MakeUppercase\imprimirtitulo}\\
  \vspace*{\fill}%
\abntex@ifnotempty{\imprimirpreambulo}{%
  \hspace{.45\textwidth}
  \begin{minipage}{.5\textwidth}
    \SingleSpacing
    \imprimirpreambulo
  \vspace{.1\textwidth}\\%
  {\imprimirorientadorRotulo:~\imprimirorientador}
  \abntex@ifnotempty{\imprimircoorientador}{%
    \hspace{.45\textwidth}
    {\large\imprimircoorientadorRotulo~:\imprimircoorientador}%
  }%
  \end{minipage}%
}%
  \vspace*{\fill}\\%
  {\ABNTEXchapterfont\large\bfseries\MakeUppercase\imprimirinstituicao}\\
  {\large\bfseries\MakeUppercase\imprimirlocal}\\
  {\large\bfseries\MakeUppercase\imprimirdata}
\end{center}
}%end of folhaderostocontent
\makeatother

\imprimirfolhaderosto*{}
%\begin{fichacatalografica}
\begin{normalsize}
  \vspace*{\fill}
  % Posição vertical
%  \hrule
  % Linha horizontal
  \begin{center}
  \fbox{
    % Minipage Centralizado
    \begin{minipage}[c]{12.5cm} % Largura
      \vspace{0.7cm}%
      \hspace{0.8cm} \imprimirAutorCitacao ~\\%
      
      \hspace{0.8cm} \imprimirtitulo ~/ \imprimirAutorUm , \imprimirAutorDois ~-- \imprimirlocal: Igti, \imprimirdata. %
      
      %\hspace{0.8cm} \pageref{LastPage} f. : il.~\\%
      \hspace{0.8cm} 141 f. : il.~\\%
      
      \hspace{0.8cm} \imprimirtipotrabalho~--~\imprimirinstituicao , Univás, \imprimircurso. % 
      
      \hspace{0.8cm} \imprimirorientadorRotulo: ~\imprimirorientador ~\\%

      \hspace{0.8cm} 1. \imprimirPalavraChaveUm. 2. \imprimirPalavraChaveDois. 3. \imprimirPalavraChaveTres. ~ \\%
    \end{minipage}
    }
  \end{center}
\end{normalsize}
\end{fichacatalografica}
\newpage 
%\setlength{\ABNTEXsignwidth}{10cm}

\begin{folhadeaprovacao}
  \begin{center}
    {\ABNTEXchapterfont\large\MakeUppercase\imprimirautor}
    \vspace*{\fill}
    \vspace*{\fill}
    \vspace*{\fill}
    \par
    {\ABNTEXchapterfont\bfseries\Large\MakeUppercase\imprimirtitulo}
    \vspace*{\fill}
  \end{center}
  Trabalho de conclusão de curso defendido e aprovado em \imprimirDataDaAprovacao ~pela banca examinadora constituída pelos professores:

  ~\newline
  \begin{flushleft}
  \assinatura*{\imprimirorientador \\ \imprimirorientadorRotulo }
  \assinatura*{\imprimirAvaliadorUm \\ \imprimirAvaliadorLabelUm }
  \assinatura*{\imprimirAvaliadorDois \\ \imprimirAvaliadorLabelDois}
  \end{flushleft}
  \vspace*{\fill}
  \vspace*{\fill}
\end{folhadeaprovacao}

%\chapter*{\centerline{DEDICATÓRIA}}

De \imprimirAutorUm.
\newline
%início da dedicatória do autor um
Dedico este trabalho a Deus, por guiar os meus passos e me manter firme no caminho, mesmo em meio às dificuldades. A minha mãe Regina Faria, por ser uma mulher de fibra, guerreira e tão merecedora desta conquista. Obrigada por estar comigo em todos os momentos e por ter acreditado em mim, nós conseguimos. A minha irmã Juliana Faria, por me apoiar e ser a melhor irmã que alguém poderia ter, amo você. Ao meu padrasto José Vitor, por ser um homem íntegro e cheio de bons conselhos. Ao meu tio amado, Carlos Campos, por me incentivar a ir em frente e não desistir dos meus sonhos, o seu apoio foi fundamental para esta realização. Ao meu companheiro Rubens Vilela, por toda dedicação e amor, você fez toda a diferença nesta caminhada. Agradeço por cada conselho, por ser o meu refúgio em momentos difíceis, por todo o apoio. Obrigada por toda essa cumplicidade, você é especial. A todos os meus familiares, aos meus amigos, por serem os melhores do mundo e todos os que de alguma forma, direta ou indiretamente, contribuíram para a concretização deste sonho. Deus os abençoe.

\vspace*{\fill}
De \imprimirAutorDois.
\newline
%início da dedicatória do autor dois
Dedico este trabalho primeiramente, a Deus, pela força e coragem durante toda esta longa caminhada. A minha mãe Cleide de Souza Justiniano, que apesar de todos os contra tempos sempre esteve ao meu lado em todos os momentos, difíceis ou não, desde minha infância. A meu pai Venicius Justiniano, exemplo de pessoa e caráter, que me mostrou o caminho das pedras, ajudando a me tornar o homem que sou hoje. A meus irmãos, Edilaine de Souza Justiniano e Everton de Souza Justiniano pelo apoio e companheirismo. A meus avós, que sempre me incentivaram a correr atrás dos meus sonhos e a estudar em busca de um futuro melhor, principalmente a minha avó Maria Aparecida do Prado (\textit{in memoriam}) que me viu iniciar esta caminhada, porém, infelizmente não pode me ver finalizá-la. A minha noiva Josy Chaves, pessoa com quem amo compartilhar todos os momentos da minha vida. Você foi o meu porto seguro nos momentos que mais precisei ao longo desta caminhada. Obrigado pelo carinho, paciência, compreensão, parceria e por sua capacidade de me trazer alegria e paz na correria de cada semestre, sempre acreditando, apoiando e me incentivando desde o princípio. Graças a você me sinto mais vivo e feliz. Obrigado amor, por fazer parte de minha vida e desta conquista. A meus amigos por estarem sempre ao meu lado me ajudando e incentivando. Aos demais familiares por confiarem em mim, o apoio de vocês foi fundamental para que eu concluísse esta fase em minha vida. Àqueles que direta ou indiretamente contribuíram para que eu alcançasse mais essa vitória em minha vida. Obrigado a todos!

%\chapter*{\centerline{AGRADECIMENTOS}}

Agradecemos,
\newline

%Início agradecimentos em conjunto
\par A Deus, por tudo. Que todas as coisas sejam para a Tua honra e glória.

\par Ao orientador Márcio Emílio, por todo apoio, incentivo e amizade, demostrados não somente neste trabalho mas em todo o período da faculdade.

\par A professora Joelma Faria, que por sua leitura minuciosa nos ajudou para que este trabalho alcançasse sua melhor versão.

\par Aos professores Roberto Rocha e André Martins pelas dicas que contribuíram para a melhorar a qualidade deste trabalho.

\par Ao professor e coordenador José Luiz, por todo apoio e suporte.

\par A todos os professores e colegas de classe, pela amizade, conselhos e todo o conhecimento compartilhado.

\vspace*{\fill}
De \imprimirAutorUm
\newline
%início do agradecimento do autor um

\par Ao meu amigo e parceiro Edilson Justiniano por toda dedicação, compreensão e apoio. Por ser tão incrível e prestativo, se tonando fundamental para a concretização deste trabalho. Muitas vezes foi meu professor, mas acima de tudo, foi e será sempre um grande amigo do qual eu me orgulho muito.

\par Aos meus amigos Sandro Augusto e Jonathan Santos, por todo incentivo e apoio contínuo. Pela amizade que se enraizou no decorrer destes anos de luta.

\par Aos meus familiares, por todo carinho e amor. Em especial a minha mãe, por ser a minha base, me incentivando sempre a buscar e fazer o melhor. 

\par Ao meu querido Rubens Vilela, por estar sempre do meu lado, a cada passo, se fazendo essencial e me sustentando em todas as etapas desta caminhada.

\par A todos os que passaram pelo meu caminho e contribuiriam de alguma forma para que eu chegasse até aqui. De coração, obrigada!


\vspace*{\fill}
De \imprimirAutorDois
\newline
%início do agradecimento do autor dois
\par A minha parceira neste trabalho Andressa de Faria Giordano pela atenção, dedicação e competência, sempre disposta a ajudar no que fosse necessário.

\par Aos amigos Sandro Oliveira e Jonathan Santos, pela amizade e apoio contínuo ao longo desses anos.

\par A minha família, de forma especial, meus pais, por contribuírem com essa realização, sempre acreditando e me incentivando a estudar em busca de um futuro melhor.

\par A minha querida Josy Chaves, obrigado por seu apoio incondicional, sempre me incentivando a fazer o meu melhor. Se tornando essencial nessa conquista.

\par Ao amigo, ex-professor e ex-patrão Túlio Faria que nos ajudou a definir algumas das tecnologias utilizadas neste trabalho.

\par Ao amigo Miller Faria por ajudar no processo de desenvolvimento do trabalho.






%\newpage
%\begin{epigrafe}
\vspace*{\fill}
\begin{flushright}
\textit{"Cada sonho que você deixa para \\trás é um pedaço do seu futuro\\que deixa de existir."
	 \\(Steve Jobs)}
\end{flushright}
\end{epigrafe}

%Lista de Figuras
%\pdfbookmark[0]{\listfigurename}{lof}
%\listoffigures*
%\cleardoublepage

% Lista de Quadros
%\pdfbookmark[0]{\listofquadrosname}{loq}
%\listofquadros*
%\cleardoublepage

%Comentar a Lista de Tabelas porque até o dia 27-04-2015 não temos nenhuma tabela no projeto escrito
%Lista de Tabelas
\pdfbookmark[0]{\listtablename}{lot}
\listoftables*
\cleardoublepage

%Lista de Códigos
%\counterwithout{lstlisting}{chapter}
%\pdfbookmark[0]{\lstlistingname}{lol}
%http://tex.stackexchange.com/questions/50031/how-to-remove-contents-line-from-table-of-contents
%\begin{KeepFromToc}
%\lstlistoflistings
%\end{KeepFromToc}
%\cleardoublepage

%%Lista de siglas
%ver http://marc.info/?l=tex-br&m=110566665520790 para colocar em ordem alfabética.

\begin{SingleSpace}

\begin{siglas}
\item[ACID]  Atomicidade, Consistência, Isolamento e Durabilidade
\item[ACM]   \textit{Association for Computing Machinery}
\item[API]   \textit{Application Programming Interface}
\item[CSS]	 \textit{Cascading Style Sheet}
\item[HTML]	 \textit{Hypertext Markup Language}
\item[HTTP]  \textit{Hypertext Transfer Protocol}
\item[IDE]	 \textit{Integrated Development Environment}
\item[JSF]   \textit{JavaServer Faces}
\item[JSON]	 \textit{Javascript Object Notation}
\item[JSP]   \textit{JavaServer Pages}
\item[JVM]   \textit{Java Virtual Machine}
\item[MVC]   \textit{Model -- View -- Controller}
\item[MySQL] Banco de dados relacional
\item[NoSQL] \textit{Not Only SQL}
\item[REST]  \textit{Representational State Transfer}
\item[SOAP]	 \textit{Simple Object Access Protocol}
\item[SQL]   \textit{Structured Query Language}
\item[UML]   \textit{Unified Modeling Language}
\item[URI]   \textit{Uniform Resource Identifier}
\item[W3C]   \textit{World Wide Web Consortium}
\item[WSDL]  \textit{Web Service Description Language}
\item[XML]   \textit{Extensible Markup Language}

\end{siglas}

\end{SingleSpace}
				
%% --- resumo em português ---

\begin{OnehalfSpacing} 

\noindent \imprimirAutorCitacaoMaiuscula. {\bfseries\imprimirtitulo}. {\imprimirdata}.  Monografia -- Curso de {\MakeUppercase\imprimircurso}, {\imprimirinstituicao}, {\imprimirlocal}, {\imprimirdata}.

\vspace{\onelineskip}
\vspace{\onelineskip}
\vspace{\onelineskip}
\vspace{\onelineskip}

\begin{resumo}
~\\
%início do texto do resumo
\noindent Esta pesquisa apresenta o estudo do banco de dados orientado a grafos aplicado à busca por mão de obra temporária, utilizando o mesmo modelo de negócio presente nas redes sociais. Esta pesquisa é do tipo aplicada, pois visa propor uma solução para um problema social, que acontece dentro de uma região específica, mas que pode ser útil a toda população. Para obter os resultados esperados, foi utilizado o banco de dados Neo4j, juntamente com a API \textit{Cypher}, usada para a navegação e inserção de dados no banco. Foi desenvolvido um sistema na plataforma \textit{web}, com o auxílio das tecnologias HTML, CSS Javascript e o \textit{framework} Angular JS, atuando do lado cliente da aplicação e as tecnologias Java, \textit{Web service} REST, juntamente com o Neo4j, atuando do lado servidor. Foram desenvolvidas funcionalidades que beneficiam tanto o prestador de serviço quanto o contratante, auxiliando-os na tomada de decisão. Como exemplo, pode-se citar a busca por determinado tipo de serviço, avaliação de desempenho de profissionais, criação de redes de parcerias, além de gráficos comparativos, que mostram ao usuário informações sobre as últimas avaliações. Essas funcionalidades tornaram esta pesquisa de grande relevância social, técnica e teórica por contribuir com a exploração de novos recursos, relacionados aos bancos de dados NoSQL. Com isso, conclui-se que os objetivos propostos por este trabalho foram atendidos.


%fim do texto do resumo
\vspace{\onelineskip}
\vspace*{\fill}
\noindent \textbf{Palavras-chave}: \imprimirPalavraChaveUm. \imprimirPalavraChaveDois. \imprimirPalavraChaveTres.
\vspace{\onelineskip}
\end{resumo}

\end{OnehalfSpacing}

%% --- resumo em inglês ---

\begin{OnehalfSpacing} 

\noindent \imprimirAutorCitacaoMaiuscula. {\bfseries\imprimirtitulo}. {\imprimirdata}.  Monografia -- Curso de {\MakeUppercase\imprimircurso}, {\imprimirinstituicao}, {\imprimirlocal}, {\imprimirdata}.

\vspace{\onelineskip}
\vspace{\onelineskip}
\vspace{\onelineskip}
\vspace{\onelineskip}

\begin{resumo}[Abstract]%
\begin{otherlanguage*}{english}%
\textit{
%início do texto do abstract
\noindent This research presents the study about graph databases applied to search for temporary workmanship by using the same business model adopted by social networks. This is an applied research, since it aims to propose a solution for a real social problem that happens inside a specific region but it can be helpful to the entire population. It was used the graph database Neo4j, together with the API Cypher to navigate and also to data management on database. This research also covers the creation of a web based software which was created using the front-end technologies HTML, CSS, Javascript and the framework Angular JS, in client side and, it was used Java programming language, on server-side, to create a REST Web service which work together with Neo4j and provide information on client side. In order to benefit the service providers as well as contractors, some functionalities have been developed to help them in decision-making. As example, it is possible to quote the search for a particular type of service, performance analysis of the professionals, creation of partners network. The application also have comparative graphs which display information about the last evaluations to the users. These functionalities have brought a high relevance to social, technical and theoretical areas because it explored new resources related to NoSQL databases. So, it is possible to conclude that the objectives proposed by this work were met.
%fim do texto do abstract
}

\vspace{\onelineskip}
\vspace*{\fill}
\noindent \textbf{Keywords}: \imprimirKeyWordOne. \imprimirKeyWordTwo. \imprimirKeyWordThree.
\end{otherlanguage*}
\vspace{\onelineskip}
\end{resumo}

\end{OnehalfSpacing}

%Sumário
%ver esse link para configuração de fonte: http://tex.stackexchange.com/questions/83377/how-to-change-chapter-font-style-in-the-middle-of-the-table-of-the-contents
\renewcommand*\contentsname{\centerline{SUMÁRIO}}
\pdfbookmark[0]{\contentsname}{toc}
\begin{SingleSpace} 
\tableofcontents*
\end{SingleSpace}
\cleardoublepage


\textual %Início dos Elementos Textuais

%\chapter*{Introdução}
\begin{flushleft}
	\vspace{1.2em}
	\textbf{\large INTRODUÇÃO}
	\vspace{2.9em}
\end{flushleft}
\thispagestyle{empty}

\addcontentsline{toc}{chapter}{INTRODUÇÃO}
\stepcounter{chapter} %incrementa o número do capítulo

Atualmente, estamos vivendo em meio a um mundo totalmente globalizado e extremamente conectado, isso afeta diretamente as nossas vidas, visto que as pessoas
anceiam por estar conectadas cada vez mais. Basta olharmos para o mercado de Telecomunicações e suas ofertas, que possuem como enfoque planos de dados e não mais
minutos de chamadas ou \textit{Short Messages} - SMS. Essas ofertas refletem a mudança de estilo de vida das pessoas nos últimos anos, demonstrando claramente
esse desejo e/ou necessidade de se manter conectado na grande rede de computadores, a internet. Toda essa conectividade impacta diretamente a vida social
das pessoas, pois muitas delas, deixam de se comunicar com seus amigos, familiares cara a cara, para se comunicar com eles ou outras pessoas por meio de mídias
sociais ou outras ferramentas \textit{online}.

Essa comunicação com pessoas em tempo real fora facilitado pela acessibilidade a tecnologia, segundo \citeonline{lima_mariana_number_of_smartphones_in_brazil} a projeção é que no Brasil haja mais de um \textit{smartphone} ativo por pessoa, esse valor corresponderia a mais de trezentos milhões de aparelhos. Além dos smartphones outras tecnologias que fazem parte do conceito de \textit{Internet of Things} - IOT (Internet das coisas)
veem se popularizando, aumentado ainda mais o tempo de conexão das pessoas a internet. Não é nada incomum hoje, encontrarmos casas que possuem um controle de seus
dispositivos eletrônicos realizados de forma totalmente remota.

Uma das ferramentas que ajudou na disseminação ou necessidade do aumento da conectividade dos usuários com a internet foram as mídias socias. Criadas a pouco mais de uma dezena de anos, as mídias sociais teve como ideal inicial prover um mecanismo para conectar pessoas estreitando os laços de amizades entre elas, removendo o impecílio da distância física entre elas. Porém, com o passar dos anos essas mídias socias se tornaram uma grande vitrine, e, algumas pessoas a utilizam como uma
ferramenta para gerar lucros. Visto que por meio delas as pessoas podem disponibilizar conteúdos de todos os tipos e, é claro dar sua opnião a respeito de um determinado assunto, sem nenhuma restrição. Portanto, muitas pessoas passaram a utilizar as mídias sociais para descrever seus dia-a-dia, e, até mesmo transcrever seus sentimentos, sejam eles bons ou ruins, assim como um antigo diário, deixando assim sua vida social totalmente exposta para as pessoas que se interessam pela sua vida pessoal. 

Atualmente as mídias socias como \textit{Facebook} e \textit{Twitter}, entre outras somam mais de três bilhões de usuários ativos, conforme \citeonline{kemp_simon_number_of_social_media_users}, esse número só tende a crescer nos próximos anos. Todo esse sucesso das redes sociais somado ao fato da facilidade de se expressar e comunicar com os demais usuários abriram oportunidades para as pessoas falarem, revelarem situações corriqueiras ou não que as incomodavam em sua vida, compartilhando assim experiências de vida. 

Isso se deve a facilidade e a capacidade de usuários de mídias socias falarem e serem ouvidos, esta por sua vez se tornou uma válvula de escape para problemas comportamentais, pessoais, financeiros e outros. Principalmente entre os jovens que são os principais alvos de \textit{Bullying}, seja ele na escola, na família ou na sociedade como um todo. \textit{Bullying} que se tornou um dos principais problemas enfrentados em escolas de todo o mundo, como apresentado por \citeonline{the_guardian_ten_years_of_bulyng_data}. 

Infelizmente  o \textit{Bullyng} veem se alastrando perante a sociedade e, de certa forma, acaba contribuindo para catástrofes, como atentados em casos extremos. Esse problema, é enfretado por milhares de pessoas, principalmente jovens no dia a dia, em muitos desses casos as próprias vítimas se sentem tão constrangidas que não conseguem falar a respeito disso com seus amigos próximos, familiares ou com algum tipo profissional.

Devido a essa dificuldade em conversar a respeito desses problemas e somado ao fato de redes sociais serem abertas em alguns casos, jovens costumam publicar mensagens ou pensamentos que poderiam ser rotulados como uma mudança de humor, como uma tristeza em alguns casos mais sérios, poderiam sugerir algum tipo de depressão e em casos extremos alguma tendência a
tirar a sua própria vida. Entre as redes sociais, hoje uma das mais utilizadas, e, com um grande número de usuários ativos é o \textit{Twitter}, isso se deve ao fato dele ser uma rede social aberta, onde o que é escrito por um usuário pode ser visto por qualquer outra pessoa conforme \citeonline{bridanne_stephen_twitter_research}.

Visto que esse problema é extramente sério e que, os casos são mais detectados em jovens estudantes esse trabalho pretende criar uma ferramenta utilizando algoritmos de \textit{Machine Learning} que será capaz de classificar um determinado \textit{tweet} entre três tipos possíveis, são eles: alto risco de ser real; risco de ser real; seguro para ser ignorado; ajudando assim escolas e instituições de ensinos, bem como os pais e familiares de alunos a identiticarem desvios de humor a fim de evitar finais trágicos e melhorar assim a vida social dos envolvidos.

%\par \citeonline[p. 2]{neo4j_team_manual} afirma que:

%\begin{citacao}
%	\textit{A single server instance can handle a graph of billions of nodes and relationships. When data throughput is insufficient, the graph database can be distributed among multiple servers in a high availability configuration.}\footnotemark[11]
%\end{citacao}
\chapter{BANCO DE DADOS}

A fim de implementar a ferramenta proposta por esta pesquisa, é necessário possuir um banco de dados, em algumas pesquisas o banco de dados é denominado \textit{dataset} contendo um determinado número de registros para que o algortimo utilize como base de treinamento e testes. Para moldar o \textit{dataset} necessário para esta pesquisa foi utilizada uma \textit{Application Program Interface} - API pública e aberta disponibilizada pelo próprio \textit{Twitter}. Essa API possibilita o monitoramente de menos de um por cento do total de \textit{tweets} postados em tempo real conforme \cite{natural_language_processing_mental_health}. Esses \textit{tweets} são filtrados de forma aleatória, não sendo possível definir especificamente os \textit{tweets} de uma só pessoa e levando em consideração uma série de parâmetros informados pelo usuário, parâmetros estes como: palavras chaves; expressões textuais; contendo uma imagem; entre muitas opções, segundo \citeonline{twitter_filter_realtime_doc} são até duzentos e cinquenta mil filtros possíveis.

Portanto, foi desenvolvida uma aplicação simples utilizando a linguagem de programação Java em sua versão \textit{Standard Edition} SE, cuja principal atividade é o consumo desta API do \textit{Twitter} e a armazenagem dos \textit{tweets} em um banco de dados \textit{Not Only SQL} - NOSQL, para essa pesquisa o banco de dados selecionado foi o MongoDB, devido a sua simplicidade para manipular objetos no formato JSON (\textit{Javascript Object Notation}) e somada ao fato de que a API do \textit{Twitter} utiliza este formato como resposta.

Para a coleta dos \textit{tweets} foram utilizados como filtro algumas expressões, são elas:
\newline
"suicidal"; "suicide"; "kill myself"; "my suicide note"; "my suicide letter"; "end my life"; "never wake up"; "can't go on";
"not worth living"; "ready to jump"; "sleep forever"; "want to die"; "be dead"; "better off without me"; "better off dead"; "suicide plan"; "suicide pact"; "tired of living"; "don't want to be here"; "die alone"; "go to sleep forever"; "bullied"; "bullyng";

Além é claro de \textit{tweets} com mensagens positivas, nesse caso as expressões usadas foram:
\newline
"be happy"; "happiness"; "enjoy the life"; "love my life"; "ready for news"; "new plans"; "vacation plan"; "be live";

Para que seja possível identificar de forma simples se um determinado \textit{tweet} é positivo ou não foi utilizada uma \textit{flag} booleana para esse fim, facilitando assim o filtro por meio do MongoDB.
\chapter{MODELAGEM}

Para a resolução do problema cujo este trabalho se propõe a resolver é necessário um processo de modelagem sob o \textit{dataset} seja aplicado. Este processo é composto de algumas fases, conforme apresentado a seguir:

\section{Pré processamento}

Nesta fase é necessário realizar o agrupamento de todos os \textit{tweets}, positivos e negativos de acordo com usuário. Obtendo como resultado uma lista de \textit{tweets} de acordo com um determinado usuário. Esse processo se torna indispensável, visto que a ideia dessa pesquisa é desenvolver um modelo cujo principal objetivo é identificar que um determinado usuário possui risco de um atentado suicida ou não de acordo com seus \textit{tweets}.

Após agrupar os \textit{tweets}, inicia-se o processo de manipulação dos conteúdos de cada \textit{tweet}. Para tanto, foi utilizado os seguintes processos:

\subsection{Remoção de elementos HTML}

O \textit{Twitter} possibilita aos seus usuários a utilização de elementos  \textit{HyperText Markup Language} - HTML nos \textit{tweets}. Portanto, é importante extrair e remover tais informações para manter apenas o que é realmente necessário, nesse caso, o conteúdo escrito pelo próprio usuário. Isso é claro, evitará processamento desnecessário e otimizará o próximo passo, a tokenização dos \textit{tweets}.

\subsection{Separação em palavras \textit{Tokenização}}

O processo denominado Tokenização consiste da separação de uma frase por palavras, removendo assim pontuações do texto. Portanto, como resultado do processo foi obtido um vetor de palavras. Após essa separação, é possível encontrar algumas palavras que não trazem muito valor ao contexto da frase, palavras como preposições, temos como exemplo: \textit{and}, \textit{of}, \textit{the}, \textit{to}, entre outras. A remoção desse tipo de palavras também é parte de todo o processo de tokenização.

\subsection{Conversão de textos para minúsculo}

A fim de simplificar o processo de análise do conteúdo de cada \textit{tweet} é necessário que todo o conteúdo do \textit{tweet} seja convertido para apenas um formato, sendo que para essa pesquisa a forma selecionada foi o \textit{lowercase}.

\subsection{Substituição de siglas por termos}

Sabe-se que a linguagem utilizada perante as mídias socias não são as mais coloquiais possíveis e em muitos casos, é possível encontrar várias siglas para se referenciar à algo ou alguém, para saudar alguém ou apenas por comodidade. Portando, é necessário que esse algoritmo avalie estas siglas e substitua por valores completos e coerentes ao conteúdo do \textit{tweet}.

\subsection{Extração de termos com \textit{hashtags}}

Um dos caracteres mais utilizados por usuários de mídias socias é o "\#" (\textit{hashtag}), pois ele é utilizado para enfatizar a ideia do \textit{post}. Por esta razão, extrair estas informações se torna muito relevante para a análise do \textit{tweet}, pois ele pode identificar uma possível ironia auxiliando no processo de classificação deste \textit{tweet}.

\section{\textit{Stemming} e \textit{Lemmatization}}

O processo de \textit{stemming} é usado para agrupar as palavras que possuem uma mesma origem, reduzindo a matriz de palavras extraídas pelo processo de \textit{tokenização}, conforme \cite[p.24]{development_of_stemming_algorithm}

\begin{citacao}
	\textit{"A stemming algorithm is a computational procedure which reduces all words with the same root (or, if prefixes are left untouched, the same stem) to a common form, usually by stripping each word of its derivational and inflectional suffixes."}
\end{citacao}

Portanto, esse processo auxilia na redução da matriz de palavras, uma vez que cada palavra pode pssuir diferentes formas e este processo almeja a identificação da forma mais próxima destas plavras, também conhecida como a forma mais genérica do termo, conforme \cite{influence_of_word_normalization_on_text_classification}. 

Outro processo muito útil na normalização dos termos é o \textit{Lemmatization} cujo o resultado obtido desse processo é o termo sem seu sufixo ou com outro segundo \cite{influence_of_word_normalization_on_text_classification}.

\subsection{Extração de \textit{emojis}}

Outro recurso muito utilizado nos dias atuais por usuários de mídias socias são os \textit{emojis}, por meio deles é possível identificar os sentimentos descritos pelo \textit{tweet}. Por esta razão, a extração deste recurso dos \textit{tweets} se fazem necessário, visto que eles trazem uma informação riquíssima a esta análise. Inclusive este recurso, foi imensamente importante a esta pesquisa, pois, através destes, foi realizada a análise de irônias nos \textit{tweets}, sendo assim: um \textit{tweet} proveniente da lista de palavras positivas e que possua ao menos um \textit{emoji}, cujo o mesmo pertença a uma lista de \textit{emojis} pré selecionado que emita algum sentimento de tristeza é identificado como uma ironia auxiliando na construção do modelo. O mesmo ocorre para \textit{tweets} provenientes da lista de palavras negativas.

\subsection{Representação}

Após todo o pré processamento conforme detalhado nas sub seções anteriores, o modelo preditivo final se baseia em uma representação binária do \textit{tweet} em um vetor de pares atributo-valor, onde os atributos são os tokens previamente definidos pela \textit{tokenização}, e seus valores podem são frequências binárias de palavras (caracteristicas) do modelo preditivo. Além desses \textit{tokens}, mais duas características foram adicionadas ao modelo, são elas: \textit{"hasEmoji"} e \textit{"isIronic"}. 

\subsection{Eliminação de características menos relevante}

A fim de reduzir a dimensionalidade do problema, foi utilizado uma função para avaliar quais as características são menos relevantes e assim remove-las reduzindo a dimensão do problema e consequentemente melhorando a performance do modelo e evitando possíveis problemas, tais como \textit{overfitting}. A função escolhida para este problema foi a PCA, segundo \citeonline{scikit_learn_pca}  essa função realiza a redução de características baseada no valor que uma determinada característica exerce sobre o problema.

\section{Modelo preditivo}

A fim de indentificar a possibilidade de um determinado \textit{tweet} ser suicida ou não, nesta pesquisa foi utilizado um método de classificação supervisionada, utilizando como entrada os próprios \textit{tweets} coletados, pré processados e rotuladosn(classificados) de acordo com as regras pré estabelecidas e descritas anteriormente. Essa abordagem foi necessária, pois, não tinhamos os recursos necessários para utilizar uma abordagem semi-supervisionada, visto que seu custo de implementação é alto. Para a utilização de tal abordagem precisariamos dispor de tempo e uma equipe de psicólogos para avaliar cada \textit{tweet} coletado para a geração do \textit{dataset} inicial o que seria extremamente custoso.

Para evitar essa abordagem de classificação manual, altamente custosa, foi desenvolvido uma aplicação utilizando \textit{Python} para classificar esses \textit{tweets} de forma automática, levando em consideração algumas regras, tais como, \textit{tweets} cuja finalidade não seja uma resposta a outro \textit{tweet}, contenham determinadas palavras (após todo o pré processamento) e que não possua ironias, baseando-se no uso de \textit{emoticons}.

Portanto, como entrada para o nosso modelo preditivo supervisionado, o mesmo \textit{dataset} coletado inicialmente foi utilizado, porém, os registros já haviam sido pré processados e rotulados, permitindo assim a utilização da abordagem de métodos supervisionados.

Como modelo final temos a seguinte estrutura de características:

\begin{table}[H]
	\caption{Modelo preditivo}
	\begin{tabular}{|l|l|l|l|l|l|l|l|l|l|l|l|l|l|l|l|l|l|l|}
		\hline
		\textbf{User} & \rotatebox{90}{\textbf{suicidal}} & \rotatebox{90}{\textbf{suicide}} & \rotatebox{90}{\textbf{kill}} & \rotatebox{90}{\textbf{myself}} & \rotatebox{90}{\textbf{end}} & \rotatebox{90}{\textbf{die}} & \rotatebox{90}{\textbf{dead}} & \rotatebox{90}{\textbf{bullied}} & \rotatebox{90}{\textbf{bullyng}} & \rotatebox{90}{\textbf{happy}} & \rotatebox{90}{\textbf{happiness}} & \rotatebox{90}{\textbf{enjoy}} & \rotatebox{90}{\textbf{love}} & \rotatebox{90}{\textbf{news}} & \rotatebox{90}{\textbf{live}} & \rotatebox{90}{\textbf{hasEmoji}} & \rotatebox{90}{\textbf{isIronic}} & \rotatebox{90}{\textbf{isSuicidal}}\\ \hline
		123456        & 0                 & 0                & 0             & 0               & 0            & 0            & 0             & 0                & 0                & 1              & 0                  & 0              & 1             & 0             & 0             & 1                 & 0                 & 0                   \\ \hline
		654321        & 0                 & 1                & 1             & 1               & 0            & 1            & 0             & 0                & 1                & 0              & 0                  & 0              & 0             & 0             & 0             & 0                 & 0                 & 1                   \\ \hline
		09876         & 1                 & 0                & 0             & 0               & 1            & 0            & 0             & 0                & 0                & 0              & 0                  & 0              & 0             & 0             & 1             & 1                 & 1                 & 0                   \\ \hline
		67890         & 0                 & 0                & 0             & 1               & 1            & 0            & 0             & 0                & 0                & 0              & 0                  & 0              & 1             & 0             & 1             & 0                 & 1                 & 1                   \\ \hline
	\end{tabular}
\end{table}

Esta é a representação gráfica de alguns registros utilizados para a geração de um data frame.
\chapter{PLANEJAMENTO}

Para esta pesquisa serão utilizados as seguintes técnicas de avaliação:

\begin{itemize}
	\item Acurácia
	\item Área sob a curva ROC
	\item Validação cruzada
\end{itemize}
%\chapter{OBJETIVOS}

\par Para que se possa levar esta proposta de pesquisa a cabo são colocados os seguintes objetivos:

\section{Objetivo geral}

\par Desenvolver uma aplicação \textit{web} utilizando um algoritmo de \textit{machine learning}, capaz de realizar a classificação de um determinado \textit{post}, ou \textit{tweet} como possível depressão ou mesmo uma mensagem suicida.

\section{Objetivos específicos}

\begin{itemize}
	\item Demostrar o uso de algoritmos de classificação para \textit{machine learning};
	\item Representar o problema utilizando \textit{machine learning};
	\item Projetar e implementar uma aplicação \textit{web} para classificação de \textit{tweets}.
\end{itemize}
 %Added by Edilson Justiniano on 18-03-2015 for "Pré-projeto"
%\chapter{JUSTIFICATIVA}

\par Por meio de uma pesquisa informal, realizada com o auxílio de formulários disponibilizados na internet, na região de Pouso Alegre, foi constatado que não há um sistema que possua como principal objetivo localizar determinados tipos de mão de obra nos quais não existam vínculos empregatícios.

\par Este projeto propõem atuar sobre esta limitação, desenvolvendo um software cujo principal objetivo é localizar e apresentar aos usuários, profissionais temporários que possuam credibilidade e boas referências.
	
\par A fim de tornar possível a realização do mesmo serão utilizadas tecnologias gratuitas e de boa aceitação pelo mercado. Desta forma, será possível reduzir o custo de desenvolvimento, possibilitando a sua distribuição aos usuários. Com esta distribuição, seus benefícios terão acesso a uma parcela maior de pessoas, aumentando consideravelmente a facilidade na busca por este tipo de profissional.
	
\par Este projeto também visa agregar valor acadêmico, proporcionando uma base de conhecimentos e explanação de tecnologias atuais e valorizadas, sendo que algumas não fazem parte do escopo do curso, como exemplo, o conceito de banco de dados orientado a grafos e o Neo4J.
	
\par Agregando todos estes benefícios à possibilidade de melhoria contínua e acréscimos de novas funcionalidades, este projeto servirá como base para novos trabalhos acadêmicos. %Added by Edilson Justiniano on 18-03-2015 for "Pré-projeto"
%% !TeX spellcheck = en_US
\chapter{QUADRO TEÓRICO}

\par Neste capítulo são discutidas as técnicas, metodologias e tecnologias que foram utilizadas no desenvolvimento deste trabalho.

\section{Iconix}

\par O ICONIX, segundo \citeonline{rosenberg_iconix_process}, foi criado em 1993 a partir de um resumo das melhores técnicas de desenvolvimento de \textit{software} utilizando como ferramenta de apoio a \textit{Unified Modeling Language} - UML\footnotemark[3]. Esta metodologia é mantida pela empresa ICONIX \textit{Software Engineering} e seu principal idealizador é Doug Rosenberg.

%Nota a respeito da sigla UML
\footnotetext[3]{UML: \textit{Unified Modeling Language} - Linguagem de modelagem para objetos do mundo real que habilita os desenvolvedores especificar, visualizar, construí-los a nível de software.}


\par Para \citeonline{rosenberg_scott_use_case_driven_object_modeling_with_uml}, o ICONIX possui como característica ser iterativo e incremental, somado ao fato de ser adequado ao padrão UML auxiliando, assim, o desenvolvimento e a documentação do sistema.

\par Atualmente, existem diversas metodologias de desenvolvimento de \textit{software} disponíveis, contudo, o ICONIX, em especial, será utilizado para auxiliar no processo de desenvolvimento deste trabalho pois, segundo
\citeonline{silva_videira_uml_metodologias_ferramentas_case}, essa metodologia nos permite gerar a documentação necessária para nortear o desenvolvimento de um projeto acadêmico.

\par De acordo com \citeonline{rosenberg_stephens_use_case_driven_object_modeling_with_uml}, os processos do ICONIX consistem em gerar alguns artefatos que correspondem aos modelos dinâmico e estático de um sistema e estes são elaborados e desenvolvidos de forma incremental e em paralelo, possibilitando ao analista dar maior ênfase no desenvolvimento do sistema do que na documentação do mesmo. A Figura~\ref{fig:visao_geral_iconix_componentes}, apresenta uma visão geral dos componentes do ICONIX.

% Imagem do Iconix
\begin{figure}[h!]
	\centerline{\includegraphics[scale=0.95]{./imagens/visao_geral_iconix.png}}
	\caption[Uma visão geral do ICONIX e seus componentes.]
	{Uma visão geral do ICONIX e seus componentes. \textbf{Fonte:} \citeonline{rosenberg_scott_use_case_driven_object_modeling_with_uml}.}
	\label{fig:visao_geral_iconix_componentes}
\end{figure}

\newpage %Pular pagina para que o texto fica abaixo da imagem na nova pagina

\par Ao utilizar essa metodologia, o desenvolvimento do projeto passa a ser norteado por casos de uso (\textit{use cases}) e suas principais fases são: análise de requisitos, análise e projeto preliminar, projeto e implementação. A seguir será apresentada uma breve descrição de cada uma das fases do ICONIX, seguindo as ideias de \citeonline{o_engenheiro_software_as_fases_iconix_mecva}.

\subsection{Análise de requisitos}

A função da fase de análise de requisitos é modelar os objetos do problema real a partir dos requisitos do software já levantados, e, a partir destes objetos gerar o diagrama de modelo de domínio. Ainda nesta fase, e, com base nos requisitos, deve-se definir os casos de uso do \textit{software}. Estes casos de uso são o elo entre os requisitos e a implementação propriamente dita do \textit{software}. Por isto, a definição destes diagramas se faz tão importante para o ICONIX. 

%A fase de análise de requisitos tem como finalidade identificar os objetos do problema real e, a partir destes objetos definir como eles serão abstraídos para o software por meio do modelo de domínio. Estes objetos são identificados a partir dos requisitos que foram levantados anteriormente. Também apresenta um protótipo das possíveis interfaces gráficas e, por fim, descrever os casos de uso. Quando possível, elabora-se os diagramas de navegação para que o cliente possa entender melhor o funcionamento do sistema como um todo.

\subsection{Análise e projeto preliminar}

Nesta fase deve-se detalhar todos os casos de uso identificados na fase de análise de requisitos, por meio de diagramas de robustez, baseando-se no texto dos casos de uso (fluxo de eventos). O diagrama de robustez não faz parte dos diagramas padrões da UML. Porém, este é utilizado para descobrir as classes de análise e detalhar superficialmente o funcionamento dos casos de uso.

Em paralelo, deve-se atualizar o modelo de domínio adicionando os atributos identificados às suas respectivas entidades, que foram descobertas na fase de análise de requisitos. A partir deste momento será possível gerar a base de dados do sistema.


\subsection{Projeto detalhado}

\par Na fase denominada projeto detalhado deve-se elaborar o diagrama de sequência com base nos diagramas de casos de uso identificados anteriormente, a fim de detalhar sua implementação.

O diagrama de sequência deve conter as classes que serão implementadas e as mensagens enviadas entre os objetos corresponderão às operações que realmente serão implementadas futuramente. Estas operações devem ser incluídas no modelo de domínio em conjunto com as novas classes do projeto identificadas, criando assim, o diagrama de classes final.

\subsection{Implementação}

Na fase de implementação deve-se desenvolver o código fonte do \textit{software} e os testes necessários para obter um software com qualidade. O ICONIX não define os passos a serem seguidos nesta fase, ficando a cargo de cada um definir a forma como será implementado o projeto.
 
Ao término de cada fase um artefato é gerado, sendo respectivamente: revisão dos requisitos, revisão do projeto preliminar, revisão detalhada e entrega.

O ICONIX é considerado um processo prático de desenvolvimento de software, pois a partir das iterações que ocorrem na análise de requisitos e na construção dos modelos de domínios (parte dinâmica), os diagramas de classes (parte estática) são incrementados e, a partir destes, o sistema poderá ser codificado.


Por proporcionar essa praticidade, o ICONIX será empregado para o desenvolvimento deste projeto, pois por meio dele é possível obter produtividade no desenvolvimento do \textit{software} ao mesmo tempo em que alguns artefatos são gerados, unindo o aspecto de abrangência e agilidade.


\section{Teoria dos Grafos}

\par A teoria dos grafos foi criada pelo matemático suíço Leonhard Euler no século XVIII com o propósito de solucionar um antigo problema, conhecido como as 7 pontes de \textit{Königsberg} \cite{harju_graph_theory}.

\par \textit{Königsberg}, atualmente conhecida como \textit{Kaliningrad}, era uma antiga cidade medieval cortada pelo rio \textit{Pregel} dividindo-a em 4 partes interligadas por 7 pontes. Ela era localizada na antiga Prússia, hoje, território Russo. O problema mencionado anteriormente consistia basicamente em atravessar toda a cidade, visitando todas as partes e utilizando todas as pontes desde que não repetisse uma das quatro partes ou uma das 7 pontes. A Figura~\ref{fig:problema_sete_pontes} ilustra o problema mencionado.

% Imagem do problema das 7 pontes da teoria dos grafos
\begin{figure}[h!]
	\centerline{\includegraphics[height=0.26\textheight,width=0.8\textwidth]{./imagens/Konigsberg_7_bridges.jpg}}
	\caption[O problema das 7 pontes de \textit{Königsberg} ]
	{O problema das 7 pontes de \textit{Königsberg}. \textbf{Fonte:} \citeonline{paoletti_seven_bridges_konigsberg}.}
	\label{fig:problema_sete_pontes}
\end{figure}

\par De acordo com \citeonline{bruggen_learning_neo4j}, para tentar solucionar o problema, Euler utilizou uma abordagem matemática ao contrário dos demais que tentaram utilizar a força bruta para solucionar tal problema, desenhando N números de diferentes possibilidades de rotas. Euler mudou o foco e passou a dar mais atenção ao número de pontes e não as partes da cidade. Por meio desta observação, foi possível perceber que realizar tal tarefa seria impossível, pois de acordo com sua teoria seria necessário possuir no mínimo mais uma ponte, uma vez que, o número de pontes era ímpar, não sendo possível realizar um caminho único e sem repetição. Desta forma, obteve-se a solução para este problema e criou-se o primeiro grafo no mundo.

\par \citeonline[p. 16]{rocha_algoritmos_particionamento_banco_dados_orientado_grafos} afirma que: 

\begin{citacao}
	um \textit{grafo G = (V,E)} consiste em um conjunto finito \textit{V} de vértices e um conjunto finito \textit{E} de arestas onde cada elemento \textit{E} possui um par de vértices que estão conectados entre si e pode ou não possuir um peso \textit{P}.
\end{citacao}

\par Esta é a definição formal de um grafo. A partir desta definição, é possível identificar, no problema mencionado anteriormente, os vértices que neste caso são as pontes e as arestas que por sua vez são as partes da cidade.

\par Segundo \citeonline{bondy_murty_graph_theory_with_applications}, muitas situações do mundo real podem ser descritas através de um conjunto de pontos conectados por linhas formando assim um grafo, como um centro de comunicações e seus \textit{links}, ou as pessoas e seus amigos, ou uma troca de emails entre pessoas, entre outras. Isto é possível pois, de acordo com \citeonline{rocha_algoritmos_particionamento_banco_dados_orientado_grafos}, existem muitos problemas atualmente que podem ser mapeados para uma estrutura genérica possibilitando assim utilizar a teoria de grafos para tentar solucioná-los, tais como: rotas geográficas, redes sociais, entre outros.

%BANCA_QUALIFICACAO. Comentado este parágrafo, porém o mesmo retornará para a banca de qualificação
\par A Figura~\ref{fig:ilustracao_grafo_simples} demostra de maneira visual um grafo, conforme ideia de \citeonline{bondy_murty_graph_theory_with_applications}, utilizando como exemplo o seguinte grafo \textit{G} = \{a, b, c, d, e, f, g, h\} e suas respectivas arestas \textit{E}$_g$ = \{(a, b), (a, h), (a, e), (b, f), (c, e), (c, d), (c, g), (d, e), (d, h), (d, g), (f, h)\}, sendo que os vértices serão representados por círculos e as arestas que os interligam por linhas.

%subscrito $_CARCTER_DESEJADO$

% Imagem do grafo simples - VOLTAR NA BANCA DE QUALIFICACAO
\begin{figure}[h!]
	\centerline{\includegraphics[scale=0.77]{./imagens/simple_graph.png}}
	\caption[Ilustração de uma representação gráfica de um simples grafo]
	{Ilustração de uma representação gráfica de um simples grafo. \textbf{Fonte:} \citeonline{rocha_algoritmos_particionamento_banco_dados_orientado_grafos}.}
	\label{fig:ilustracao_grafo_simples}
\end{figure}

%BANCA_QUALIFICACAO. Comentado este parágrafo, porém o mesmo retornará para a banca de qualificação
\par \citeonline{ruohonen_graph_theory} afirma que os grafos podem ser gerados com a possibilidade de permitir \textit{loops}\footnotemark[4] e arestas paralelas ou multiplas entre os vértices, obtendo um \textit{multigraph}. A Figura~\ref{fig:ilustracao_multigrafo_simples} ilustra um simples \textit{multigraph}.

\footnotetext[4]{\textit{loops} - Uma aresta que interliga o mesmo vértice.}

% Imagem de um multigraph - VOLTAR NA BANCA DE QUALIFICACAO
\begin{figure}[h!]
	\centerline{\includegraphics[scale=0.9]{./imagens/multigraph_example.png}}
	\caption[Ilustração de uma representação gráfica de um \textit{multigraph}]
	{Ilustração de uma representação gráfica de um \textit{multigraph}. \textbf{Fonte:} Adaptado de \citeonline{harju_graph_theory}.}
	\label{fig:ilustracao_multigrafo_simples}
\end{figure}

\newpage

%BANCA_QUALIFICACAO. Comentado este parágrafo, porém o mesmo retornará para a banca de qualificação
\par Para \citeonline{harju_graph_theory}, os grafos podem ser direcionados (\textit{dígrafo}) ou não direcionados. Os  direcionados são aqueles cujos vértices ligados a uma aresta são ordenados e permitem que uma aresta que conecta os vértices \textit{x} e \textit{y} seja representada apenas de uma forma, sendo ela \{x, y\} ou \{y, x\}, ao contrário dos não direcionados que, para este mesmo caso, podem ser representado por ambas as formas \cite{rocha_algoritmos_particionamento_banco_dados_orientado_grafos}. A Figura~\ref{fig:ilustracao_grafo_direcionado} demonstra um grafo direcionado.

% Imagem de um grafo direcionado - VOLTAR NA BANCA DE QUALIFICACAO
\begin{figure}[h!]
	\centerline{\includegraphics[scale=0.6]{./imagens/simple_digraph_graph.png}}
	\caption[Imagem de uma representação gráfica de um grafo direcionado]
	{Imagem de uma representação gráfica de um grafo direcionado. \textbf{Fonte:} \citeonline{robert_keller_acylic_graph}.}
	\label{fig:ilustracao_grafo_direcionado}
\end{figure}

Segundo \citeonline{harju_graph_theory}, os tipos de grafos são:

\begin{itemize}
	\item \textbf{grafo simples:} são aqueles grafos que não possuem \textit{loops} ou arestas paralelas;
	
	\item \textbf{grafo completo:} são aqueles em que, qualquer par de vértices são adjacentes;
	
	\item \textbf{subgrafos:} são pequenos grafos que em conjunto constituem um grafo maior.
	
	\item \textbf{grafos isomórficos:} dois grafos são isomórficos se, ambos possuírem a mesma estrutura de nós, e relacionamentos, exceto pelos identificadores de cada nó que podem ser diferentes, veja na Figura~\ref{fig:ilustracao_grafo_isomorfico} um exemplo de dois grafos isomórficos \cite{harju_graph_theory}.
	
	\begin{figure}[h!]
		\centerline{\includegraphics[scale=1]{./imagens/grafos_isomorficos.png}}
		\caption[Imagem de um grafo isomórfico]
		{Imagem de um grafo isomórfico. \textbf{Fonte:} \citeonline{harju_graph_theory}.}
		\label{fig:ilustracao_grafo_isomorfico}
	\end{figure}
	
	\item \textbf{caminho (travessia):} é uma sequência de vértices \{v1, v2,...,vn\} conectados por meio de arestas. Exemplo: \{e1 = \{v1, v2\}, \{v1, v3\}..., \{vn, vm\}\};

	\item \textbf{grafo conexo:} são aqueles que, para qualquer par de vértices, há um caminho que os ligam;
	
	\item \textbf{grau do vértice:} é definido pela quantidade de arestas que se conectam ao vértice.
	
\end{itemize}

Para \citeonline{rocha_algoritmos_particionamento_banco_dados_orientado_grafos} existem várias formas de se representar um grafo computacionalmente utilizando diferentes estruturas de dados. Entretanto, a mais utilizada e simples é a matriz adjacência.

Uma matriz adjacência consiste em uma matriz contendo o mesmo número de linhas e colunas (\textit{n x n}). Veja na Figura~\ref{fig:ilustracao_matriz_adjacencia} um exemplo de representação de um grafo utilizando esta estrutura.

\begin{figure}[h!]
	\centerline{\includegraphics[scale=0.7]{./imagens/matriz_adjacencia.png}}
	\caption[Grafo representado por meio de uma matriz adjacência]
	{Grafo representado por meio de uma matriz adjacência. \textbf{Fonte:} \citeonline{rocha_algoritmos_particionamento_banco_dados_orientado_grafos}.}
	\label{fig:ilustracao_matriz_adjacencia}
\end{figure}

Na Figura~\ref{fig:ilustracao_matriz_adjacencia}, as posições da matriz cujo o valor é igual a 1, definem que há uma aresta conectando os vértices, tornando-os assim, adjacentes. Caso o valor seja 0, os vértices não estão conectados entre si no grafo e, portanto, não são vértices adjacentes.

\par Este conteúdo teórico foi escolhido para ser utilizado neste trabalho pois este visa equacionar o problema relacionado à busca por mão de obra, através do modelo utilizado pelas redes sociais. Isto é possível pois, como mencionado anteriormente por meio desta teoria, é possível descrever várias situações do mundo real e como ela é muito bem aplicada a redes sociais, inclusive grandes empresas desta área já a utilizam. Devido a esses motivos, a teoria dos grafos foi utilizada para auxiliar no desenvolvimento deste trabalho.



\section{Tecnologias}

\par Nesta seção serão abordadas as linguagens de programação e as tecnologias que foram utilizadas no desenvolvimento deste.

\subsection{Banco de dados}

\par A expressão ''banco de dados'' teve origem a partir do termo inglês \textit{Databanks}, que foi substituído, mais tarde, pela palavra \textit{Databases} (Base de dados)  por possuir um significado mais apropriado \cite {setzer_silva_banco_dados_aprenda_o_que_sao_melhore_conhecimento}.

\par De acordo com \citeonline{date_introducao_sistemas_bancos_dados}, um banco de dados é uma coleção de dados persistentes, usada pelos sistemas de aplicação em uma determinada empresa. Sendo assim, um banco de dados é um local onde são armazenados os dados necessários para manter as atividades de determinadas organizações.

\par Um banco de dados possui, implicitamente, as seguintes propriedades: representa aspectos do mundo real; é uma coleção lógica de dados que possuem um sentido próprio e armazena dados para atender uma necessidade específica. O tamanho do banco de dados pode ser variável, desde que ele atenda às necessidades dos interessados em seu conteúdo \cite{elmasri_navathe_sistemas_banco_dados}.

\par A escolha do banco de dados que será utilizado em um projeto é uma decisão importante e que deve ser tomada na fase de planejamento, pois determina características da futura aplicação, como a integridade dos dados, o tratamento de acesso de usuários, a forma de realizar uma consulta e o desempenho. Portanto, essa decisão deve ser bem analisada, levando-se em consideração o tipo de \textit{software} e o ambiente de produção em que será utilizado.

\par Nas seções seguintes, são demonstrados os principais modelos de banco de dados, abordando suas características.

%\subsubsection{Tipos de bancos de dados}

\par A escolha do banco de dados que será utilizado em um projeto é uma decisão importante e que deve ser tomada na fase de planejamento, pois determina características da futura aplicação, como a integridade dos dados, o tratamento de acesso de usuários, a forma de realizar uma consulta, o desempenho. Portanto, essa decisão deve ser bem analisada, levando em consideração o tipo de software e no ambiente de produção que será utilizado.

\par A seguir, são demonstrados os principais modelos de banco de dados, abordando suas características.

\subsection{Banco de dados relacionais}

\par O modelo de banco de dados relacional foi introduzido em 1970, por Edgar Frank Codd, em uma publicação com o título: “A relational model of data for large shared data banks”, na revista \textit{Association for Computing Machinery} (ACM). Essa publicação demonstrou como tabelas podem ser usadas para representar objetos do mundo real e como os dados podem ser armazenados para os objetos. Neste conceito, a integridade dos dados foi levada mais a sério do que em qualquer modelo de banco de dados antes visto. A partir desta publicação, surgiram muitos bancos de dados que passaram a utilizar este conceito e se tornaram muito utilizados no desenvolvimento de aplicações
\cite{matthew_stones_beginning_databases_with_postgresql}.

\par Segundo \citeonline{matthew_stones_beginning_databases_with_postgresql}, o conceito é baseado na teoria reacional da matemática e por isso há uma grande flexibilidade para o acesso e a manipulação de dados que são gravados no banco. Utilizam-se técnicas simples, como normalização na modelagem do banco de dados, criando várias tabelas relacionadas, que servem como base para consultas usando uma linguagem de consulta quase padronizada, a \textit{Structured Query Language} – SQL\footnotemark[5].

\footnotetext[5]{SQL: \textit{Structured Query Language} - Linguagem para consultas e alterações em bancos de dados.}

% Pode voltar mais a frente no TCC quando for necessário aumentar esta parte
%\par Ainda segundo \citeonline{matthew_stones_beginning_databases_with_postgresql}, um banco de dados relacional contém relações (tabelas) com atributos (colunas) e tuplas (linhas). Todo atributo possui um tipo de dado predefinido, uma tupla representa um conjunto de dados contendo um valor para cada atributo da linha e as tabelas são relacionadas através de chaves.

\par A utilização de banco de dados relacionais geraram a necessidade de dividir os dados agregados utilizados na aplicação em várias relações conforme as regras da normalização. Para recuperar o mesmo dado agregado são necessárias consultas utilizando \textit{joins}\footnotemark[6], uma operação que, dependendo do tamanho das relações e da quantidade de dados, pode não ser tão eficiente. Nos casos em que se precisa obter uma resposta rápida de um sistema, isso se torna uma desvantagem \cite{sadalage_fowler_nosql_distilled_brief_guide}.

\footnotetext[6]{\textit{joins} - função utilizada para realizar a junção entre tabelas, facilitando a busca em bancos de dados relacionais.}

% Pode voltar mais a frente no TCC quando for necessário aumentar esta parte
%\par A figura 6 ilustra bem esta situação de divisão de um agregado e as suas relações resultantes.

% Imagem do exemplo de join - PODE VOLTAR MAIS TARDE NO TCC (Confirmar com o Márcio)
%\begin{figure}[h!]
	%\centerline{\includegraphics[scale=0.8]{./imagens/example_joins_sadalage.png}}
	%\caption[Agregado no UI é conjunto de várias tuplas de várias tabelas]
	%{Agregado no UI é conjunto de várias tuplas de várias tabelas. \textbf{Fonte:} \citeonline[p. 29]{sadalage_fowler_nosql_distilled_brief_guide}}
	%\label{fig:exemplo1}
%\end{figure}

\par Este foi um dos fatores determinantes que motivaram a criação de novas tecnologias, a fim de sanar o problema mencionado acima. A partir desta motivação, foram desenvolvidos novos modelos de banco de dados, que serão apresentados a seguir.


\subsection{Banco de dados NoSQL}

\par A expressão NoSQL é um termo não definido claramente. Ela foi ouvida pela primeira vez em 1998 como um nome para o banco de dados relacional de Carlo Strozzi, que assim o nomeou por não fornecer uma SQL-API. O mesmo termo foi usado como nome do evento NoSQL Meetup em 2009, que teve como objetivo a discussão sobre sistemas de bancos de dados distribuídos.

\par Devido à explosão de conteúdos na \textit{web} no início do século XXI, houve a necessidade de substituir os bancos de dados relacionais por bancos que oferecessem maior capacidade de otimização e performance, a fim de suportar o grande volume de informações eminentes a esta mudança \cite{bruggen_learning_neo4j}.

\par \citeonline[p. 27]{rocha_algoritmos_particionamento_banco_dados_orientado_grafos} afirma que NoSQL é "um acrônimo para Not only SQL, indicando que esses bancos não usam somente o recurso de Structured Query Language (SQL), mas outros recursos que auxiliam no armazenamento e na busca de dados em um banco  não relacional".

\par Segundo \citeonline{bruggen_learning_neo4j}, os banco de dados NoSQL podem ser categorizados de 4 maneiras diferentes, são elas: \textit{Key-Value stores}\footnotemark[7], \textit{Column-Family stores}\footnotemark[8], \textit{Document stores}\footnotemark[9] e \textit{Graph Databases}\footnotemark[10].

\footnotetext[7]{\textit{Key-Value stores} - armazenamento por um par de chave e valor.}

\footnotetext[8]{\textit{Column-Family stores} - armazenamento por colunas e linhas.}

\footnotetext[9]{\textit{Document stores} - armazenamento em arquivos.}

\footnotetext[10]{\textit{Graph Databases} - banco de dados orientado a grafo.}
%\begin{itemize}
%\item \textit{Key-Value stores};
%\item \textit{Column-Family stores};
%\item \textit{Document stores};
%\item \textit{Graph Databases}.
%\end{itemize}

\par De acordo com \citeonline{bruggen_learning_neo4j}, o banco de dados orientado a grafo (\textit{graph database}) pertence a categoria NoSQL, contudo, ele possui particularidades que o torna muito diferente dos demais tipos de bancos de dados NoSQL. A seguir, será descrito com maiores detalhes o banco de dados orientado a grafos Neo4j.


\subsection{Neo4j}

\par O Neo4j foi criado no início do século XXI por desenvolvedores que queriam resolver um problema em uma empresa de mídias. Porém, eles não obtiveram êxito ao tentar resolver tal problema utilizando as tecnologias tradicionais, portanto, decidiram arriscar e criar algo novo. A princípio, o Neo4j não era um sistema de gerenciamento de banco de dados orientado a grafos como é conhecido nos dias atuais. Ele era mais parecido com uma \textit{graph library} (biblioteca de grafo) em que as pessoas poderiam usar em seus projetos \cite{bruggen_learning_neo4j}.

\par De acordo com \citeonline{bruggen_learning_neo4j}, inicialmente ele foi desenvolvido para ser utilizado em conjunto com alguns bancos de dados relacionais como MySQL e outros, com a intenção de criar uma camada de abstração dos dados em grafos. Mas com o passar dos anos, os desenvolvedores decidiram tirar o Neo4j da estrutura dos bancos relacionais e criar sua própria estrutura de armazenamento em grafos.

\par O Neo4j, como vários outros, também é um projeto de sistema de gerenciamento de banco de dados NoSQL de código fonte aberto.

\par Segundo \citeonline{robinson_webber_eifrem_graph_databases}, os bancos de dados orientados a grafos possuem como diferencial a sua performance, agilidade e flexibilidade. Entretanto, a performance é o que mais se destaca entre eles, pois, a maneira como armazenam e realizam buscas no banco de dados é diferente dos bancos convencionais. Primeiramente, esse tipo de banco de dados não utiliza tabela; ele armazena os dados em vértices e arestas. Isso permite realizar buscas extremamente velozes através de \textit{traversals} (travessias), uma vez que estas implementam algoritmos para otimizar tais funcionalidades, evitando assim o uso de \textit{joins} complexos, tornando-o tão veloz.

\par \citeonline[p. 2]{neo4j_team_manual} afirma que:

\begin{citacao}
	\textit{A single server instance can handle a graph of billions of nodes and relationships. When data throughput is insufficient, the graph database can be distributed among multiple servers in a high availability configuration.}\footnotemark[11]
\end{citacao}

\footnotetext[11]{Um único servidor pode manipular um grafo de bilhões de nós e relacionamentos. Quando a taxa de transferência de dados é insuficiente, o banco de dados orientado a grafo pode ser distribuído entre vários servidores mantendo a mesma velocidade de processamento.}

\par Com estas informações, é possível mensurar o quanto o Neo4j pode ser rápido e robusto, sendo possível, até mesmo distribuí-lo a fim de obter uma melhor configuração, organização e facilidade de manutenção.

%BANCA_QUALIFICACAO. Comentado este parágrafo, porém o mesmo retornará para a banca de qualificação
\par Segundo \citeonline{neo4j_team_manual}, o banco de dados Neo4j é composto por nós (vértices), relacionamentos (arestas) e propriedades. Os relacionamentos são responsáveis por organizar os nós e ambos podem possuir seus atributos. É possível realizar as buscas e/ou alterações no Neo4j de duas formas diferentes. Sendo a primeira através da API \textit{Cypher Query Language}, que é uma \textit{query language} para banco de dados orientado a grafos muito próxima da linguagem humana, cuja descrição completa será apresentada a seguir. A segunda é o \textit{framework\footnotemark[12] Traversal} que utiliza a API \textit{Cypher} internamente para navegar pelo grafo.

\footnotetext[12]{\textit{Framework} - Abstração que une códigos comuns entre vários projetos de software, a fim de obter uma funcionalidade genérica.}


\citeonline{rocha_algoritmos_particionamento_banco_dados_orientado_grafos} afirma que o Neo4j permite criar mais de um relacionamento entre o mesmo par de vértices, desde que estes sejam de tipos distintos. Isso possibilita navegar pelos vértices do grafo de forma mais rápida devido a esses diferentes tipos de arestas, o que torna possível implementar o algoritmo de busca desejado. A Figura~\ref{fig:grafo_simples_neo4j} exemplifica um simples grafo utilizando um banco de dados Neo4j.

\begin{figure}[h!]
	\centerline{\includegraphics[scale=0.8]{./imagens/simple_graph_neo4j.png}}
	\caption[Exemplo simples de um grafo armazenado no Neo4j]
	{Exemplo simples de um grafo armazenado no Neo4j. \textbf{Fonte:} \citeonline{neo4j_team_manual}.}
	\label{fig:grafo_simples_neo4j}
\end{figure}

%BANCA_QUALIFICACAO. Comentado este parágrafo, porém o mesmo retornará para a banca de qualificação
\par Há duas formas de executar o Neo4j, segundo \citeonline{robinson_webber_eifrem_graph_databases}. A primeira é conhecida como \textit{Server} e a segunda \textit{Embedded}. O modo \textit{Server} é utilizado principalmente em \textit{web-service} em conjunto com a API REST, este será aplicado neste trabalho. Já no modo \textit{Embedded} o banco de dados é executado embarcado à aplicação Java.

%BANCA_QUALIFICACAO. Comentado este parágrafo, porém o mesmo retornará para a banca de qualificação
\par Conforme \citeonline{neo4j_team_manual}, o Neo4j possui suporte as transações ACID (com atomicidade, consistência, isolamento e durabilidade).

O Neo4j é distribuído em duas versões sendo elas a \textit{Entreprise} e a \textit{Community}. A primeira possui um tempo de avaliação de 30 dias e após esse tempo, é necessário comprar uma licença para continuar a utilizá-lo. Como diferencial essa versão possui ferramentas para o gerenciamento do banco de dados, incluindo melhorias relacionadas à escalabilidade. A segunda versão é disponibilizada gratuitamente sem data limite de expiração, contudo, ela não possui os recursos mencionados anteriormente que estão presentes na versão \textit{Enterprise}, mas é muito utilizada para fins didáticos e para pequenos projetos, aplicando-se perfeitamente a este projeto \cite{neo4j_team_manual}.

\par Por ser um banco de dados orientado a grafo bastante robusto, seguro e possuir uma documentação de fácil entendimento, além, é claro, de possuir um baixo custo de implantação devido a sua licença \textit{open source}, esse banco de dados foi escolhido para ser utilizado neste trabalho.


\subsection{\textit{Cypher Query Language}}

\par O \textit{Cypher Query Language} é uma linguagem para consultas em banco de dados orientado a grafo específica para o banco Neo4j. Ela foi criada devido à necessidade de manipular os dados e realizar buscas em grafos de uma forma mais simples, uma vez que, não é necessário escrever \textit{traversals} (\textit{travessias}) para navegar pelo grafo \cite{neo4j_team_manual}.


\par \citeonline{robinson_webber_eifrem_graph_databases}, afirmam que o \textit{Cypher} foi desenvolvido para ser uma \textit{query language} que utiliza uma linguagem formal, permitindo a um ser humano entendê-la. Desta forma, qualquer pessoa envolvida no projeto é capaz de compreender as consultas realizadas no banco de dados. 

%BANCA_QUALIFICACAO. Comentado este parágrafo, porém o mesmo retornará para a banca de qualificação
\par Segundo \citeonline{neo4j_team_manual}, o \textit{Cypher} foi inspirado em uma série de abordagens e construído sob algumas práticas já estabelecidas, inclusive a SQL. Por este motivo, é possível notar que ele utiliza algumas palavras reservadas que são comuns na SQL como \textit{WHERE} e \textit{ORDER BY}.

%BANCA_QUALIFICACAO. Comentado este parágrafo, porém o mesmo retornará para a banca de qualificação
\par De acordo com \citeonline{neo4j_team_manual}, o \textit{Cypher} é composto por algumas cláusulas, dentre elas, se destacam:

%%BANCA_QUALIFICACAO. Comentado estes itens, porém os mesmos retornarão para a banca de qualificação
\begin{itemize}
	\item \textit{START}: define um ponto inicial para a busca, esse ponto pode ser um relacionamento ou um nó.
	\item \textit{MATCH}: define o padrão de correspondência entre os nós. Para identificar um nó é necessário incluí-lo entre um par de parenteses, e os relacionamentos são identificados um hífen e um sinal de maior ou menor.
	\item \textit{CREATE}: utilizado para criar nós e relacionamentos no grafo.
	\item \textit{WHERE}: define um critério de busca.
	\item \textit{RETURN}: define quais nós, relacionamentos e propriedades de ambos devem ser retornados da \textit{query} realizada. 
	\item \textit{SET}: utilizado para editar as propriedades de um nó ou de um relacionamento.
	\item \textit{UNION}: possibilita juntar o resultado de duas ou mais consultas.
	\item \textit{FOREACH}: realiza uma ação de atualização para cada elemento na lista.
\end{itemize} 

\par Outras \textit{Query Languages} existem, inclusive com suporte ao Neo4j, porém devido às vantagens apresentadas acima, somada ao fato de que ele possui uma curva de aprendizagem menor e é excelente para lhe oferecer uma base a respeito de grafos,  este \textit{framework} será utilizado para realizar as tarefas de manipulação dos dados no banco de dados.


\subsection{Java}

\par Segundo \citeonline{schildt_java_complete_reference}, a primeira versão da linguagem Java foi criada por James Gosling, Patrick Naughton, Chris Warth, Ed Frank e Mike Sheridan na \textit{Sun Microsystems} em 1991 e denominada "Oak" tendo como principal foco a interatividade com a TV. Mais tarde, em 1995, a \textit{Sun Mycrosystems} renomeia esta linguagem e anuncia publicamente a tecnologia Java, focando nas aplicações \textit{web}, que em pouco tempo e devido à grande ascensão da internet, cresceu e se mantém em constante evolução até os dias atuais.

\par De acordo com a \citeonline{oracle_about_java_technology}, a tecnologia Java não é apenas uma linguagem, mas também uma plataforma, que teve como modelo uma outra linguagem, o C++ que, por sua vez foi derivada da linguagem C. O C++ e o Java possuem em comum o conceito de orientação a objetos, o que permite a esta linguagem utilizar recursos como: generalização (herança), implementação, polimorfismo, entre outras. Tais funcionalidades permitem ao desenvolvedor escrever códigos reutilizáveis, a fim de facilitar o desenvolvimento do projeto.

\par Para \citeonline{schildt_java_complete_reference}, o paradigma de orientação a objetos foi criado devido às limitações que o conceito estrutural apresentava quando era utilizado em projetos de grande porte, dificultando o desenvolvimento e manutenção dos mesmos. Este paradigma possibilita ao desenvolvedor aproximar o mundo real ao desenvolvimento de \textit{software}, deixando os objetos do mundo real semelhantes a seus respectivos objetos da computação, possibilitando ao desenvolvedor modelar seus objetos de acordo com suas necessidades \cite{tcc_univas_faria_aspectj_programacao_orientada_aspecto_java}.

Segundo \citeonline{schildt_java_complete_reference}, o recurso denominado generalização (herança) permite ao desenvolvedor criar uma classificação hierárquica de classes. Além disso, é possível escrever uma classe genérica contendo comportamentos comuns, e as demais classes, cujos comportamentos também serão aplicados a ela, somente precisa generalizar esta classe.

O polimorfismo se refere ao princípio da biologia em que um organismo pode ter diferentes formas ou estados. Esse mesmo princípio também pode ser aplicado à programação orientada a objeto. Desta forma, é possível definir comportamentos que serão compartilhadaos entre as classes e suas respectivas subclasses, além de comportamentos próprios que apenas as sub classes possuem. Com isso, o comportamento pode ser diferente de acordo com a forma e/ou o estado do objeto \cite{polymorphism_oracle}.

\par Retomando a ideia da \citeonline{oracle_about_java_technology}, uma das vantagens da tecnologia Java sob as demais é o fato dela ser multiplataforma, possibilitando ao desenvolvedor escrever o código apenas uma vez e este, ser executado em qualquer plataforma inclusive em hardwares com menor desempenho. Isso é possível, pois, para executar um programa em Java é necessário possuir uma \textit{Java Virtual Machine} - JVM\footnotemark[13] - instalada no computador. A JVM compreende e executa apenas \textit{bytecodes}\footnotemark[14] e estes por sua vez são obtidos através do processo de compilação do código escrito em Java.

%Nota a respeito da sigla JVM
\footnotetext[13]{JVM: \textit{Java Virtual Machine} - Máquina virtual utilizada pela linguagem Java para execução e compilação de \textit{softwares} desenvolvidos em Java.}

%Nota a respeito de bytecodes
\footnotetext[14]{\textit{Bytecode} é o código interpretado pela JVM. Ele é obtido por meio do processo de compilação de um programa java como mencionado anteriormente.}

\par Todo programa que utiliza Java necessita passar por algumas etapas essenciais. Conforme ilustrado na Figura~\ref{fig:processo_compilacao_java}, o código é escrito em arquivo de texto com extensão \texttt{.java}, após isso ele será compilado e convertido para um arquivo com extensão \texttt{.class}, cujo o texto é transformado em \textit{bytecodes}. Este arquivo com extensão \texttt{.class} é interpretado pela JVM que é responsável por executar todo o código do programa.

% Imagem do Processo de compilação do Java
\begin{figure}[h!]
	\centerline{\includegraphics[scale=1]{./imagens/processo_compilacao_java.png}}
	\caption[Uma visão geral do processo de desenvolvimento de software.]
	{Uma visão geral do processo de desenvolvimento de software. \textbf{Fonte:} \citeonline{oracle_about_java_technology}.}
	\label{fig:processo_compilacao_java}
\end{figure}

\par Por todas as vantagens descritas anteriormente, foi empregada esta tecnologia neste trabalho.


\subsection{Tomcat 7}

\par O Tomcat é uma aplicação \textit{container} capaz de hospedar aplicações \textit{web} baseadas em Java. A princípio ele foi criado para executar \textit{servlets}\footnotemark[15] e \textit{JavaServer Pages} - JSP\footnotemark[16]. Inicialmente ele era parte de um sub projeto chamado \textit{Apache-Jakarta}, porém, devido ao seu sucesso ele passou a ser um projeto independente, e hoje, é responsabilidade de um grupo de voluntários da comunidade \textit{open source} do Java \cite{vukotic_goodwill_apache_tomcat_7}.

\footnotetext[15]{\textit{Servlet} - Programa Java executado no servidor, semelhante a um \textit{applet}.}

\footnotetext[16]{JSP: \textit{JavaServer Pages} - Tecnologia utilizada para desenvolver páginas interativas utilizando Java \textit{web}.}

\par Segundo a \citeonline{apache_about_tomcat}, o Tomcat é um software que possui seu código fonte aberto e disponibilizado sob a \textit{Apache License Version 2}. Isto o fez se tornar uma das aplicações \textit{containers} mais utilizadas por desenvolvedores.

\par \textit{Containers} são aplicações que são executadas em servidores e possuem a capacidade de hospedar aplicações desenvolvidas em Java \textit{web}. O servidor ao receber uma requisição do cliente, entrega esta ao \textit{container} no qual é distribuído. O \textit{container}, por sua vez, entrega ao \textit{servlet} as requisições e respostas HTTP\footnotemark[17] e inicia os métodos necessários do \textit{servlet} de acordo com o tipo de requisição realizada pelo cliente \cite{basham_sierra_bates_use_cabeca_servlets_jsp}.

\footnotetext[17]{HTTP: \textit{Hypertext Transfer Protocol} - Protocolo de transferência de dados mais utilizado na rede mundial de computadores.}

%BANCA_QUALIFICACAO. Comentado este parágrafo, porém o mesmo retornará para a banca de qualificação
\par \citeonline{brittain_darwin_apache_tomcat_2nd_edition} afirmam que o Tomcat foi desenvolvido utilizando a linguagem de programação Java, sendo necessário possuir uma versão do \textit{Java Runtime Environment} - JRE\footnotemark[18] - instalada e atualizada para executá-lo.

\footnotetext[18]{JRE: \textit{Java Runtime Environment} - Conjunto de ferramentas necessárias para a execução de aplicações desenvolvidas na linguagem Java.}
%BANCA_QUALIFICACAO. Comentado este parágrafo, porém o mesmo retornará para a banca de qualificação
\par De acordo com  \citeonline{laurie_laurie_apache_the_definitive_guide}, o Tomcat é responsável por realizar a comunicação entre a aplicação e o servidor Apache\footnotemark[19] por meio do uso de \textit{sockets}.

\footnotetext[19]{Apache - Servidor cujo aplicação Tomcat é executada.}

\par Assim como outros \textit{containers}, \citeonline{basham_sierra_bates_use_cabeca_servlets_jsp} afirmam que o Tomcat oferece gerenciamento de conexões \textit{sockets}, suporta \textit{multithreads}, ou seja, ele cria uma nova \textit{thread} para cada requisição realizada pelo cliente e gerencia o acesso aos recursos do servidor, além de outras tarefas.

\par O Tomcat, em especial, foi escolhido para ser utilizado neste trabalho, pois o objetivo é desenvolver uma aplicação \textit{web} e para hospedá-la em um servidor, uma aplicação \textit{container} se faz necessária. Por este motivo, e somado a sua facilidade de configuração, além das vantagens acima descritas tal decisão foi tomada.


\subsection{\textit{Web Service} REST}

A definição computacional de serviço é um \textit{software} que disponibiliza sua funcionalidade por meio de uma interface denominada contrato de serviço \cite{erl_soa_with_rest}.

\textit{Web Service} de acordo com \citeonline{marzulo_soa_na_pratica}, é uma materialização da ideia de um serviço que é disponibilizado na internet, e que, devido a isso, pode ser acessado em qualquer lugar do planeta e por diferentes tipos de dispositivos. Para ter acesso aos serviços que o \textit{Web Service} disponibiliza, o solicitante envia requisições de um tipo anteriormente definido e recebe respostas síncronas ou assíncronas.

\citeonline{marzulo_soa_na_pratica} afirma que, a implementação de um \textit{Web Service} é relativamente simples, uma vez que, há inúmeras ferramentas que facilitam a implementação do mesmo. Outro fator que permite a um \textit{Web Service} ser mais dinâmico é possuir uma estrutura interna fracamente acoplada, permitindo, assim, mudanças em suas estruturas sem afetar a utilização pelo cliente.

\citeonline{erl_soa_with_rest} afirmam que o REST\footnotemark[20], é uma das várias implementações utilizadas para criar serviços. Outra implementação também muito conhecida é: a SOAP\footnotemark[21] em conjunto com o WSDL\footnotemark[22], cujo responsabilidade é definir o contrato dos serviços.

\footnotetext[20]{REST: \textit{Representational State Transfer} - Tecnologia utilizada por \textit{web services}}

\footnotetext[21]{SOAP: \textit{Simple Object Access Protocol} - Tecnologia utilizada por \textit{web services} anteriores aos \textit{Web services} REST.}

\footnotetext[22]{WSDL: \textit{Web Service Description Language} - Padrão de mercado utilizado para descrever \textit{Web Services}.}

O primeiro\textit{Web Service} REST foi criado por \textit{Roy Fielding} no ano 2000 na universidade da California. Ele foi criado para suceder a tecnologia SOAP. A fim de facilitar a utilização e a aceitação desta nova tecnologia, \textit{Roy Fielding} lançou mão do protocolo HTTP e o definiu como o protocolo de comunicação para \textit{Web Services} REST, sua decisão se baseou no fato de que, este protocolo já possui mecanismos de segurança implementados, além de ser o protocolo padrão utilizado na internet para transferência de dados e acesso a recursos como páginas \textit{web}, o que tornaria-o mais bem aceito \cite{ibm_web_service}.

Segundo \citeonline{oracle_web_service}, na arquitetura do REST os dados e as funcionalidades são considerados recursos e ambos são acessados por meio de  URIs\footnotemark[23]. A arquitetura REST se assemelha a arquitetura \textit{client/server} e foi desenvolvido para funcionar com protocolos de comunicação baseados em estado, como o HTTP. Como nos \textit{Web Services} SOAP o acesso aos recursos do REST também é realizado por meio de contratos anteriormente definidos.

\footnotetext[23]{URI: \textit{Uniform Resource Identifier} - \textit{Link} completo para acessar um determinado recurso.}

Para \citeonline{ibm_web_service}, o \textit{Web service} REST segue quatro princípios básicos. São eles: 

\begin{itemize}
	\item utiliza os métodos HTTP explicitamente;
	\item é orientado à conexão;
	\item expõe a estrutura de diretório por meio das URIs;
	\item trabalha com -\textit{Extensible Markup Language} - XML\footnotemark[24], \textit{Javascript Object Notation} - JSON\footnotemark[25] - ou ambos.
\end{itemize}

\footnotetext[24]{XML: \textit{Extensible Markup Language} - Tecnologia utilizada para transferência de dados, configuração, entre outras atividades.}

\footnotetext[25]{JSON: \textit{Javascript Object Notation} - Notação de objetos via Javascript, porém seu uso não é restrito apenas ao Javascript.}

Essa tecnologia foi selecionada para ser utilizada neste trabalho, pois a forma de acesso ao banco de dados Neo4j utiliza uma API REST fornecida pelo próprio Neo4j,

\subsection{HTML 5}

Segundo \citeonline{w3c_html_fundamentals}, \textit{Hypertext Markup Language}  - HTML\footnotemark[26] - é a linguagem usada para descrever o conteúdo das páginas \textit{web}. Ela utiliza marcadores denominados \textit{tags} para identificar aos navegadores de internet como eles devem interpretar tal documento.

\footnotetext[26]{HTML: \textit{Hypertext Markup Language} - Linguagem usada para descrever o conteúdo das páginas \textit{web}.}

\citeonline{silva_css_3} afirma que o HTML foi criado única e exclusivamente para ser uma linguagem de marcação e estruturação de documentos (páginas) \textit{web}. Portanto, não cabe a ele definir os aspectos dos componentes como cores, espaços, fontes, etc.

\citeonline{w3c_html_fundamentals} afirma que a primeira versão do HTML foi criada no ano de 1991 pelo inventor da \textit{web}, Tim Berners-Lee. A partir desta versão, o HTML foi e continua em constante atualização e hoje se encontra na sua oitava versão. As oito versões são: HTML, HTML +, HTML 2.0, 3.0, 3.2, 4.0, 4.01 e a versão atual é a 5.

Ao longo dos anos e da evolução propriamente dita do HTML, novas \textit{tags} foram criadas, padrões adotados, e claro, novas versões foram criadas, até que, em maio de 2007  o \textit{World Wide Web Consortium} - W3C\footnotemark[27] - confirma a decisão de voltar a trabalhar na atual versão do HTML, também conhecida por HTML 5 \cite{w3c_html_fundamentals}.

\footnotetext[27]{W3C:  \textit{World Wide Web Consortium} - Consórcio internacional formado por empresas, instituições, pesquisadores, desenvolvedores e público em geral, com a finalidade de elevar a web ao seu 	potencial máximo.}

\citeonline{silva_html5} afirma que em novembro daquele mesmo ano o W3C publicou uma nota contendo uma série de diretrizes que descrevem os princípios a serem seguidos ao desenvolver utilizando o HTML 5 em algumas áreas. Tais princípios permitiram a esta versão maior segurança, maior compatibilidade entre navegadores e interoperabilidade entre diversos dispositivos.

Por estes motivos, somado ao fato de que ao utilizar essa tecnologia é possível obter uma maior flexibilidade no desenvolvimento das páginas \textit{web}, o HTML foi selecionado para ser utilizado neste trabalho, em conjunto com outras tecnologias cujas descrições serão apresentadas nas próximas seções.

\subsection{CSS 3}

\citeonline{silva_css_3} afirma que a principal função do \textit{Cascading Style Sheet} - CSS\footnotemark[28] - é definir como os componentes anteriormente estruturados nos documentos \textit{web}, por meio do HTML, devem ser apresentados ao usuário.

\footnotetext[28]{CSS: \textit{Cascading Style Sheet} - Documentos que definem estilos aos componentes da página \textit{web}.}

Para \citeonline{w3c_css_definition}, o CSS é "\textit{a simple mechanism for adding style (e.g., fonts, colors, spacing) to Web documents}\footnotemark[29]".

\footnotetext[29]{O CSS é um simples mecanismo para adicionar estilo, como: fontes, cores, espaços para os documentos \textit{Web}.}

Tim Berners-Lee inicialmente escrevia as estilizações de seus documentos \textit{web}, mesmo que de forma simples e limitada, nos próprios documentos HTML. Isto se deve ao fato dele acreditar que tal função deveria ser realizada pelos navegadores. Entretanto, em 1994 a primeira proposta de criação do CSS surgiu e, em 1996, a primeira versão do CSS denominada CSS 1 foi lançada como recomendação do W3C \cite{silva_css_3}.

De acordo com \citeonline{silva_css_3}, atualmente o CSS possui quatro versões, são elas: A CSS 1, a 2, a 2.1 e atualmente a 3, que foi utilizada para o desenvolvimento deste trabalho.

Há três formas de incorporar o CSS em seu documento web segundo \cite{silva_css_3}, são elas: 

\begin{itemize}
	\item \textbf{\textit{Inline}:} É possível aplicar o estilo diretamente ao componente desejado, por meio do uso da propriedade \texttt{style} do componente HTML. Como é apresentado no Código~\ref{list:css_inline};
	
	% Inserindo o código HTML via listagem
	\begin{lstlisting} [style=custom_HTML,caption={[Inclusão do estilo CSS \textit{inline}.]{Inclusão do estilo CSS \textit{inline}. \textbf{Fonte:} Elaborado pelos autores.}}, label=list:css_inline] 	
	<!DOCTYPE html>
		<html>
		<head>
			<title>Titulo da pagina</title>
		</head>
		<body>
			<p style="color:red;">
				Exemplo de estilo CSS aplicado diretamente ao componente
			</p>
		</body>
	</html>
	\end{lstlisting}
	
	% Removido pois não utilizará mais figura e sim listagem
	%\begin{figure}[h!]
	%	\centerline{\includegraphics[scale=0.8]{./imagens/example_css_inline.png}}
	%	\caption[Exemplo de inclusão do estilo CSS inline]
	%	{Exemplo de inclusão do estilo CSS inline. \textbf{Fonte:} Elaborado pelos autores.}
	%	\label{fig:css_inline}
	%\end{figure}
	
	\item \textbf{Incorporado:} Outra forma, é escrever todo CSS referente ao  documento \textit{web} dentro da \textit{tag} \texttt{style} do documento HTML. Para tanto, esta \textit{tag} deve ser inserida entre o início e o fim da \textit{tag} \texttt{head} do documento. Como é apresentado no Código~\ref{list:css_incorporado};
	
	% Inserindo o código HTML via listagem
	\begin{lstlisting} [style=custom_HTML,caption={[Inclusão do estilo CSS incorporado à página HTML.]{Inclusão do estilo CSS incorporado à página HTML. \textbf{Fonte:} Elaborado pelos autores.}}, label=list:css_incorporado] 	
	<!DOCTYPE html>
	<html>
		<head>
			<title>Titulo da pagina</title>
			<style>
				p {
					color: red;
				}
			</style>
		</head>
		<body>
			<p>
				Exemplo de estilo CSS aplicado a pagina HTML por meio da TAG style
			</p>
		</body>
	</html>
	\end{lstlisting}
	
	% Removido pois não utilizará mais figura e sim listagem
	%\begin{figure}[h!]
	%	\centerline{\includegraphics[scale=0.65]{./imagens/example_css_incorpored.png}}
	%	\caption[Exemplo de inclusão do estilo CSS incorporado à página HTML]
	%	{Exemplo de inclusão do estilo CSS incorporado à página HTML. \textbf{Fonte:} Elaborado pelos autores.}
	%	\label{fig:css_incorporado}
	%\end{figure}
	
	\item \textbf{Externo:} A última forma, é criar um arquivo externo com extensão \texttt{.css} e definir todas as regras de estilização do documento \textit{web} neste arquivo. Desta forma, para vincular tal arquivo a um documento HTML específico será necessário utilizar a \textit{tag} \texttt{link} entre o início e o fim da \textit{tag} \texttt{head} do documento, como apresentado no Código~\ref{list:css_externo};
	
	% Inserindo o código HTML via listagem
	\begin{lstlisting} [style=custom_HTML,caption={[Inclusão do estilo CSS a partir de um arquivo externo.]{Inclusão do estilo CSS a partir de um arquivo externo. \textbf{Fonte:} Elaborado pelos autores.}}, label=list:css_externo] 	
	<!DOCTYPE html>
	<html>
		<head>
			<title>Titulo da pagina</title>
			<link rel="stylesheet" href="style.css">
		</head>
		<body>
			<p>
				Exemplo de estilo CSS carregado a partir de um arquivo externo
			</p>
		</body>
	</html>
	\end{lstlisting}
	
	% Removido pois não utilizará mais figura e sim listagem
	%\begin{figure}[h!]
	%	\centerline{\includegraphics[scale=0.8]{./imagens/example_external_css.png}}
	%	\caption[Exemplo de inclusão do estilo CSS a partir de um arquivo externo]
	%	{Exemplo de inclusão do estilo CSS a partir de um arquivo externo. \textbf{Fonte:} Elaborado pelos autores.}
	%	\label{fig:css_externo}
	%\end{figure}
	
\end{itemize}

O CSS 3 será utilizado neste trabalho, pois, ele permite definir estilos aos componentes das páginas \textit{web} e possui recursos que não existem em suas versões anteriores, o que nos permite desenvolver páginas mais atrativas com menos recursos.

\subsection{Javascript}

\citeonline{javascript_diego_leonard} afirmam que o Javascript foi criado e lançado pela Netscape em 1995, em conjunto com o navegador de internet Netscape Navigator 2.0. A partir deste lançamento, as páginas \textit{web} passaram a ganhar vida com a possibilidade de implementar um mínimo de dinamicidade. Isto se deve ao modo como a linguagem acessa e manipula os componentes do navegador. Contudo, ela pode ser utilizada em diferentes dispositivos como \textit{smartphones, smart tv}, entre outros, não limitando-se apenas a navegadores de internet.

O Javascript é uma linguagem de programação para \textit{web}. A maioria dos sites usa essa linguagem, inclusive todos os navegadores mais modernos, vídeo games, \textit{tablets}, \textit{smart phones}, \textit{smart tvs} possuem interpretadores de \textit{Javascript}, o que a tornou, a linguagem de programação mais ambígua da história \cite{flanagan_javascript_definitive_guide}.

Segundo \citeonline{javascript_diego_leonard}, as semelhanças entre o Javascript e o Java se limitam apenas ao nome. A primeira linguagem não deriva da segunda, apesar de ambas compartilharem alguns conceitos e detalhes. O Javascript, por ser uma linguagem interpretada, é mais flexível que o Java, que, por sua vez, é uma linguagem compilada.

De acordo com \citeonline{flanagan_javascript_definitive_guide}, o Javascript possui 6 versões, sendo elas: 1.0, 1.1, 1.2, 1.3, 1.4 e a atual versão 1.5.

Para \citeonline{javascript_diego_leonard}, há duas maneiras de incluir e executar o código escrito em Javascript nos documentos HTML. A primeira é incluir o código Javascript entre as \textit{tags} \texttt{script} como mostra o Código~\ref{list:javascript_incorporado}. 

% Inserindo o código Javascript via listagem
\begin{lstlisting} [style=custom_HTML,caption={[Inclusão do código em Javascript incorporado ao HTML.]{Inclusão do código em Javascript incorporado ao HTML. \textbf{Fonte:} Elaborado pelos autores.}}, label=list:javascript_incorporado] 	
<!DOCTYPE html>
<html>
	<head>
		<title>Titulo da pagina</title>
	</head>
	<body>
		<script type="text/javascript">
			document.writeLine("Exemplo de codigo Javascript incorporado " +
								 "ao documento HTML por meio da TAG SCRIPT");
		</script>
	</body>
</html>
\end{lstlisting}

% Trocado de imagem para listagem
%\begin{figure}[h!]
%	\centerline{\includegraphics[scale=0.8]{./imagens/javascript_code.png}}
%	\caption[Exemplo de inclusão do código em Javascript incorporado ao HTML]
%	{Exemplo de inclusão do código em Javascript incorporado ao HTML. \textbf{Fonte:} Elaborado pelos autores.}
%	\label{fig:javascript_incorporado}
%\end{figure}

A segunda forma é incluir um arquivo externo com extensão \texttt{.js} através da mesma \textit{tag} \texttt{script}. Veja um exemplo de inclusão de um arquivo contendo códigos em Javascript no documento HTML no Código~\ref{list:javascript_externo}.

% Inserindo o código Javascript via listagem
\begin{lstlisting} [style=custom_HTML,caption={[Inclusão do código Javascript de um arquivo externo.]{Inclusão do código Javascript de um arquivo externo. \textbf{Fonte:} Elaborado pelos autores.}}, label=list:javascript_externo] 	
<!DOCTYPE html>
<html>
	<head>
		<title>Titulo da pagina</title>
	</head>
	<body>
		<script type="text/javascript" src="Exemplo.js"></script>
	</body>
</html>
\end{lstlisting}

% Trocado de imagem para listagem
%\begin{figure}[h!]
%	\centerline{\includegraphics[scale=0.8]{./imagens/javascript_include.png}}
%	\caption[Exemplo de inclusão do código Javascript de um arquivo externo]
%	{Exemplo de inclusão do código Javascript de um arquivo externo. \textbf{Fonte:} Elaborado pelos autores.}
%	\label{fig:javascript_externo}
%\end{figure}

Para \citeonline{flanagan_javascript_definitive_guide}, as três tecnologias (HTML, CSS e Javascript) devem ser usadas em conjunto, uma vez que, cada uma delas possui seu papel específico, sendo eles: o HTML usado para especificar o conteúdo da página \textit{web}, o CSS para especificar como os componentes serão apresentados e o Javascript para especificar o comportamento da página.

Pelos motivos acima mencionados, somado ao fato de que o Javascript permite ao desenvolvedor criar páginas \textit{web} mais dinâmicas e flexíveis, atendendo perfeitamente os requisitos deste trabalho, essa tecnologia será utilizada para auxiliar na criação das páginas \textit{web} deste trabalho.

\subsection{Angular JS}

Segundo \citeonline{green_seshadri_angularjs}, o framework Angular JS foi criado para facilitar o desenvolvimento de aplicativos \textit{web}, pois através dele, é possível criar um aplicativo \textit{web} com poucas linhas.

Para o criador do Angular JS, Miško Hevery, o que o motivou a criar o Angular JS foi a necessidade de ter que reescrever determinados trechos de códigos em todos os outros projetos, o que para ele, tornou-se inviável. Portanto, Miško decidiu criar algo para facilitar o desenvolvimento de aplicativos \textit{web} de uma forma que ninguém havia pensado antes.

O Angular JS é um framework \textit{Model-View-Controller} - MVC\footnotemark[30] - escrito em Javascript. Ele é executado pelos navegadores de internet e ajuda os desenvolvedores a escreverem modernos aplicativos \textit{web} \cite{kozlowski_darwin_mastering_web_application_angular_js}.

\footnotetext[30]{MVC: \textit{Model-View-Controller} - \textit{Design pattern}.}

Devido às facilidades que o Angular JS nos traz, é que ele foi escolhido para ser utilizado, a fim de auxiliar no desenvolvimento, não apenas das páginas web e seus respectivos conteúdos, como também na lógica e comunicação com o \textit{Web Service} REST.

%\chapter{QUADRO METODOLÓGICO}
\label{cap:quadroMetodologico}


\par Neste quadro metodológico serão apresentados os passos que se fizeram necessários para a realização desta pesquisa. Nele estão descritos desde a escolha do perfil da pesquisa até os procedimentos utilizados para o seu desenvolvimento. \citeonline{gil_metodos_e_tecnicas_de_pesquisa} cita que a metodologia é um conjunto de procedimentos intelectuais e técnicos que trabalham para a realização do objetivo proposto.
\par Posterior ao estudo do levantamento teórico e técnico, foram definidos os procedimentos para a construção deste trabalho, iniciando pela escolha do tipo de pesquisa, demonstrada na seção a seguir.


%Será apresentado neste capítulo a metodologia de pesquisa a ser utilizada para a realização deste projeto e os passos necessários até a sua conclusão.

\section{Tipo de pesquisa}

\par Para \citeonline[p. 31]{padua_metodologia_pesquisa}, pesquisa é:

\begin{citacao}
	Toda atividade voltada para a solução de problemas; como atividade de busca, indagação, investigação, inquirição da realidade, e a atividade que visa nos permitir, no âmbito da ciência, elaborar um  conhecimento, ou um conjunto de conhecimentos, que nos auxilie na compreensão desta realidade e nos oriente em nossas ações.
\end{citacao}

% Comentário, pois estava repetindo demais o fato de utilizar  apesquisa aplicada no desenvolvimento do trabalho. 
%\par O tipo de pesquisa aplicada foi escolhido para o desenvolvimento da pesquisa, pois conforme \citeonline[p. 32]{cooper_schindler_metodos_pesquisa_administracao}, ela ``tem uma ênfase prática na solução de problemas, embora a solução de problemas nem sempre seja gerada por uma circunstância negativa.''

\par De forma objetiva, a pesquisa é o meio utilizado para buscar respostas aos mais diversos tipos de indagações, tendo por base procedimentos racionais e sistemáticos. A pesquisa é realizada quando se tem um problema e não se tem informações suficientes para solucioná-lo. Esta seção tem como objetivo explicar o tipo de pesquisa que norteou o desenvolvimento deste trabalho, justificando também como ele se enquadra no tipo escolhido.

\par Pesquisar é um trabalho que envolve planejamento, para que ela seja satisfatória, o pesquisador precisa estar envolvido e  desenvolver habilidades técnicas que o levem a escolher o melhor caminho em busca da obtenção dos resultados.

\par Segundo \citeonline{fonseca_metodologia_da_pesquisa}, ter um método de pesquisa envolve o estudo dos fatores que compõem o contexto da mesma, tais como, a escolha do caminho e o planejamento do percurso. Essa escolha inicia-se com a definição do tipo de pesquisa utilizada. Para este trabalho, é utilizada a pesquisa aplicada, que é aquela cujo o pesquisador tem como objetivo aplicar os conhecimentos obtidos durante o período da pesquisa, em um projeto real, a fim de conhecer os seus resultados. \citeonline{gil_como_elaborar_projeto_de_pesquisa} afirma que este tipo de pesquisa dirige-se à solução de problemas específicos, de interesses locais.

%\par Conforme \citeonline[p. 32]{cooper_schindler_metodos_pesquisa_administracao}, a pesquisa aplicada tem uma ênfase prática na solução de problemas. 
\par Nesta pesquisa foram estudados os conceitos de banco de dados orientado a grafos e a sua aplicabilidade, desenvolvendo, por meio dos conhecimentos obtidos, uma solução prática, disponibilizada por meio de um sistema \textit{web}, que auxilia na busca por mão de obra temporária, que não caracterize vínculo empregatício.

\par Seguindo o enquadramento desta pesquisa, ela deve ser aplicada a um contexto específico, conforme será abordado a seguir. 



%\par Seguindo esta ideia, este tipo de pesquisa será utilizado no desenvolvimento deste projeto, pois, o objetivo é gerar uma solução para o problema de localização de mão de obra, por meio de um sistema \textit{web}. Este projeto adequa-se perfeitamente ao tipo de pesquisa aplicada, uma vez que o mesmo busca analisar e gerar uma possível solução para o problema em destaque.

%Parágrafo utilizado no pré-projeto. Foi corrigido por Edilson no dia 02/04/15 e criado o parágrafo acima
%\par Este tipo de pesquisa será utilizada no desenvolvimento deste projeto, pois a mesma busca analisar o problema e gerar uma solução para o mesmo através de um aplicativo ou serviço.  Neste caso, será o desenvolvimento de um sistema \textit{web} que auxilia na busca por profissionais temporários.


\section{Contexto de pesquisa}

\par O trabalho informal é um elemento estrutural da economia no Brasil e nos países em desenvolvimento. Ele faz parte do cenário atual, crescente a cada dia, e contribui ativamente para a geração de renda. É considerado como um desdobramento do excesso de mão de obra, definido a partir de pessoas que criam sua própria forma de trabalho como estratégia de sobrevivência ou como forma alternativa de recolocação no mercado de trabalho. O fortalecimento deste tipo de trabalho ocorre a partir da construção de redes, formadas por parentes e amigos, criando laços de confiança que são fundamentais para o desempenho da atividade. No entanto, há uma grande dificuldade de se encontrar estes profissionais, uma vez que não há um lugar centralizado para divulgar o seu perfil profissional.

\par O desenvolvimento deste trabalho se propôs a atuar sobre essa limitação. Uma pesquisa informal, realizada por meio de um questionário, na região do sul de Minas Gerais, com pessoas de diferentes perfis sociais, constatou que uma aplicação capaz de centralizar a busca por estes profissionais seria muito bem aceita. A partir deste resultado, validou-se a ideia de construir um ambiente \textit{web} onde o trabalhador informal tenha espaço para centralizar suas habilidades e manter um perfil visível aos possíveis contratantes. Qualquer prestador de serviço informal pode ter acesso a este ambiente, desde que possua um dispositivo eletrônico capaz de se conectar à internet. 

\par O ambiente desenvolvido também visa facilitar ao contratante a busca por estes profissionais, uma vez que não é fácil localizá-los por meio dos mecanismos de busca tradicionais. Desta forma, existe um benefício mútuo, em que contratados e contratantes dispõem da praticidade.

\par Enfim, o contexto ao qual esta pesquisa se destina busca ser bem abrangente, com o intuito de contribuir de forma relevante, proporcionando uma boa experiência aos envolvidos.
 
%Exemplo de contexto que a Joelma deu na sala de aula. Usar como exemplo
%\par Este sistema volta-se para implantação e execução em todas as empresas de pequeno porte do ramo varejista no sul de minas gerais que tenha apresentado a necessidade de um controle mais rigoroso da sua entrada e saída de mercadorias

%Segundo a Joelma e o Márcio este seria o contexto da pesquisa: As pessoas que buscam mão de obra para determinados tipos de trabalho (contratantes) e as pessoas que disponibilizam tais mãos de obra (contratados ou prestadores de serviços).

%Antigo Contexto que a Joelma disse que estava muito geral na correção do pré-projeto
%\par Esta pesquisa terá como foco todas as pessoas que necessitam de mão de obra temporária para realizar tarefas domésticas e rotineiras, além daquelas que não ocorrem com tanta intensidade. Uma vez que o sistema será desenvolvido em uma plataforma \textit{web}, todas as pessoas terão fácil acesso ao serviço, sendo necessário apenas possuir uma comunicação com a internet.

%\par Esta pesquisa terá como foco todas as pessoas da região do sul de Minas Gerais, que necessitam de determinados tipos de mão de obra (contratantes), bem como para aqueles que necessitam de um espaço para divulgar esta oferta (contratados ou prestadores de serviços). Tomando por base que encontrar este tipo de profissional tem se tornado uma tarefa cada vez mais complicada, que acaba gerando transtornos na vida da população, pois, em muitos casos, a falta de opções leva a contratações equivocadas e infelizes.

%\par Como o sistema será desenvolvido em uma plataforma \textit{web}, todas as pessoas neste contexto terão fácil acesso ao serviço, permitindo a disseminação desta ferramenta, sendo necessário apenas possuir uma comunicação com a internet.


%\section{Participantes}

\par Participaram desta pesquisa dois acadêmicos e um professor orientador, sendo eles Andressa Faria, Edilson Justiniano e Márcio Emílio.
\par Andressa de Faria Giordano, acadêmica do 7º período do curso de bacharelado em Sistemas de Informação pela Universidade do Vale do Sapucaí - UNIVAS - com formação técnica e profissionalizante pelo Instituto Nacional de Pesquisa Tecnologica e Computacional - INPETTECC. Sua participação, neste projeto, será o levantamento dos requisitos e a elaboração de todos os diagramas descritos nas fases do ICONIX, a modelagem do banco de dados, a implantação da lógica de negócios, comunicação entre os módulos, acesso aos dados pelo sistema e a documentação do trabalho.

\par  Edilson Justiniano, acadêmico do 7º período do curso de bacharelado em Sistemas de Informação pela Universidade do Vale do Sapucaí - UNIVAS - com formação técnica e profissionalizante pelo Instituto Nacional de Pesquisa Tecnologica e Computacional - INPETTECC. Sua participação, neste projeto, será o levantamento dos requisitos e a elaboração de todos os diagramas descritos nas fases do ICONIX, a modelagem do banco de dados, a implantação da lógica de negócios, comunicação entre os módulos, acesso aos dados pelo sistema e a documentação do trabalho.

\par Márcio Emílio Cruz Vono de Azevedo,professor orientador, engenheiro elétrico em modalidade eletrônica pelo Instituto Nacional de Telecomunicações - INATEL - e mestre em Ciência da Computação pela Universidade Federal de Itajubá - UNIFEI. Professor do INATEL e da Universidade do Vale do Sapucaí - UNIVAS - na área de engenharia de \textit{software} e especialista em sistemas do Inatel.


\par A elaboração do texto escrito foi realizado com base nas ações que foram desenvolvidas pelos participantes, feitas em conjunto.


\section{Instrumentos}

\par Os instrumentos de pesquisa são as ferramentas usadas para a coleta de dados. Como afirma \citeonline[p. 117]{marconi_lakatos_metodologia_trabalho_cientifico}, eles abrangem “desde os tópicos de entrevista, passando pelo questionamento e formulário, até os testes ou escala de medida de opiniões e atitudes”. Eles são de suma importância para o desenvolvimento de um projeto, pois visam a levantar o máximo de informações possíveis para nortear as tomadas de decisões. Para a realização desta pesquisa foram utilizados questionários e análise documental.

\par Segundo \citeonline{gil_como_elaborar_projeto_de_pesquisa}, o questionário é um dos procedimentos mais utilizados para obter informações, pois é uma técnica de custo razoável, apresenta as mesmas questões para todas as pessoas, garante o anonimato e pode conter as questões para atender a finalidades específicas de uma pesquisa. Se aplicada criteriosamente, essa técnica apresenta elevada confiabilidade. Os questionários podem ser desenvolvidos para medir opiniões, comportamento, entre outras questões, também pode ser aplicada individualmente ou em grupos.

\par Para este trabalho, foi desenvolvido um questionário informal, aplicado de forma individual, disponibilizado no ambiente virtual, o qual foi respondido com o intuito de analisar se a pesquisa aqui pretendida seria bem aceita por pessoas de diferentes perfis sociais. %A seguir são discriminados os resultados obtidos.

%Aqui colocar o resultado da análise do questionário

\par A análise documental consiste em identificar, verificar e apreciar os documentos, atendendo a uma finalidade específica, que visa extrair uma informação objetiva da fonte. Para este trabalho foi utilizada a análise documental como material de apoio, que norteou tanto o desenvolvimento teórico quanto o prático.

\par Feita a escolha dos instrumentos, foram definidos os procedimentos necessários para que a pesquisa fosse realizada. Estes procedimentos serão descritos na próxima seção.




%\subsection{Questionários}

\par Para \citeonline{silva_menezes_metodologia_pesquisa_elaboracao_dissertacao}, questionário é:

\begin{citacao}
	Uma série ordenada de perguntas que devem ser respondidas por escrito pelo informante. O questionário deve ser objetivo, limitado em extensão e estar acompanhado de instruções As instruções devem esclarecer o propósito de sua aplicação, ressaltar a importância da colaboração do informante e facilitar o preenchimento.
\end{citacao}

%\par Qual o objetivo de aplicar o questionário? Serão aplicados os questionários para a empresa tal, a fim de levantar o grau de satisfação dos funcionários com o sistema.

%\par Quantos ou quantas e para quem?

\par Serão disponibilizados questionários \textit{on-line} a fim de avaliar o interesse da população por um sistema que ofereça o serviço de localização de mão de obra temporária. Para desenvolver tal \textit{software}, uma bateria de testes será feita visando oferecer a melhor interação entre o usuário e o software, levando em conta sua aplicabilidade, desempenho e facilidade de utilização.

%\par A observação será feita através de formulários de pesquisa que avaliarão o interesse dos usuários pelo sistema que ofereça o serviço de localização de mão de obra temporária. Os testes serão feitos visando oferecer a melhor interação entre o usuário e o software, levando em conta sua aplicabilidade, desempenho e facilidade de utilização.


%Reunião não é aplicado ao nosso trabalho pq reunião não é um instrumento de pesquisa
%\section{Reuniões}

%\par Serão realizados encontros presenciais e também virtuais com os acadêmicos e com o professor orientador, para o levantamento dos requisitos, da aplicabilidade do software, dos modelos de engenharia de software e codificação do sistema, como também questões teóricas que estejam relacionadas com as melhores práticas para se obter um software aplicável e ágil.


%\subsection{Análise documental}

\par Para auxiliar no desenvolvimento deste projeto serão utilizados documentos como: manuais, tutoriais, livros, trabalhos e artigos acadêmicos, além de pesquisa web. Estes serão utilizados para o embasamento teórico que consiste no levantamento de requisitos e tecnologias a serem abordadas, e também para o desenvolvimento prático servindo como apoio e referencial para futuras consultas.


\section{Procedimentos e resultados}

\par Para que esta pesquisa fosse levada a cabo, se fez necessária a implementação de algumas ações, as quais serão detalhadas.

\par O início da pesquisa deu-se através da escolha do tema, seguido pelo levantamento das tecnologias que seriam utilizadas. A princípio, foi definido um escopo contendo algumas tecnologias que haviam sido ministradas no ambiente acadêmico, diminuindo assim a curva de aprendizado. No entanto, foi preciso agregar alguns conhecimentos novos, despendendo um tempo maior para estudo. As tecnologias que acompanharam o desenvolvimento deste trabalho até a sua conclusão estão descritas no quadro teórico desta pesquisa.

\par Para garantir que as tecnologias selecionadas fossem as melhores para o desenvolvimento, foram realizados alguns testes, por meio de aplicações simples. Os testes foram focados na avaliação do comportamento do banco de dados aplicado ao contexto desta pesquisa, cujo objetivo foi desenvolver uma aplicação de busca por mão de obra baseando-se em uma rede de relacionamentos. Estes testes também foram realizados como fins didáticos, visando gerar a familiarização com as tecnologias utilizadas.

\par Para nortear este trabalho, usou-se uma metodologia de desenvolvimento, apresentada a seguir.

\section{Iconix}

\par O ICONIX, segundo \citeonline{rosenberg_iconix_process}, foi criado em 1993 a partir de um resumo das melhores técnicas de desenvolvimento de \textit{software} utilizando como ferramenta de apoio a \textit{Unified Modeling Language} - UML\footnotemark[3]. Esta metodologia é mantida pela empresa ICONIX \textit{Software Engineering} e seu principal idealizador é Doug Rosenberg.

%Nota a respeito da sigla UML
\footnotetext[3]{UML: \textit{Unified Modeling Language} - Linguagem de modelagem para objetos do mundo real que habilita os desenvolvedores especificar, visualizar, construí-los a nível de software.}


\par Para \citeonline{rosenberg_scott_use_case_driven_object_modeling_with_uml}, o ICONIX possui como característica ser iterativo e incremental, somado ao fato de ser adequado ao padrão UML auxiliando, assim, o desenvolvimento e a documentação do sistema.

\par Atualmente, existem diversas metodologias de desenvolvimento de \textit{software} disponíveis, contudo, o ICONIX, em especial, será utilizado para auxiliar no processo de desenvolvimento deste trabalho pois, segundo
\citeonline{silva_videira_uml_metodologias_ferramentas_case}, essa metodologia nos permite gerar a documentação necessária para nortear o desenvolvimento de um projeto acadêmico.

\par De acordo com \citeonline{rosenberg_stephens_use_case_driven_object_modeling_with_uml}, os processos do ICONIX consistem em gerar alguns artefatos que correspondem aos modelos dinâmico e estático de um sistema e estes são elaborados e desenvolvidos de forma incremental e em paralelo, possibilitando ao analista dar maior ênfase no desenvolvimento do sistema do que na documentação do mesmo. A Figura~\ref{fig:visao_geral_iconix_componentes}, apresenta uma visão geral dos componentes do ICONIX.

% Imagem do Iconix
\begin{figure}[h!]
	\centerline{\includegraphics[scale=0.95]{./imagens/visao_geral_iconix.png}}
	\caption[Uma visão geral do ICONIX e seus componentes.]
	{Uma visão geral do ICONIX e seus componentes. \textbf{Fonte:} \citeonline{rosenberg_scott_use_case_driven_object_modeling_with_uml}.}
	\label{fig:visao_geral_iconix_componentes}
\end{figure}

\newpage %Pular pagina para que o texto fica abaixo da imagem na nova pagina

\par Ao utilizar essa metodologia, o desenvolvimento do projeto passa a ser norteado por casos de uso (\textit{use cases}) e suas principais fases são: análise de requisitos, análise e projeto preliminar, projeto e implementação. A seguir será apresentada uma breve descrição de cada uma das fases do ICONIX, seguindo as ideias de \citeonline{o_engenheiro_software_as_fases_iconix_mecva}.

\subsection{Análise de requisitos}

A função da fase de análise de requisitos é modelar os objetos do problema real a partir dos requisitos do software já levantados, e, a partir destes objetos gerar o diagrama de modelo de domínio. Ainda nesta fase, e, com base nos requisitos, deve-se definir os casos de uso do \textit{software}. Estes casos de uso são o elo entre os requisitos e a implementação propriamente dita do \textit{software}. Por isto, a definição destes diagramas se faz tão importante para o ICONIX. 

%A fase de análise de requisitos tem como finalidade identificar os objetos do problema real e, a partir destes objetos definir como eles serão abstraídos para o software por meio do modelo de domínio. Estes objetos são identificados a partir dos requisitos que foram levantados anteriormente. Também apresenta um protótipo das possíveis interfaces gráficas e, por fim, descrever os casos de uso. Quando possível, elabora-se os diagramas de navegação para que o cliente possa entender melhor o funcionamento do sistema como um todo.

\subsection{Análise e projeto preliminar}

Nesta fase deve-se detalhar todos os casos de uso identificados na fase de análise de requisitos, por meio de diagramas de robustez, baseando-se no texto dos casos de uso (fluxo de eventos). O diagrama de robustez não faz parte dos diagramas padrões da UML. Porém, este é utilizado para descobrir as classes de análise e detalhar superficialmente o funcionamento dos casos de uso.

Em paralelo, deve-se atualizar o modelo de domínio adicionando os atributos identificados às suas respectivas entidades, que foram descobertas na fase de análise de requisitos. A partir deste momento será possível gerar a base de dados do sistema.


\subsection{Projeto detalhado}

\par Na fase denominada projeto detalhado deve-se elaborar o diagrama de sequência com base nos diagramas de casos de uso identificados anteriormente, a fim de detalhar sua implementação.

O diagrama de sequência deve conter as classes que serão implementadas e as mensagens enviadas entre os objetos corresponderão às operações que realmente serão implementadas futuramente. Estas operações devem ser incluídas no modelo de domínio em conjunto com as novas classes do projeto identificadas, criando assim, o diagrama de classes final.

\subsection{Implementação}

Na fase de implementação deve-se desenvolver o código fonte do \textit{software} e os testes necessários para obter um software com qualidade. O ICONIX não define os passos a serem seguidos nesta fase, ficando a cargo de cada um definir a forma como será implementado o projeto.
 
Ao término de cada fase um artefato é gerado, sendo respectivamente: revisão dos requisitos, revisão do projeto preliminar, revisão detalhada e entrega.

O ICONIX é considerado um processo prático de desenvolvimento de software, pois a partir das iterações que ocorrem na análise de requisitos e na construção dos modelos de domínios (parte dinâmica), os diagramas de classes (parte estática) são incrementados e, a partir destes, o sistema poderá ser codificado.


Por proporcionar essa praticidade, o ICONIX será empregado para o desenvolvimento deste projeto, pois por meio dele é possível obter produtividade no desenvolvimento do \textit{software} ao mesmo tempo em que alguns artefatos são gerados, unindo o aspecto de abrangência e agilidade.


\subsection{Preparação do ambiente}

\par Visto que o trabalho seria desenvolvido em equipe, foi necessário estabelecer uma ferramenta de controle de versão. Esta ferramente permitiu o gerenciamento de diferentes versões de arquivos, mantendo um histórico com as modificações que foram realizadas no decorrer do processo de desenvolvimento. Este histórico permite o retorno de alguma revisão, caso haja necessidade. A ferramenta escolhida para realizar esse controle foi o GitHub, que já havia sido utilizado em alguns trabalhos do contexto acadêmico, evitando o desprendimento de tempo para estudo de uma nova ferramenta de apoio. O GitHub é uma ferramenta bem difundida e permite que os seus usuários colaborem com os projetos que estão armazenados em seus repositórios\footnotemark[31]. A Figura~\ref{fig:github_inicio} demonstra a tela de serviços provida pelo GitHub.

\footnotetext[31]{Repositório: local cujo desenvolvedor utiliza para armazenar os documentos relacionados ao \textit{software}.}

\begin{figure}[h!]
	\centerline{\includegraphics[scale=0.4]{./imagens/github.jpg}}
	\caption[Tela de serviços do GitHub.]
	{Tela de serviços do GitHub. \textbf{Fonte:} Elaborado pelos autores.}
	\label{fig:github_inicio}
\end{figure}

\par Os passos de instalação detalhados do GitHub são descritos  no Apêndice III deste trabalho.

\par Como mencionado no quadro teórico, neste trabalho foi utilizada a linguagem Java, sendo assim necessária a utilização de uma \textit{Integrated Development Environment} - IDE\footnotemark[32] - de apoio. A IDE escolhida foi o Eclipse, pois se trata de uma ferramenta \textit{open source}, muito utilizada no mercado e que permite a escrita de um código mais legível, facilitando tarefas como \textit{debug} e configurações do trabalho.

\footnotetext[32]{IDE: \textit{Integrated Development Environment} - Aplicação contendo uma série de ferramentas para auxiliar no desenvolvimento de \textit{software}.}

\par O Eclipse possui várias ferramentas, dentre elas, pode-se citar o editor de texto, usado não somente para a escrita de códigos em Java, e também a perspectiva de configuração para servidores \textit{web}, utilizada neste trabalho, conforme apresenta a Figura~\ref{fig:ide_eclipse}. Por meio desta perspectiva, foi configurada a aplicação \textit{container} Tomcat na versão 7.

\begin{figure}[h!]
	\centerline{\includegraphics[scale=0.3]{./imagens/eclipse-editor-texto.png}}
	\caption[Ferramentas da IDE Eclipse.]
	{Ferramentas da IDE Eclipse. \textbf{Fonte:} Elaborado pelos autores.}
	\label{fig:ide_eclipse}
\end{figure}

\par O Tomcat desempenhou um papel fundamental na execução desta aplicação, pois serviu como hospedeiro para a aplicação Java desenvolvida neste trabalho. 

\par Os passos de instalação e configuração do Eclipse e do Tomcat são descritos no Apêndice IV deste trabalho.

\par Para a escrita do código relacionado ao HTML, CSS e Javascript, foi utilizado o mesmo editor de texto citado anteriormente.

\par O trabalho fez uso de um banco de dados orientado a grafos, o Neo4j. A escolha desse banco se deu pela sua simplicidade de instalação, configuração, facilidade de integração com a API \textit{Cypher} e por disponibilizar uma API REST para acesso aos seus dados, conforme descrito no quadro teórico deste trabalho. O Neo4j faz parte do enquadramento de softwares livres, seguindo o conceito \textit{open source}, o que permite ao desenvolvedor utilizá-lo da forma que melhor lhe convier. 

\par Os passos para a instalação do banco de dados Neo4j são detalhados no Apêndice V deste trabalho.

\par Posterior à configuração do ambiente, iniciou-se o desenvolvimento propriamente dito, apresentado a seguir.


\subsection{Desenvolvimento}

\par A princípio, utilizou-se as tecnologias Neo4j, sendo executado de forma \textit{embedded}, Primefaces e JSF. Porém não estava fluindo como o esperado. Outro problema encontrado ao utilizar tais tecnologias foi que tanto a parte cliente (\textit{front end}) quanto a parte servidor (\textit{back end}) se encontravam totalmente acoplados em uma aplicação Java \textit{web}. Por estes motivos decidiu-se mudar algumas das tecnologias utilizadas.

\par Posterior a esse incidente, passou-se a utilizar então as linguagens HTML 5, CSS 3, Javascript e o \textit{framework} Angular JS para auxiliar no desenvolvimento do \textit{front end}, ao invés de Primefaces e JSF. Para acesso ao banco de dados, lançou-se mão da forma \textit{embedded} e passou-se a utilizar a API REST disponibilizada pelo próprio banco. Tais decisões nos permitiram desacoplar o sistema e manter o \textit{front end} e o \textit{back end} independentes, evitando, assim, que o mesmo problema voltasse a ocorrer.

\par Como a forma de conexão ao banco de dados foi alterada, houve a necessidade de reescrever a classe responsável por realizar esta conexão, conforme apresenta o Código~\ref{list:codigo_comunicacao_banco}.

\begin{lstlisting} [style=custom_Java,caption={[Código de comunicação com o banco de dados]{Código de comunicação com o banco de dados. \textbf{Fonte:} Elaborado pelos autores.}}, label=list:codigo_comunicacao_banco]
public class FactoryDAO {

	private static final String DATABASE_ENDPOINT =
								 "http://localhost:7474/db/data";
	private static final String DATABASE_USERNAME = "neo4j";
	private static final String DATABASE_PASSWORD = "admin";
	private static final String cypherUrl = 
								 DATABASE_ENDPOINT + "/cypher";
	
	private static WebResource instance;
	
	private FactoryDAO() {
	}
	
	public static WebResource GetInstance() {
		WebResource resource = null;
		if (instance == null) {
			Client c = Client.create();
			c.addFilter(new HTTPBasicAuthFilter(DATABASE_USERNAME,
												DATABASE_PASSWORD));
			resource = c.resource(cypherUrl);
		}
		return resource;
	}
}
\end{lstlisting}

% Trocado de imagem para listagem
%\begin{figure}[h!]
%	\centerline{\includegraphics[scale=0.3]{./imagens/conexao-banco.jpg}}
%	\caption[Código de comunicação com o banco]
%	{Código de comunicação com o banco \textbf{Fonte:} Elaborado pelos autores.}
%	\label{fig:codigo_comunicacao_banco}
%\end{figure}

\par Após realizar a mudança de tecnologias, foram executados alguns procedimentos para compreender o funcionamento do \textit{web service} REST e em paralelo, foi feito o levantamento dos materiais de referência do \textit{framework} Angular JS. Foi preciso realizar testes para validar a conexão com o banco de dados Neo4j via API REST, fornecida por ele, além de realizados testes funcionais para envio de requisições e recebimento de respostas do \textit{web service} REST, utilizando o Angular JS. Para validar a conexão ao banco de dados via API REST foi necessário desenvolver algumas consultas em \textit{cypher}, como apresenta o Código~\ref{list:consulta_usando_api_cypher}.

\begin{lstlisting} [style=custom_Java,caption={[Exemplo de consulta usando a API \textit{cypher}]{Exemplo de consulta usando a API \textit{cypher}. \textbf{Fonte:} Elaborado pelos autores.}}, label=list:consulta_usando_api_cypher] 	

public class PersonDAO {
...

	/**
	* Used to get all data of person to show the profile data
	* 
	* @param partnerEmail
	* @return
	* @throws JSONException 
	*/
	public JSONArray getPersonData(String partnerEmail) throws JSONException {
		
		WebResource resource = FactoryDAO.GetInstance();
		
		String query = null;
		query = "{\"query\":\" MATCH (partner:Person {email: '"
			+ partnerEmail + "'}), (city:City), "
			+ "(company:Company), "
			+ "(partner)-[:LIVES_IN]->(city), "
			+ "(partner)-[:WORKS_IN]->(company) "
			+ "RETURN DISTINCT({name: partner.name, "
			+ "email: partner.email, photo: partner.photo, " 
			+ "city: city.name, company: company.name, " 
			+ "cpf: partner.cpf, cnpj: partner.cnpj, " 
			+ "typeOfPerson: partner.typeOfPerson, " 
			+ "gender: partner.gender}) as partner; \"}";
		ClientResponse responseCreate = resource
									.accept(MediaType.APPLICATION_JSON)
									.type(MediaType.APPLICATION_JSON).entity(query)
									.post(ClientResponse.class);
		String resp = responseCreate.getEntity(String.class);
		
		JSONObject json = new JSONObject(resp);
		JSONArray objData = json.getJSONArray("data");
		List<JSONObject> parser = JSONUtil
												.parseJSONArrayToListJSON(objData);
		JSONArray arr = new JSONArray(parser);
		
		return arr;
	}
...
}
\end{lstlisting}

% Trocado de imagem para listagem
%\begin{figure}[h!]
%	\centerline{\includegraphics[scale=0.45]{./imagens/query-cypher.png}}
%	\caption[Exemplo de consulta usando a API \textit{cypher}]
%	{Exemplo de consulta usando a API \textit{cypher}. \textbf{Fonte:} Elaborado pelos autores.}
%	\label{fig:consulta_usando_api_cypher}
%\end{figure}

Nesse trecho de código entre as linhas 17 e 27 é apresentada uma consulta escrita usando a API \textit{Cypher}, ela tem por finalidade recuperar os dados de um determinado usuário. Como já mencionado no quadro teórico deste trabalho, o \textit{Cypher} utiliza algumas cláusulas, dentre elas é possível mencionar a \texttt{MATCH} e \texttt{RETURN} cuja utilização delas é apresentada nessa consulta. Na cláusula \texttt{MATCH} são definidos os padrões para realizar a busca, nesse caso, uma pessoa que viva em qualquer cidade, trabalhe em uma empresa qualquer e que possua o \textit{e-mail} igual ao recebido como parâmetro pelo método \texttt{getPersonData}. Já a clásula \texttt{RETURN}, são definidos os dados desejados pela consulta, nesse caso, são eles: o nome do usuário, o \textit{e-mail}, a foto, a cidade onde vive, a empresa onde trabalha, o CPF (em caso de pessoas físicas), o CNPJ (para pessoas jurídicas), o tipo da conta (pessoa jurídica ou física), e o sexo do usuário (usado para pessoas físicas). Mas como é possível notar, foi necessário gerar um objeto JSON manualmente contendo os dados desejados como pode ser visualizado a partir da linha 22 até a linha 27, pois, por padrão o Neo4j não retorna os resultados no formato JSON comum, como demonstra o Código~\ref{list:exemplo_json_retornado_neo4j}. 

\begin{lstlisting} [style=custom_HTML,caption={[JSON retornado de uma consulta via \textit{Cypher}]{JSON retornado de uma consulta via \textit{Cypher}. \textbf{Fonte:} Elaborado pelos autores.}}, label=list:exemplo_json_retornado_neo4j] 	
{
	...
	column: [
		"name",
		"email",
		"password"
	],
	data: [
		"Andressa Faria",
		"andressa_faria18@hotmail.com",
		"78hweqroqy5brlfgvqpOIoi9uijkhgyteqwr"
	]
	...
}
\end{lstlisting}

Portanto, foi necessário desenvolver uma forma de converter os resultados obtidos nas buscas realizadas no banco de dados, a fim de retornar um JSON válido ao usuário, que futuramente viria a utilizar a API REST fornecida por este \textit{software}. É possível visualizar este tratamento no Código~\ref{list:consulta_usando_api_cypher} a partir da linha 34 até a linha 38.

Na linha 34 foi necessário criar um objeto JSON da classe \texttt{JSONObject} passando o retorno da consulta em seu construtor. Já na linha 35 foi obtido os dados retornados da consulta, porém, como demonstrado no Código~\ref{list:exemplo_json_retornado_neo4j} eles são retornados em uma \textit{collection} (coleção) e não em objetos, portanto, para obtê-los foi necessário criar um objeto da classe \texttt{JSONArray} e recuperá-los por meio do campo \texttt{data} do resultado.

Após recuperar os dados, foi necessário extraí-los dos resultados da consulta que até este ponto estavam armazenados em um \texttt{array} e transferí-los para uma lista de objetos da classe \texttt{JSONObject} como apresenta a linha 36 do Código~\ref{list:consulta_usando_api_cypher}, para tanto, foi criada uma classe estática denominada \texttt{JSONUtil} responsável por realizar essa tarefa e retonar os dados no formato padrão. Esse método é apresentado no Código~\ref{list:parser_jsonarray_to_list_json_object}.

\begin{lstlisting} [style=custom_JAVA,caption={[JSON padrão com os resultados das consultas \textit{Cypher}]{JSON padrão com os resultados das consultas \textit{Cypher}. \textbf{Fonte:} Elaborado pelos autores.}}, label=list:parser_jsonarray_to_list_json_object] 	
public class JSONUtil {
	...
	public static List<JSONObject>
									parseJSONArrayToListJSON(JSONArray array) throws JSONException {
		List<JSONObject> response = new ArrayList<JSONObject>();
		for (int i = 0; i < array.length(); i++) {
			JSONArray arr1 = array.getJSONArray(i);
			for (int j = 0; j < arr1.length(); j++) {
				JSONObject obj = arr1.getJSONObject(j);
				response.add(obj);
			}
		}
		return response;
	}
	...
}
\end{lstlisting}

Após essa extração de dados foi necessário criar um objeto da classe \texttt{JSONArray} passando o retorno do método \texttt{parseJSONArrayToListJSON} da classe \texttt{JSONUtil} em seu construtor, uma vez que, o limite de resultados obtidos em uma consulta depende, exclusivamente da própria consulta e da quantidade de registros armazenados no banco de dados. Após realizar todos esses passos, um objeto da classe \texttt{JSONArray} contendo todos os dados devidamente preenchidos é retornado para o método que requisitou essa consulta, como mostra a linha 38 e 40 do Código~\ref{list:consulta_usando_api_cypher}.
 
\par A partir deste ponto, a aplicação estava totalmente desacoplada, sendo necessário realizar uma configuração, a fim de permitir que as requisições enviadas pelo \textit{front end} fossem aceitas pelo \textit{back end}, localizado em outro domínio.

\par Devido à mudança de tecnologias já comentada, houve a necessidade de atualizar os diagramas de sequência e de classe, inserindo os contratos de serviços do \textit{web service} REST. Com a definição deste contrato que é apresentado no Apêndice II deste trabalho, deu-se início ao desenvolvimento dos casos de uso, identificados na primeira fase do ICONIX. 

\par Posterior à realização dos testes e da escolha definitiva da arquitetura que seria utilizada, iniciou-se a implementação dos casos de uso. O primeiro a ser implementado foi o caso de uso de criação de conta. Para este caso de uso, teve-se o cuidado de criar um mecanismo de criptografia de dados sigilosos, como usuário e senha, visando garantir a segurança da aplicação. Estas informações criptografadas são enviadas a cada requisição e validadas pelo \textit{web service}, sendo atualizadas caso sejam válidas, tornado mais complexo a quebra desta criptografia. Este mecanismo foi desenvolvido com base no sistema de \textit{login} via \textit{token}. Segundo o embasamento usado na criação de contas, deu-se início ao desenvolvimento do sistema de \textit{login} e \textit{logoff}, que também utilizam o conceito de criptografia via \textit{token}. A Figura~\ref{fig:pagina_login} apresenta a página de \textit{login}.

\begin{figure}[h!]
	\centerline{\includegraphics[scale=0.60]{./imagens/login.jpg}}
	\caption[Tela de \textit{log in}.]
	{Tela de \textit{log in}. \textbf{Fonte:} Elaborado pelos autores.}
	\label{fig:pagina_login}
\end{figure}

Segundo \citeonline{token_traditional_babal}, os sistemas de autenticações tradicionais utilizam recursos como sessão e \textit{cookies}. A autenticação do usuário é realizada por meio de alguns dados, geralmente nome de usuário (ou email) e senha, com esses dados a aplicação no \textit{back-end}, os validam junto a base de dados e caso obtenha sucesso nesse processo de validação é criada uma sessão e armazenada no servidor, após realizada toda a validação a aplicação retorna a informação dessa sessão para quem a requisitou de modo que ela seja armazenada no navegador de internet por meio de sessão e/ou \textit{cookies}. A partir desse momento, a cada nova solicitação a aplicação \textit{back-end} irá comparar a sessão armazenada no servidor com a fornecida pelo \textit{front-end}. A Figura~\ref{fig:autenticacao_via_sessao} demonstra o fluxo utilizado pelos sistemas cuja autenticação de sessão é a tradicional.

\begin{figure}[h!]
	\centerline{\includegraphics[scale=0.5]{./imagens/tokens-traditional.png}}
	\caption[Fluxo de autenticação de usuários utilizando a forma tradicional]
	{Fluxo de autenticação de usuários utilizando a forma tradicional. \textbf{Fonte:} \cite{authentication_via_token_chris_sevilleja}.}
	\label{fig:autenticacao_via_sessao}
\end{figure}

Para \citeonline{authentication_via_token_chris_sevilleja}, a autenticação via \textit{token}, diferente da forma convencional não utiliza os recursos de sessão e \textit{cookies}. Contudo o processo inicial é o mesmo, a autenticação se inicia por meio dos mesmos dados da forma tradicional, realizando o mesmo o processo de validação, junto à base de dados, porém caso obtenha sucesso não cria uma sessão, e sim um \textit{token} com os dados necessários para a sua validação \textit{criptografados}. Após criado o \textit{token} ele é enviado de volta ao usuário solicitante de modo que ele seja armazenado pela aplicação cliente, sendo ela um aplicativo \textit{web} ou \textit{mobile}, entre outras. A partir desse momento, a cada nova solicitação, a aplicação cliente deverá enviar o \textit{token} anteriormente recebido do servidor e armazenado por ela, para que ele seja validado. A Figura~\ref{fig:autenticacao_via_token} demonstra o fluxo utilizado pelos sistemas cuja autenticação é realizada via \textit{token}.

\newpage
\begin{figure}[h!]
	\centerline{\includegraphics[scale=0.5]{./imagens/tokens-new.png}}
	\caption[Fluxo de autenticação de usuários utilizando \textit{token}]
	{Fluxo de autenticação de usuários utilizando \textit{token}. \textbf{Fonte:} \cite{authentication_via_token_chris_sevilleja}.}
	\label{fig:autenticacao_via_token}
\end{figure}

Para construir a aplicação seguindo o modelo de autenticação via \textit{token} foi necessário criar uma classe chamada \texttt{Base64Util} responsável por \textit{criptografar e descriptografar} as informações fornecidas no \textit{token} a fim de validá-lo junto a base de dados da aplicação. Essa classe é apresentada no Código~\ref{list:classe_criptografa_descriptografa_token}.

\begin{lstlisting} [style=custom_Java,caption={[Classe responsável pela criptografia e descriptografia do \textit{token}]{Classe responsável pela criptografia e descriptografia do \textit{token}. \textbf{Fonte:} Elaborado pelos autores.}}, label=list:classe_criptografa_descriptografa_token]
public class Base64Util {
	
	public static final String BASE64_TOKEN_SEPARATOR = "|";
	
	public static byte[] encodeToken(String email, String password) {
		/* Get Current Time in order to check if the session
		 * is valid yet.
		 */
		Long currentTime = new Timestamp(new Date().getTime()).getTime();
		byte[] token = Base64.encode(email + BASE64_TOKEN_SEPARATOR + password + BASE64_TOKEN_SEPARATOR + currentTime);
		return token;
	}
	
	public static Token decodeToken(byte[] tokenDecoded) {
		Token token = new Token();
		
		byte[] bytes = Base64.decode(tokenDecoded);
		String tokenStr = new String(bytes);
		
		String[] splitStr = tokenStr.split("\\" + BASE64_TOKEN_SEPARATOR);
		token.setEmail(splitStr[0]);
		token.setPassword(splitStr[1]);
		token.setLastAccessTime(Long.parseLong(splitStr[2]));
		
		return token;
	}
}
\end{lstlisting}

No código acima, o método chamado \texttt{encodeToken} recebe como parâmetro o \textit{e-mail} e a senha fornecidos pelo usuário no momento da realização do login, a partir dessas informações somado a hora atual do sistema que é obtida na linha 10 é gerado o \textit{token} criptografado em \texttt{Base64}. Para realizar a decodificação do \textit{token} é utilizado o método \texttt{decodeToken} cuja chamada é realizada a cada requisição que o cliente realiza, esse método recebe como parâmentro o \textit{token} criptografado e o descriptografa usando a classe \texttt{Base64} como é apresentado na linha 18. Após a descriptografia dele é criado um objeto da classe \texttt{Token} com as informações obtidas pelo \textit{token} fornecido pela requisição.

Diferentemente das aplicações que utilizam este conceito de sessão via \textit{token}, neste trabalho houve-se a necessidade de validar além das informações do usuário incluídas no próprio \textit{token}, a data e hora da última requisição realizada pelo usuário. Portanto, para validar o \textit{token} expirado foi necessário criar a classe \texttt{TokenBi} como demonstra o Código~\ref{list:classe_valida_token}.

\begin{lstlisting} [style=custom_Java,caption={[Classe responsável pela validação do \textit{token}]{Classe responsável pela validação do \textit{token}. \textbf{Fonte:} Elaborado pelos autores.}}, label=list:classe_valida_token]
public class TokenBi {
	
	private static final int MINUTES_OF_SESSION_ACTIVE = 15; //15min
	...
	
	/* Method to check if session is still alive */
	public boolean isExpiredSession(Token token) {
		
		final long ONE_MINUTE_IN_MILLIS = 60000;//millisecs
		
		Date currentTime = new Date();
		Date timeLastAccess = new Date(token.getLastAccessTime() 
				+ (MINUTES_OF_SESSION_ACTIVE * ONE_MINUTE_IN_MILLIS));
		
		if (timeLastAccess.before(currentTime)) {
			return true; //session already expired
		}
		return false; //session activate yet
	}
	...
}
\end{lstlisting}

Para realizar a validação do \textit{token} expirado, foi criado o método \texttt{isExpiredSession} que  recebe como parâmetro um objeto da classe \texttt{Token} contendo além dos dados do usuário, também a informação relacionada à data e hora da última requisição, com base nessa última informação na linha 12 é realizado um cálculo para obter um objeto do tipo \texttt{Date} contendo a data da última requisição do usuário somada a mais 15 minutos. A partir dessa informação é realizada uma comparação entre ela e a data atual do sistema, como é possível visualizar na linha 15, caso essa informação seja anterior a data atual do sistema o \textit{token} está expirado e o método \texttt{isExpiredSession} retorna \textit{true}. Caso contrário ele irá retornar \textit{false}.

\par Com o funcionamento do sistema de \textit{login}, passou-se a desenvolver a página inicial da aplicação, contendo as informações que são restritas ao usuário cadastrado. O sistema apresenta uma página inicial diferente para cada tipo de conta, sendo elas: contratantes, provedores de serviço ou ambos, contendo apenas as informações que são liberadas de acordo com o acesso do usuário, sendo essas relatórios, últimas atualizações na rede de parceiros, avaliações de serviços e prováveis parceiros. A página inicial do tipo contratante é apresentada na Figura~\ref{fig:pagina_inicial_contratante}.

\newpage
\begin{figure}[h!]
	\centerline{\includegraphics[scale=0.3]{./imagens/home-contratante.png}}
	\caption[Página inicial do usuário contratante.]
	{Página inicial do usuário contratante. \textbf{Fonte:} Elaborado pelos autores.}
	\label{fig:pagina_inicial_contratante}
\end{figure}


\par O caso de uso localizar parceiros foi desenvolvido após a conclusão do caso de uso criar conta. A lógica deste caso de uso consiste em buscar por possíveis parceiros, com base na rede de parceria do contrante. O Código~\ref{list:consulta_possiveis_parceiros} apresenta a consulta realizada no banco de dados a fim de obter essas informações.

\begin{lstlisting} [style=custom_SQL,caption={[\textit{Query} para apresentar possíveis parceiros]{\textit{Query} para apresentar possíveis parceiros. \textbf{Fonte:} Elaborado pelos autores.}}, label=list:consulta_possiveis_parceiros] 	
MATCH (me:Person {email: 'andressa_faria18@hotmal.com'}), (users:Person),
(users)-[:WORKS_IN]->(company)<-[:WORKS_IN]-(me)
WHERE users <> me AND users.typeOfAccount <> 'SERVICE_PROVIDER'
AND NOT((users)-[:PARTNER_OF]->(me)-[:PARTNER_OF]->(users)) 
OPTIONAL MATCH
	pMutualFriends=(me)-[:PARTNER_OF]->(another)-[:PARTNER_OF]->(me),
	(users)-[:PARTNER_OF]->(another)-[:PARTNER_OF]->(users)
RETURN DISTINCT({name: users.name, email: users.email, length: 1, 
photo: users.photo, qtde: count(DISTINCT pMutualFriends)}) as person
ORDER BY person.length, person.qtde DESC
UNION ALL
MATCH (me:Person {email: 'andressa_faria18@hotmal.com'}), (users:Person),
(users)-[:LIVES_IN]->(city)<-[:LIVES_IN]-(me)
WHERE NOT((users)-[:WORKS_IN]->()<-[:WORKS_IN]-(me))
AND users.typeOfAccount <> 'SERVICE_PROVIDER'
AND NOT((users)-[:PARTNER_OF]->(me)-[:PARTNER_OF]->(users))
OPTIONAL MATCH 
	pMutualFriends=(me)-[:PARTNER_OF]->(another)-[:PARTNER_OF]->(me),
	(users)-[:PARTNER_OF]->(another)-[:PARTNER_OF]->(users)
RETURN DISTINCT({name: users.name, email: users.email, length: 2, 
photo: users.photo, qtde: count(DISTINCT pMutualFriends)}) as person
ORDER BY person.length, person.qtde DESC
\end{lstlisting}

\par Essa consulta é dividida em duas sub consultas separadas pela cláusula \texttt{UNION ALL}, a fim de atingir um número maior de possíveis parceiros ao usuário, sendo que a primeira delas está recuperando os dados de pessoas que trabalham na empresa cujo o usuário autenticado no sistema trabalha e não possua o relacionamento de ''parceria'' com ele. A última sub consulta é responsável por obter os dados de pessoas que vivem na cidade cujo o usuário autenticado vive, porém não possuem relacionamento de ''parceria'' e não trabalham na mesma empresa. Ambas as sub consultas levam em consideração os ''parceiros'' em comum entre o contratante autenticado e o possível parceiro. A Figura~\ref{fig:exemplo_funcionamento_possivel_parceiro} exemplifica o funcionamento dessa consulta.

\begin{figure}[h!]
	\centerline{\includegraphics[scale=0.4]{./imagens/exemplo-funcionamento-consulta-possiveis-parceiros.png}}
	\caption[Exemplificação da consulta por possíveis parceiros.]
	{Exemplificação da consulta por possíveis parceiros. \textbf{Fonte:} Elaborado pelos autores.}
	\label{fig:exemplo_funcionamento_possivel_parceiro}
\end{figure}


\par Ainda relacionado ao tipo de conta contratante ou ambos, foi implementado o caso de uso adicionar parceiro, que permite ao usuário convidar um possível parceiro para fazer parte da sua rede.

\par Ao enviar a solicitação de parceria, a aplicação executa uma consulta no banco de dados. Essa consulta é responsável por criar uma aresta entre o nó do usuário autenticado no sistema e o parceiro convidado, conforme apresenta o Código~\ref{list:query_adicionar_parceiro}.

\begin{lstlisting} [style=custom_SQL,caption={[\textit{Query} para enviar solicitação de parceria]{\textit{Query} para enviar solicitação de parceria. \textbf{Fonte:} Elaborado pelos autores.}}, label=list:query_adicionar_parceiro] 	
MATCH (me:Person {email: 'edilsonjustiniano@gmail.com'}),
(partner:Person {email: 'andressa_faria18@hotmail.com'})
CREATE (me)-[:PARTNER_OF {since: 9898723435424281}]->(partner)
RETURN {myName: me.name, myEmail: me.email, partnerName: partner.name, 
partnerEmail: partner.email} as added
\end{lstlisting}

\par Essa consulta é separada em três partes, sendo elas dividas pelas cláusulas \texttt{MATCH}, \texttt{CREATE} e \texttt{RETURN}. A primeira parte dessa consulta irá localizar os vértices cujos \textit{e-mails} sejam semelhantes ao do usuário autenticado no sistema e do parceiro convidado, respectivamente. A segunda parte é responsável por criar a aresta entre os vértices obtidos pela primeira parte da consulta. A terceira e última parte, irá retornar os dados de ambos os usuários, sendo eles, o usuário autenticado no sistema e parceiro recém convidado a fazer parte da rede de parceiros do usuário autenticado.

\par Após a implementação da lógica para adicionar um novo parceiro, houve a necessidade de implementar o serviço de requisições de parcerias, uma vez que não bastava apenas um contratante convidar outro para se tornarem parceiros, mas sim que o contratante convidado aceitasse sua solicitação de parceria, para assim se tornarem parceiros. Visando disponibilizar estas solicitações de forma agradável ao usuário, foi desenvolvida uma funcionalidade para que o usuário pudesse aceitar ou rejeitar a solicitação enviada a ele, essa funcionalidade apresentada em destaque na Figura~\ref{fig:aceitar_rejeitar_solicitacao_parceria}.

\begin{figure}[h!]
	\centerline{\includegraphics[scale=0.4]{./imagens/aceitar_rejeitar_solicitacao_parceria.png}}
	\caption[Tela para aceitar ou rejeitar solicitação de parceria.]
	{Tela para aceitar ou rejeitar solicitação de parceria. \textbf{Fonte:} Elaborado pelos autores.}
	\label{fig:aceitar_rejeitar_solicitacao_parceria}
\end{figure}

\par A partir dessa funcionalidade, o usuário será capaz de aceitar ou recusar a solicitação apenas com um \textit{click}. Caso o usuário aceite a solicitação, o sistema irá realizar os procedimentos necessários e irá executar a mesma consulta apresentada no Código~\ref{list:query_adicionar_parceiro}, porém com os \textit{e-mails} invertidos. Essa consulta será reutilizada, pois, para que dois usuários se tornem parceiros é necessário que ambos possuam uma aresta do tipo \texttt{PARTNER OF} apontando um ao outro.

\par Se o usuário rejeitar a solicitação, o sistema realizará os procedimentos necessários e executará a consulta apresentada no Código~\ref{list:query_remover_parceiro} a fim de excluir a aresta criada anteriormente pela solicitação de parceria.

\begin{lstlisting} [style=custom_SQL,caption={[\textit{Query} para remover solicitação de parceria]{\textit{Query} para remover solicitação de parceria. \textbf{Fonte:} Elaborado pelos autores.}}, label=list:query_remover_parceiro] 	
MATCH (me:Person {email: 'andressa_faria18@hotmail.com'}),
(partner:Person {email: 'edilsonjustiniano@gmail.com'}),
(partner)-[rel:PARTNER_OF]->(me)
DELETE rel
RETURN {myName: me.name, myEmail: me.email, 
partnerName: partner.name, partnerEmail: partner.email} as deleted;
\end{lstlisting}

\par Essa consulta, a exemplo da anterior, é separada em três partes, sendo elas dividas pelas cláusulas \texttt{MATCH}, \texttt{DELETE} e \texttt{RETURN}. A primeira parte dessa consulta irá localizar os vértices cujos \textit{e-mails} sejam semelhantes ao do usuário autenticado no sistema e do parceiro convidado, respectivamente, e que possuam uma aresta do tipo \texttt{PARTNER OF} os conectando, porém, essa aresta se inicia no vértice relacionado ao parceiro convidado e o final seja vértice relacionado ao usuário autenticado. A segunda parte consiste apenas na exclusão da aresta obtida na primeira parte da consulta. A terceira e última parte, irá retornar os dados de ambos os usuários, sendo eles, o usuário autenticado no sistema e parceiro cujo, convite para se tornar parceiro foi rejeitado pelo usuário autenticado no sistema.

\par Após realizada a implementação do caso de uso adicionar parceiro, houve a necessidade de desenvolver a busca por todos os usuários que possuíam o tipo de conta contratante ou ambos e que possuíam um relacionamento de parceria com o usuário autenticado no sistema, além da funcionalidade de localizar novos parceiros, baseando-se na localização da empresa na qual o usuário trabalha e na cidade onde ele vive, sempre ordenando os resultados por meio da quantidade de parceiros em comum. O Código~\ref{list:consulta_novos_parceiros} apresenta a \textit{query} utilizada para realizar esta busca.

\begin{lstlisting} [style=custom_SQL,caption={[\textit{Query} para apresentar novos parceiros]{\textit{Query} para apresentar novos parceiros. \textbf{Fonte:} Elaborado pelos autores.}}, label=list:consulta_novos_parceiros] 	
MATCH (me:Person {email: 'andressa_faria18@hotmail.com'}), (users:Person),
(users)-[:WORKS_IN]->(company)<-[:WORKS_IN]-(me)
WHERE users.name =~ 'Edil.*'
AND users.typeOfAccount <> 'SERVICE_PROVIDER'
AND users <> me AND NOT((users)-[:PARTNER_OF]->(me)-[:PARTNER_OF]->(users))  
OPTIONAL MATCH 
	pMutualFriends=(me)-[:PARTNER_OF]->(another)-[:PARTNER_OF]->(me), 
	(users)-[:PARTNER_OF]->(another)-[:PARTNER_OF]->(users) 
RETURN DISTINCT({name: users.name, email: users.email, length: 1, 
photo: users.photo, mutualFriends: count(DISTINCT pMutualFriends)}) 
as partner ORDER BY partner.length, partner.mutualFriends DESC 
UNION ALL 
MATCH (me:Person {email: 'andressa_faria18@hotmail.com'}), (users:Person),
(users)-[:LIVES_IN]-(city)<-[:LIVES_IN]-(me)
WHERE users.name =~ 'Edil.*'
AND users.typeOfAccount <> 'SERVICE_PROVIDER' 
AND NOT((users)-[:WORKS_IN]->()<-[:WORKS_IN]-(me)) 
AND NOT((users)-[:PARTNER_OF]->(me)-[:PARTNER_OF]->(users)) 
OPTIONAL MATCH 
	pMutualFriends=(me)-[:PARTNER_OF]->(another)-[:PARTNER_OF]->(me), 
	(users)-[:PARTNER_OF]->(another)-[:PARTNER_OF]->(users) 
RETURN DISTINCT({name: users.name, email: users.email, length: 2, 
photo: users.photo, mutualFriends: count(DISTINCT pMutualFriends)})
as partner ORDER BY partner.length, partner.mutualFriends DESC
\end{lstlisting}

\par Essa consulta é muito parecida com a apresentada no Código~\ref{list:consulta_possiveis_parceiros}, porém há uma diferença, neste caso é levado em consideração o nome informado pelo usuário como critério de busca.
 

\par O caso de uso ''Gerenciar serviços'' foi implementado em sequência, abrangendo as principais funcionalidades de gerenciamento: cadastrar e adicionar um novo serviço ao usuário, cujo tipo de conta é provedor de serviços, listar os serviços atribuídos a ele, e remover serviços quando necessário. Visando melhorar a usabilidade, foi implementado um mecanismo de busca, que permitiu filtrar os resultados por meio de um campo que possui a função  auto completar, evitando assim, possíveis erros e diminuindo o tempo gasto pelo usuário para adicionar o serviço. A função realiza a busca em uma lista de serviços anteriormente cadastrados, no entanto, caso não haja o serviço solicitado, o usuário tem a liberdade de cadastrá-lo e atribuí-lo a si mesmo. A Figura~\ref{fig:adicionar_servicos} apresenta essa funcionalidade.

\newpage
\begin{figure}[h!]
	\centerline{\includegraphics[scale=0.3]{./imagens/adcionar-servico.png}}
	\caption[Página para adicionar serviços.]
	{Página para adicionar serviços. \textbf{Fonte:} Elaborado pelos autores.}
	\label{fig:adicionar_servicos}
\end{figure}


\par A partir deste ponto, foi possível iniciar o desenvolvimento do caso de uso "localizar mão de obra", uma vez que, este caso de uso dependia diretamente das implementações das funcionalidades adicionar parceiros para os usuários contratantes e adicionar serviços aos provedores de serviço. Para facilitar a localização e deixar o \textit{software} mais usual, esta busca se baseia inicialmente no serviço buscado pelo usuário, sendo posteriormente modificada para também levar em consideração a funcionalidade, avaliar serviço que foi implementada paralelamente. A avaliação de serviço permite ao contratante dar uma nota ao serviço que foi prestado a ele. Com estas informações foi possível desenvolver uma busca que levaria em consideração, além destas informações, a rede de parceiros do usuário contratante, a fim de lhe apresentar as melhores opções possíveis. O Código~\ref{list:consulta_busca} apresenta a \textit{query} utilizada para realizar esta busca.


\begin{lstlisting} [style=custom_SQL,caption={[\textit{Query} para localização de mão de obra]{\textit{Query} para localização de mão de obra. \textbf{Fonte:} Elaborado pelos autores.}}, label=list:consulta_busca] 	
MATCH (me:Person {email: 'edilsonjustiniano@gmail.com'}), (sp:Person), 
(service:Service), (executed:Execute), (partners:Person), 
(partners)-[:PARTNER_OF]->(me)-[:PARTNER_OF]->(partners),
(sp)-[:PROVIDE]->(service), (service)-[:EXECUTE]->(executed),
(sp)-[:EXECUTE]->(executed)
WHERE sp.typeOfAccount <> 'CONTRACTOR' 
AND partners.typeOfAccount <> 'SERVICE_PROVIDER'
AND UPPER(service.name) = UPPER('Domestica')
AND (executed)-[:TO]->(partners) AND me <> sp
RETURN DISTINCT({serviceProviderName: sp.name, 
serviceProviderEmail: sp.email, service: service.name,
total: count(executed), media: avg(executed.note), order: 1}) as sp 
ORDER BY sp.order, sp.media DESC
UNION ALL
MATCH (me:Person {email: 'edilsonjustiniano@gmail.com'}), (sp:Person),
(service:Service {name: 'Domestica'}), (executed:Execute),
(partners:Person), (partners)-[:WORKS_IN]->(company)<-[:WORKS_IN]-(me),
(sp)-[:PROVIDE]->(service), (service)-[:EXECUTE]->(executed), 
(sp)-[:EXECUTE]->(executed), (executed)-[:TO]->(partners)
WHERE sp.typeOfAccount <> 'CONTRACTOR' 
AND partners.typeOfAccount <> 'SERVICE_PROVIDER'
AND NOT((partners)-[:PARTNER_OF]->(me)-[:PARTNER_OF]->(partners))
AND me <> sp
RETURN DISTINCT({serviceProviderName: sp.name, 
serviceProviderEmail: sp.email, service: service.name, 
total: count(executed), media: avg(executed.note), order: 2}) as sp 
ORDER BY sp.order, sp.media DESC
UNION ALL
MATCH (me:Person {email: 'edilsonjustiniano@gmail.com'}), (sp:Person),
(service:Service {name: 'Domestica'}), (executed:Execute), 
(partners:Person), (partners)-[:LIVES_IN]->(city)<-[:LIVES_IN]-(me),
(sp)-[:PROVIDE]->(service), (service)-[:EXECUTE]->(executed), 
(sp)-[:EXECUTE]->(executed), (executed)-[:TO]->(partners)
WHERE sp.typeOfAccount <> 'CONTRACTOR' 
AND partners.typeOfAccount <> 'SERVICE_PROVIDER'
AND NOT((partners)-[:PARTNER_OF]->(me)-[:PARTNER_OF]->(partners))
AND NOT((partners)-[:WORKS_IN]->()<-[:WORKS_IN]-(me))
AND me <> sp
RETURN DISTINCT({serviceProviderName: sp.name, 
serviceProviderEmail: sp.email, service: service.name, 
total: count(executed), media: avg(executed.note), order: 3}) as sp 
ORDER BY sp.order, sp.media DESC 
UNION ALL
MATCH (me:Person {email: 'edilsonjustiniano@gmail.com'}), (sp:Person),
(service:Service {name: 'Domestica'}), (partners:Person),
(partners)-[:LIVES_IN]->(city)<-[:LIVES_IN]-(me), 
(sp)-[:PROVIDE]->(service)
WHERE sp.typeOfAccount <> 'CONTRACTOR' 
AND partners.typeOfAccount <> 'SERVICE_PROVIDER'
AND NOT((partners)-[:PARTNER_OF]->(me)-[:PARTNER_OF]->(partners))
AND NOT((partners)-[:WORKS_IN]->()<-[:WORKS_IN]-(me))
AND me <> sp
RETURN DISTINCT({serviceProviderName: sp.name, 
serviceProviderEmail: sp.email, service: service.name, total: 0,
media: 0, order: 4}) as sp ORDER BY sp.order, sp.media DESC;
\end{lstlisting}

Essa consulta é dividida em quatro sub consultas, separadas pela cláusula \texttt{UNION ALL}, a fim de atingir um número maior de possibilidades de prestadores de serviços. Visando facilitar o entendimento, suas partes são apresentadas isoladamente a seguir.

A primeira delas está recuperando os dados de pessoas que proveram o serviço de doméstica para usuários que possuem o relacionamento de parceria com o usuário autenticado no sistema, nesse caso o \textit{e-mail} dele é \texttt{edilsonjustiniano@gmail.com}. Portanto, nessa primeira parte da consulta serão retornadas as avaliações realizadas pelos parceiros do usuário autenticado no sistema voltadas às pessoas que proveem o serviço de doméstica. Essa sub consulta é demonstrada no trecho de Código~\ref{list:consulta_busca_parte1}.

\begin{lstlisting} [style=custom_SQL,caption={[Primeira \textit{sub query} para localização de mão de obra.]{Primeira \textit{sub query} para localização de mão de obra. \textbf{Fonte:} Elaborado pelos autores.}}, label=list:consulta_busca_parte1] 	
MATCH (me:Person {email: 'edilsonjustiniano@gmail.com'}), (sp:Person), 
(service:Service), (executed:Execute), (partners:Person), 
(partners)-[:PARTNER_OF]->(me)-[:PARTNER_OF]->(partners),
(sp)-[:PROVIDE]->(service), (service)-[:EXECUTE]->(executed),
(sp)-[:EXECUTE]->(executed)
WHERE sp.typeOfAccount <> 'CONTRACTOR' 
AND partners.typeOfAccount <> 'SERVICE_PROVIDER'
AND UPPER(service.name) = UPPER('Domestica')
AND (executed)-[:TO]->(partners) AND me <> sp
RETURN DISTINCT({serviceProviderName: sp.name, 
serviceProviderEmail: sp.email, service: service.name,
total: count(executed), media: avg(executed.note), order: 1}) as sp 
ORDER BY sp.order, sp.media DESC
\end{lstlisting}

A segunda sub consulta irá obter os dados de pessoas que proveram o mesmo serviço da sub consulta anterior para usuários que trabalhem na mesma empresa, mas que não possuem o relacionamento de parceria com o usuário autenticado no sistema. Portanto, nessa segunda parte da consulta serão retornados as avaliações realizadas por pessoas que não são parceiras do usuário autenticado no sistema, mas que trabalham na mesma empresa. Essa sub consulta é demonstrada no trecho de Código~\ref{list:consulta_busca_parte2}.

\begin{lstlisting} [style=custom_SQL,caption={[Segunda \textit{sub query} para localização de mão de obra.]{Segunda \textit{sub query} para localização de mão de obra. \textbf{Fonte:} Elaborado pelos autores.}}, label=list:consulta_busca_parte2] 	
MATCH (me:Person {email: 'edilsonjustiniano@gmail.com'}), (sp:Person),
(service:Service {name: 'Domestica'}), (executed:Execute),
(partners:Person), (partners)-[:WORKS_IN]->(company)<-[:WORKS_IN]-(me),
(sp)-[:PROVIDE]->(service), (service)-[:EXECUTE]->(executed), 
(sp)-[:EXECUTE]->(executed), (executed)-[:TO]->(partners)
WHERE sp.typeOfAccount <> 'CONTRACTOR' 
AND partners.typeOfAccount <> 'SERVICE_PROVIDER'
AND NOT((partners)-[:PARTNER_OF]->(me)-[:PARTNER_OF]->(partners))
AND me <> sp
RETURN DISTINCT({serviceProviderName: sp.name, 
serviceProviderEmail: sp.email, service: service.name, 
total: count(executed), media: avg(executed.note), order: 2}) as sp 
ORDER BY sp.order, sp.media DESC
\end{lstlisting}

Já a terceira sub consulta irá obter os dados de pessoas que proveram o mesmo serviço das sub consultas anteriores para usuários que vivem na mesma cidade que o usuário autenticado no sistema, mas que não trabalhem na mesma empresa que ele e não possuam o relacionamento de parceria com ele. Portanto, nessa terceira parte da consulta serão retornadas as avaliações realizadas por pessoas que não são parceiras do usuário autenticado no sistema, que não trabalhem na mesma empresa dele, mas que vivam na mesma cidade. Essa sub consulta é demonstrada no trecho de Código~\ref{list:consulta_busca_parte3}.

\begin{lstlisting} [style=custom_SQL,caption={[Terceira \textit{sub query} para localização de mão de obra.]{Terceira \textit{sub query} para localização de mão de obra. \textbf{Fonte:} Elaborado pelos autores.}}, label=list:consulta_busca_parte3] 	
MATCH (me:Person {email: 'edilsonjustiniano@gmail.com'}), (sp:Person),
(service:Service {name: 'Domestica'}), (executed:Execute), 
(partners:Person), (partners)-[:LIVES_IN]->(city)<-[:LIVES_IN]-(me),
(sp)-[:PROVIDE]->(service), (service)-[:EXECUTE]->(executed), 
(sp)-[:EXECUTE]->(executed), (executed)-[:TO]->(partners)
WHERE sp.typeOfAccount <> 'CONTRACTOR' 
AND partners.typeOfAccount <> 'SERVICE_PROVIDER'
AND NOT((partners)-[:PARTNER_OF]->(me)-[:PARTNER_OF]->(partners))
AND NOT((partners)-[:WORKS_IN]->()<-[:WORKS_IN]-(me))
AND me <> sp
RETURN DISTINCT({serviceProviderName: sp.name, 
serviceProviderEmail: sp.email, service: service.name, 
total: count(executed), media: avg(executed.note), order: 3}) as sp 
ORDER BY sp.order, sp.media DESC 
\end{lstlisting}

Já a quarta e última sub consulta foi inserida a fim de abranger a busca e possibilitar que novos prestadores de serviços sejam avaliados pelos contratantes. Ela irá obter os dados de pessoas que ainda não foram avaliadas por nenhum parceiro, ou por pessoas que vivam na mesma cidade ou trabalhem na mesma empresa do usuário autenticado no sistema, o que permitiu aos usuários que acabaram de criar sua conta e não obtiveram a oportunidade de serem avaliados que fossem apresentados como opções para o serviço. Essa sub consulta é demonstrada no trecho de Código~\ref{list:consulta_busca_parte4}.

\begin{lstlisting} [style=custom_SQL,caption={[Quarta \textit{sub query} para localização de mão de obra.]{Quarta \textit{sub query} para localização de mão de obra. \textbf{Fonte:} Elaborado pelos autores.}}, label=list:consulta_busca_parte4] 	
MATCH (me:Person {email: 'edilsonjustiniano@gmail.com'}), (sp:Person),
(service:Service {name: 'Domestica'}), (partners:Person),
(partners)-[:LIVES_IN]->(city)<-[:LIVES_IN]-(me), 
(sp)-[:PROVIDE]->(service)
WHERE sp.typeOfAccount <> 'CONTRACTOR' 
AND partners.typeOfAccount <> 'SERVICE_PROVIDER'
AND NOT((partners)-[:PARTNER_OF]->(me)-[:PARTNER_OF]->(partners))
AND NOT((partners)-[:WORKS_IN]->()<-[:WORKS_IN]-(me))
AND me <> sp
RETURN DISTINCT({serviceProviderName: sp.name, 
serviceProviderEmail: sp.email, service: service.name, total: 0,
media: 0, order: 4}) as sp ORDER BY sp.order, sp.media DESC
\end{lstlisting}

\par A Figura~\ref{fig:explificar_consulta_busca_mao_de_obra} exemplifica todo o funcionamento da consulta de forma detalhada, facilitando a compreensão desta, que é, a maior e mais complexa consulta deste trabalho.

\begin{figure}[h!]
	\centerline{\includegraphics[scale=0.4]{./imagens/exemplo-funcionamento-consulta-buscar-mao-de-obra.png}}
	\caption[Exemplificação da consulta por mão de obra.]
	{Exemplificação da consulta por mão de obra. \textbf{Fonte:} Elaborado pelos autores.}
	\label{fig:explificar_consulta_busca_mao_de_obra}
\end{figure}


\par Para auxiliar na tomada de decisão do usuário contratante, foi implementada uma funcionalidade que realiza o cálculo da média de avaliação de um serviço prestado por um profissional temporário, tendo como base as avaliações já realizadas pela rede de parceiros do usuário autenticado, da empresa onde ele trabalha e da cidade onde vive, oferecendo assim uma forma simples de obter acesso a qualidade do serviço prestado como demonstra a Figura~\ref{fig:taxa_avaliacao}.

\begin{figure}[h!]
	\centerline{\includegraphics[scale=0.45]{./imagens/taxa-avaliacao.png}}
	\caption[Tela contendo a taxa de avaliação do serviço prestado.]
	{Tela contendo a taxa de avaliação do serviço prestado. \textbf{Fonte:} Elaborado pelos autores.}
	\label{fig:taxa_avaliacao}
\end{figure}

\par Após realizada todas as implementações já descritas, houve a preocupação de desenvolver uma interface, que além de amigável fosse prática ao usuário, desta forma, foi disponibilizada algumas informações relevantes, que auxiliam o usuário a compreender o que está ocorrendo em sua rede de parceria. Como exemplo é possível citar a lista de parceiros em comum entre o usuário autenticado no sistema e um determinado contratante por meio da página de perfil dele.

\par A fim de agregar mais funcionalidades para o usuário provedor de serviços, foi criado na página inicial do \textit{software} uma funcionalidade que visa apresentar algumas dicas interessantes que contribuem com a sua imagem perante ao \textit{software}, levando-o assim a obter uma quantidade maior de oportunidades de trabalho como mostra a Figura~\ref{fig:dicas_randomicas}.

\newpage
\begin{figure}[h!]
	\centerline{\includegraphics[scale=0.4]{./imagens/dicas-randomicas.png}}
	\caption[Tela de dicas para provedores de serviços.]
	{Tela de dicas para provedores de serviços. \textbf{Fonte:} Elaborado pelos autores.}
	\label{fig:dicas_randomicas}
\end{figure}

\par Para finalizar o desenvolvimento foram desenvolvidos gráficos que apresentem ao usuário informações a respeito da qualidade do serviço prestado pelo provedor de serviços, comparando-os com os demais prestadores. Esses gráficos foram gerados usando a biblioteca chamada \textit{Charts}, disponibilizada no \textit{link} https://developers.google.com/chart. Para utilizá-la, é preciso realizar o \textit{download} e, após concluído, incluir o arquivo chamado \texttt{Charts.js} na página, cujo o gráfico será apresentado, conforme demonstra o Código~\ref{list:codigo_insercao_charts}.

\begin{lstlisting} [style=custom_HTML,caption={[Inserção da biblioteca \textit{charts} na página inicial.]{Inserção da biblioteca \textit{charts} na página inicial. \textbf{Fonte:} Elaborado pelos autores.}}, label=list:codigo_insercao_charts] 	
<script type="text/javascript" src="js/charts/Chart.js"></script>
\end{lstlisting}

\par Para gerar um gráfico, segundo a sua própria documentação, e conforme utilizado neste trabalho, deve-se incluir um elemento \texttt{div} contendo um \texttt{canvas} com um \texttt{id} único, a fim de facilitar o acesso a ele, por meio do código Javascript, que irá manipular os dados apresentados neste gráfico. Essa inclusão do elemento é apresentado no Código~\ref{list:codigo_incluir_canvas_charts}.

\begin{lstlisting} [style=custom_HTML,caption={[Inclusão do elemento necessário para geração do gráfico.]{Inclusão do elemento necessário para geração do gráfico. \textbf{Fonte:} Elaborado pelos autores.}}, label=list:codigo_incluir_canvas_charts] 	
<div>
	<canvas id="canvas" height="450" width="600"></canvas>
</div>
\end{lstlisting}

\par Para manipular as informações do gráfico, tais como dados, cores, entre outras, é necessário criar um objeto em Javascript que contenha todos os atributos necessários para gerá-lo corretamente. Este objeto é apresentado no Código~\ref{list:codigo_objeto_grafico}.

\begin{lstlisting} [style=custom_HTML,caption={[Objeto contendo os atributos de configuração do gráfico.]{Objeto contendo os atributos de configuração do gráfico. \textbf{Fonte:} Elaborado pelos autores.}}, label=list:codigo_objeto_grafico] 	
$scope.lineChartData = {
	labels : ["5 ultimas","10 ultimas","15 ultimas","20 ultimas"],
	datasets : [{
		label: "Minhas avaliacoes como " + $scope.selectedService,
		fillColor : "rgba(166,246,166,0.2)",
		strokeColor : "rgba(61,213,61,1)",
		pointColor : "rgba(61,213,61,1)",
		pointStrokeColor : "#fff",
		pointHighlightFill : "#fff",
		pointHighlightStroke : "rgba(41,157,41,1)",
		data : [
			//filled by lastEvaluateOfServiceProvider
		]   
	}, {
		label: "As avaliacoes de " + $scope.selectedService + " em minha cidade",
		fillColor : "rgba(98,168,248,0.2)",
		strokeColor : "rgba(25,123,235,1)",
		pointColor : "rgba(25,123,235,1)",
		pointStrokeColor : "#fff",
		pointHighlightFill : "#fff",
		pointHighlightStroke : "rgba(33,97,170,1)",
		data : [
			//filled by lastEvaluateOfServiceInNetwork
		]
	}]
};
\end{lstlisting}

\par Após realizados os passos anteriormente descritos, para finalizar a geração do gráfico, foi necessário criar uma nova instância da classe \texttt{Chart}, informando o modelo do gráfico a ser utilizado, nesse caso, o modelo de linhas, a fim de facilitar a comparação entre as avaliações, juntamente com o objeto contendo todas as informações do gráfico, conforme apresentado na linha 3 do Código~\ref{list:codigo_criar_grafico_charts} que demonstra a criação dessa nova instância.

\begin{lstlisting} [style=custom_HTML,caption={[Criação de nova instância para gráficos de linha.]{Criação de nova instância para gráficos de linha. \textbf{Fonte:} Elaborado pelos autores.}}, label=list:codigo_criar_grafico_charts] 	
$scope.loadGraph = function(){
	var ctx = document.getElementById("canvas").getContext("2d");
	window.myLine = new Chart(ctx).Line($scope.lineChartData, {
		responsive: true
	});
};
\end{lstlisting}

\par Essa biblioteca foi selecionada, pois, ela é facilmente integrada com o \textit{framework} Angular JS, facilitando a sua utilização neste trabalho. 

\par O primeiro gráfico foi disponibilizado na página de perfil do prestador de serviços, estando visível ao contratante assim que ele efetue a busca por uma mão de obra. Este gráfico apresenta a média avaliativa dos trabalhos prestados pelo provedor selecionado em uma determinada função. O sistema toma como base as vinte últimas avaliações recebidas, independente do período em que elas foram feitas. Desta forma, um provedor de serviços que trabalha a muito tempo em um mesmo local não será prejudicado, pois terá registrada as avaliações que recebeu neste tempo, mesmo que elas não sejam atuais. O gráfico que representa esta funcionalidade é apresentado na Figura \ref{fig:grafico_pagina_perfil}.

\begin{figure}[h!]
	\centerline{\includegraphics[scale=0.65]{./imagens/grafico-pagina-perfil.png}}
	\caption[Gráfico de avaliação do serviço na página de perfil do provedor.]
	{Gráfico de avaliação do serviço na página de perfil do provedor. \textbf{Fonte:} Elaborado pelos autores.}
	\label{fig:grafico_pagina_perfil}
\end{figure}

\par O segundo gráfico foi disponibilizado visando possibilitar ao prestador de serviços ter um \textit{feedback} dos trabalhos realizados por ele. Este gráfico é apresentado em sua página inicial e trás informações referente a sua média avaliativa. A média é gerada com base nas últimas vinte avaliações recebidas, seguindo o mesmo padrão do gráfico disponibilizado na página de perfil. Para que o provedor tenha uma base da qualidade do serviço prestado, o gráfico apresenta um comparatiParavo da média recebida por ele e a média geral de prestadores de serviço que desempenham o mesmo trabalho em sua cidade. O gráfico que representa esta funcionalidade é apresentado na Figura \ref{fig:grafico_pagina_inicial}.

\newpage
\begin{figure}[h!]
	\centerline{\includegraphics[scale=0.65]{./imagens/grafico-pagina-inicial.png}}
	\caption[Gráfico de avaliação do serviço na página inicial do provedor.]
	{Gráfico de avaliação do serviço na página inicial do provedor. \textbf{Fonte:} Elaborado pelos autores.}
	\label{fig:grafico_pagina_inicial}
\end{figure}

\par Seguindo todos os procedimentos descritos nessa seção obteve-se como resultado a aplicação proposta por este trabalho.

%\newpage
\section{Cronograma}

\par Será exibido aqui uma tabela contendo o cronograma a ser seguido por este projeto, desde a sua definição até a sua conclusão.

\begin{table}[htbp]
	\scriptsize
	\centering
	\begin{tabular}{|p{60mm}|p{3mm}|p{3mm}|p{3mm}|p{3mm}|p{3mm}|p{3mm}|p{3mm}|p{3mm}|p{3mm}|p{3mm}|p{3mm}|p{3mm}|}%Largura das colunas
		\hline\vspace{.99cm} %adicionando o espaçamento para que o nome dos meses caibam verticalmente
		\textbf{Ação} / \textbf{Mês} & 
		\parbox[t]{2mm}{\multirow{3}{*}{\rotatebox[origin=c]{90}{Janeiro}}} & \parbox[t]{2mm}{\multirow{3}{*}{\rotatebox[origin=c]{90}{Fevereiro}}} & 
		\parbox[t]{2mm}{\multirow{3}{*}{\rotatebox[origin=c]{90}{Março}}} & 
		\parbox[t]{2mm}{\multirow{3}{*}{\rotatebox[origin=c]{90}{Abril}}} &
		\parbox[t]{2mm}{\multirow{3}{*}{\rotatebox[origin=c]{90}{Maio}}} &
		\parbox[t]{2mm}{\multirow{3}{*}{\rotatebox[origin=c]{90}{Junho}}} &
		\parbox[t]{2mm}{\multirow{3}{*}{\rotatebox[origin=c]{90}{Julho}}} &
		\parbox[t]{2mm}{\multirow{3}{*}{\rotatebox[origin=c]{90}{Agosto}}} &
		\parbox[t]{2mm}{\multirow{3}{*}{\rotatebox[origin=c]{90}{Setembro}}} &
		\parbox[t]{2mm}{\multirow{3}{*}{\rotatebox[origin=c]{90}{Outubro}}} &
		\parbox[t]{2mm}{\multirow{3}{*}{\rotatebox[origin=c]{90}{Novembro}}} &
		\parbox[t]{2mm}{\multirow{3}{*}{\rotatebox[origin=c]{90}{Dezembro}}} \\
		\hline 
		Definição do Pré projeto & \cellcolor[HTML]{000000} &  &  &  &  &  &  &  &  &  &  & \\ \hline %1ª linha da tabela
		
		Aprovação do Pré projeto &  & \cellcolor[HTML]{000000} &  &  &  &  &  &  &  &  &  & \\ \hline %2ª linha
		
		Levantamento bibliográfico & \cellcolor[HTML]{000000} & \cellcolor[HTML]{000000} & \cellcolor[HTML]{000000} &  &  &  &  &  &  &  &  & \\ \hline %3ª linha
		
		Primeira entrega do Pré projeto &  & \cellcolor[HTML]{000000} &  &  &  &  &  &  &  &  &  & \\ \hline %4ª linha
		
		Orientação sobre Introdução &  & \cellcolor[HTML]{000000} &  &  &  &  &  &  &  &  &  & \\ \hline %5ª linha
		
		Entrega da Introdução &  & \cellcolor[HTML]{000000} &  &  &  &  &  &  &  &  &  & \\ \hline %6ª linha
		
		Orientações sobre Objetivos e Justificativas &  & \cellcolor[HTML]{000000} &  &  &  &  &  &  &  &  &  & \\ \hline %7ª linha
		
		Entrega dos Objetivos e Justificativas &  & \cellcolor[HTML]{000000} &  &  &  &  &  &  &  &  &  & \\ \hline %8ª linha
		
		Orientações sobre o Quadro Teórico &  & \cellcolor[HTML]{000000} & \cellcolor[HTML]{000000} &  &  &  &  &  &  &  &  & \\ \hline %9ª linha
		
		Entrega do Quadro Teórico &  &  & \cellcolor[HTML]{000000} &  &  &  &  &  &  &  &  & \\ \hline %10ª linha
		
		Orientações sobre o Quadro Metodológico &  &  & \cellcolor[HTML]{000000} & \cellcolor[HTML]{000000} &  &  &  &  &  &  &  & \\ \hline %11ª linha 
		
		Entrega do Quadro Metodológico &  &  &  & \cellcolor[HTML]{000000} &  &  &  &  &  &  &  & \\ \hline %12ª linha 
		
		Revisão de Referências &  &  &  & \cellcolor[HTML]{000000} & \cellcolor[HTML]{000000} &  &  &  &  &  &  & \\ \hline %13ª linha
		
		Qualificação do projeto  &  &  &  &  & \cellcolor[HTML]{000000} &  &  &  &  &  &  & \\ \hline %14ª linha
		
		Levantamento de requisitos &  &  &  &  & \cellcolor[HTML]{000000} & \cellcolor[HTML]{000000} &  &  &  &  &  & \\ \hline %15ª linha
		
		Desenvolvimento da pesquisa &  &  &  &  &  & \cellcolor[HTML]{000000} & \cellcolor[HTML]{000000}  & \cellcolor[HTML]{000000} & \cellcolor[HTML]{000000} & \cellcolor[HTML]{000000} & \cellcolor[HTML]{000000} & \\ \hline %16ª linha
	\end{tabular}
	\caption{Cronograma do desenvolvimento do projeto. \textbf{Fonte:} Elaborado pelos autores}
\end{table}

%\newpage
\section{Orçamento}

\par Abaixo serão apresentadas as despesas de forma geral previstas para a realização deste projeto.

\begin{table}[h]
	\begin{center}
	  	\rowcolors{1}{}{lightgray} %Deixar a tabela zebrada
  		\begin{tabular}{|l|c|}
  			\hline
  			%\cellcolor[HTML]{E6E4E4} 
  			\textbf{Despesas} & 
  			%\cellcolor[HTML]{E6E4E4} 
  			\textbf{Valor Previsto} \\
  			\hline
  			Impressão 				& R\$ 70,00 \\ \hline
  			Encadernação 			& R\$ 35,00 \\ \hline
  			Impressão em capa dura	& R\$ 80,00 \\ \hline
  			Livros 					& R\$ 1.700,00 \\ \hline
  			%\cellcolor[HTML]{E6E4E4} 120 + 200 + 200 + 80 + 160 + 210 + 150 + 95 + 160 + 180 + 145
  			\textbf{Total}			& 
  			%\cellcolor[HTML]{E6E4E4}
  			\textbf{R\$ 1.885,00} \\ \hline
  		\end{tabular}
  	\end{center}
  	\caption{Orçamento previsto do projeto. \textbf{Fonte:} Elaborado pelos autores}
\end{table}

%\chapter{DISCUSSÃO DOS RESULTADOS} 

\par Neste capítulo são apresentados e discutidos os resultados obtidos por esta pesquisa e desenvolvimento do \textit{software}, apresentando os seus pontos positivos e negativos.

\par Inicialmente foram utilizadas as tecnologias Java, Primefaces e JSF, no entanto, viu-se a necessidade de alterar algumas destas tecnologias, visando à agilidade no processo de desenvolvimento. Desta forma, passou-se a utilizar HTML, CSS, Javascript em conjunto com o \textit{framework} Angular JS.

\par A mudança de tecnologias trouxe como benefício o desacoplamento das partes cliente (\textit{front-end}) e servidor (\textit{back end}), não sendo mais necessário recompilar, construir e publicar a aplicação no servidor \textit{web}, como era feito até então a cada alteração.

\par Após esta mudança, notou-se que a forma como o banco de dados era acessado deveria ser ajustada, a fim de seguir a mesma ideia proposta pela troca de tecnologias já mencionadas, uma vez que o banco de dados Neo4j permite duas maneiras de conexão, sendo elas: \textit{embedded} e por meio da API REST \cite{robinson_webber_eifrem_graph_databases}. Desta forma, passou-se a utilizar a API REST ao invés da forma \textit{embedded}, utilizada até então.

\par Com a mudança das tecnologias utilizadas no \textit{front end}, foi possível desenvolver uma aplicação com a interface \textit{clean} e moderna, permitindo ao usuário uma boa experiência de navegação. Também é possível citar, o uso do \textit{framework} Angular JS, que possibilitou não só o desenvolvimento ágil, como também a comunicação entre a aplicação e o \textit{web service}.

\par A linguagem Java, que foi utilizada para escrever todo o \textit{back end} deste trabalho, se mostrou uma ótima escolha, pois possui uma ampla biblioteca de materiais para apoio, disponibilizada pela Oracle, além de se integrar perfeitamente com o banco de dados Neo4j, que trás como exemplos, em sua documentação, aplicações que utilizam essas tecnologias em conjunto. Com a sua utilização obteve-se como resultado uma aplicação enquadrada nos padrões de desenvolvimento recomendados, além de ter um código legível e com qualidade, permitindo a utilização dos conceitos de orientação a objetos, o que facilitou o desenvolvimento da aplicação, pois permitiu o reaproveitamento de comportamentos por meio do conceito de herança e polimorfismo \cite{schildt_java_complete_reference}.

\par A fim de aplicar no desenvolvimento o mesmo conceito de relacionamentos, utilizado pela maioria das redes sociais, fez-se uso do banco de dados Neo4j em conjunto com a API \textit{Cypher}. Com o uso deste banco obteve-se como resultado uma aplicação cuja base de dados é toda orientada a grafos, deste modo, foi possível realizar consultas mais complexas de maneira simplificada, o que não seria possível com o uso de um banco de dados relacional, uma vez que para escrever este mesmo tipo de consulta seria necessário acessar várias tabelas por meio de \textit{joins}, deixando a consulta extensa e com baixa \textit{performance} \cite{sadalage_fowler_nosql_distilled_brief_guide}.

\par Neste escopo, concluiu-se o desenvolvimento desta aplicação, apresentando a seguir seus resultados mais relevantes.

\section{Localizar mão de obra}

\par Uma das funcionalidades implementadas no desenvolvimento foi a busca por mão de obra. Esta busca permite que o usuário encontre profissionais que desempenhem a função específica procurada por ele. Para realizar esta busca, foi preciso escrever uma consulta que leva em conta três níveis de análise, sendo elas, a busca pelo profissional que desempenhe o trabalho dentro da rede de parceiros do usuário, dentro da empresa onde o ele trabalha e por último, dentro da cidade onde vive.  

\par Para escrever esta consulta foi utilizada a tecnologia \textit{Cypher}, que é uma linguagem específica para o banco Neo4j \cite{neo4j_team_manual}. O Código~\ref{list:consulta_busca} apresenta a \textit{query} utilizada para realizar esta busca.

\par A Figura~\ref{fig:busca_domestica_edilson} demonstra a página apresentada ao usuário após realizar esta busca por um determinado serviço. No exemplo, é apresentado o resultado ao se realizar a busca por um profissional que desempenhe o serviço de doméstica.

\newpage
\begin{figure}[h!]
	\centerline{\includegraphics[scale=0.3]{./imagens/busca-domestica-edilson.png}}
	\caption[Página resultante da busca por doméstica.]
	{Página resultante da busca por doméstica. \textbf{Fonte:} Elaborado pelos autores.}
	\label{fig:busca_domestica_edilson}
\end{figure}

\par A ideia desta funcionalidade foi apresentar um possível prestador de serviços que já tenha sido avaliado por alguém em comum ao usuário solicitante, desta forma, há uma confiança maior em relação ao profissional que está sendo contratado. Portanto, essa busca irá apresentar diferentes prestadores de serviços, incluindo a média de cada um deles, a cada usuário solicitante.

%Seria uma conclusão e não uma discussão de resultado
%\par O uso de um banco de dados orientado a grafos como o Neo4j simplificou a forma de obter este resultado, uma vez que ao utilizar um banco relacional o mesmo resultado poderia ser obtido, porém o gasto em processamento e \textit{performance} seria extremamente maior. A API \textit{Cypher} facilitou a escrita de todas as consultas inclusive essa, por meio dela foi possível desenvolver consultas mais complexas como esta de forma mais rápida, agilizando o desenvolvimento da principal parte do trabalho.





\section{Apresentar possíveis parceiros}

\par Esta funcionalidade apresenta ao usuário possíveis parceiros que poderão compor a sua rede de relacionamentos. Para realizar esta busca, também foi preciso escrever uma consulta que levasse em conta dois níveis de análise, sendo elas, a busca pelo possível parceiro dentro da empresa onde o usuário trabalha, levando em conta a quantidade de parceiros em comum entre ambos (usuário autenticado e o possível parceiro) e a busca por parceiros dentro da mesma cidade, também seguindo este mesmo critério. O Código~\ref{list:consulta_possiveis_parceiros} apresenta a \textit{query} utilizada para realizar esta busca.

\par A Figura~\ref{fig:busca_possiveis_parceiros} apresenta o resultado da busca por possíveis parceiros, contendo uma lista com as pessoas que atendam os requisitos pré estabelecidos na consulta.

\newpage
\begin{figure}[h!]
	\centerline{\includegraphics[scale=0.4]{./imagens/busca-possiveis-parceiros.png}}
	\caption[Funcionalidade que apresenta a lista com possíveis parceiros.]
	{Funcionalidade que apresenta a lista com possíveis parceiros. \textbf{Fonte:} Elaborado pelos autores.}
	\label{fig:busca_possiveis_parceiros}
\end{figure}

\par A ideia desta funcionalidade foi apresentar um possível parceiro que provavelmente já tenha algum vínculo com o usuário ou que possua afinidade com alguém da sua rede de parceria.

\section{Localizar novos parceiros}

\par Esta busca permite ao usuário encontrar novos parceiros para compor a sua rede de relacionamentos. Para realizar esta busca, utilizou-se uma consulta muito semelhante a da funcionalidade anterior, acrescentando apenas o filtro para permitir a busca por nomes informado pelo usuário.

\par Para auxiliar o usuário nessa tarefa, foi criada uma nova consulta que leva em consideração apenas o nome do usuário, visando a sua localização independente de relacionamentos. Essa consulta é realizada apenas quando o número de resultados encontrados pelo nome informado pelo usuário for menor que um determinado valor definido, nesse caso, cinco.

\par A Figura~\ref{fig:busca_novos_parceiros} apresenta o resultado da busca por novos parceiros, contendo uma lista com as pessoas que atendam os requisitos pré estabelecidos na consulta.

\newpage
\begin{figure}[h!]
	\centerline{\includegraphics[scale=0.3]{./imagens/busca-novos-parceiros.png}}
	\caption[Funcionalidade que apresenta a busca por novos parceiros.]
	{Funcionalidade que apresenta a busca por novos parceiros. \textbf{Fonte:} Elaborado pelos autores.}
	\label{fig:busca_novos_parceiros}
\end{figure}

\par  Esta funcionalidade permite então que o usuário localize novos parceiros, que são indicados por ele e não sugeridos pelo sistema, como acontece na funcionalidade anterior.

\section{Solicitações de parcerias}

Essa funcionalidade visa facilitar o gerenciamento de requisições de novas parcerias solicatadas ao usuário autenticado no sistema, como apresenta a Figura~\ref{fig:aceitar_rejeitar_solicitacao_parceria}. Após o desenvolvimento dessa funcionalidade, o usuário autenticado no sistema passou a ter um local onde é possível visualizar, aceitar ou rejeitar as solicitações de novas parcerias enviadas a ele apenas com um clique, facilitando o gerenciamento dessas requisições.


\section{Avaliações de serviços}

A fim de disponibilizar uma ferramenta de auxílio à tomada de decisão aos usuários contratantes que procuram por um determinado provedor de serviço que exerça uma função requisitada por ele, foi desenvolvida essa funcionalidade que é detalhada na Figura~\ref{fig:taxa_avaliacao}. Ela apresenta a média das avaliações e os comentários realizados por parceiros do usuário autenticado no sistema, por pessoas que trabalhem na mesma empresa do usuário e das avaliações realizadas por pessoas que vivam na mesma cidade cujo usuário viva. Essas médias são apresentadas por meio de uma barra de progresso, a fim de facilitar o entendimento e demonstrar seus resultados de uma forma simples e clara, atingindo o seu objetivo principal que é auxiliar o usuário na tomada de decisão.

\section{Gráficos}

Foram desenvolvidos dois gráficos para auxiliarem os usuários (contratantes e/ou provedores) a tomarem as melhores decisões. No primeiro deles, cuja localização é a página de \textit{perfil} do provedor de serviço, apresentado na Figura~\ref{fig:grafico_pagina_perfil} é demonstrado a média das últimas avaliações que um provedor de serviço recebeu por um serviço provido por ele, em conjunto com a média das últimas avaliações de provedores de serviços que disponibilizam o mesmo serviço e que foram avaliados por parceiros do usuário autenticado no sistema. Com essas informações, o gráfico é gerado apresentando uma comparação entre as avaliações do provedor selecionado e outros provedores que também exerçam o mesmo trabalho.

No segundo gráfico apresentado na Figura~\ref{fig:grafico_pagina_inicial} é demonstrado a média das últimas avaliações recebidas pelo provedor de serviços para o serviço selecionado e a média das últimas avaliações de provedores de serviços que exerçam o mesmo trabalho e vivam na mesma cidade. Com essas informações, é gerado o gráfico apresentando essa comparação, auxiliando o provedor de serviço a entender como está a sua reputação perante a sua cidade.	



\par Neste capítulo foram apresentados os resultados mais relevantes obtidos por esta pesquisa, sendo eles satisfatórios, pois atenderam aos objetivos propostos por este trabalho.
%\chapter{CONCLUSÃO} 


\par O número de pessoas que utiliza a internet para realizar tarefas rotineiras vem crescendo nos últimos tempos e isso tem gerado um grande impacto social. Por meio de toda essa metamorfose, viu-se a necessidade de adaptar o tempo de execução de algumas tarefas, a fim de otimizá-lo.  Este trabalho propôs atuar nessa limitação, produzindo um serviço capaz de atender a necessidade na busca por profissionais temporários, sem vínculo empregatício formalizado. A escolha do tema justifica-se pela escassez de aplicações que ofereçam este tipo de auxílio aos usuários.

\par Visando propor uma possível alternativa para esta situação foram realizadas pesquisas e constatou-se que o modelo de negócios mais difundido seria o equiparado aos das redes sociais, tão comuns atualmente. Este modelo consiste em relacionamentos que podem ser facilmente resolvido aplicando a teoria dos grafos, onde cada usuário pode possuir vínculos de amizade com base em vários aspectos, seja por afinidades, cidade em que vive, empresa onde trabalha, entre tantos outros. Por se tratar de um trabalho que possui características de relacionamento parecidas com as mencionadas acima, este modelo foi escolhido para a realização deste trabalho, além de se tratar de um modelo bem aceito e conhecido, facilitando a sua aceitação no mercado.

\par Com o desenvolvimento deste trabalho foi possível observar a dificuldade encontrada pelas pessoas que buscam por estes tipos de profissionais, sendo que, muitos alegam que para efetuar a contratação é necessário ter alguma referência sobre o profissional que desempenhará o serviço. Desta forma, foi observado que as pessoas ainda são conservadoras nestes aspectos, escolhendo sempre profissionais que possuam boas referências, principalmente de pessoas próximas a elas. Em contra partida, foi observado também, que muitas vezes o profissional temporário não possui um espaço para ofertar o seu trabalho, o que os leva a ficar no anonimato e muitas vezes não conseguirem trabalho. Estes profissionais podem ser tanto os que preferem trabalhar de maneira informal, quanto os que não conseguem uma colocação ou recolocação no mercado de trabalho e precisam trabalhar.

\par  Por meio deste trabalho foi desenvolvido um ambiente \textit{web} que oferece suporte tanto ao contrante quanto o profissional temporário. Foram desenvolvidas funcionalidades específicas para cada tipo de usuário, sendo que a tecnologia escolhida para a realização foi de suma importância. Para fazer o armazenamento das informações optou-se por utilizar o banco de dados Neo4j, que como já mencionado no decorrer deste projeto, oferece várias vantagens, inclusive possuir uma base de dados orientada a grafos, o que permite várias formas de relacionamentos, se enquadrando na proposta de desenvolvimento deste trabalho. Para realizar as operações, foi utilizada a API \textit{Cypher}, que possibilitou a escrita das diversas consultas necessárias para retorno de informações e também para a inserção de dados no banco, de uma forma simples, clara e objetiva. 

\par De forma breve, foi observado, também, por meio deste trabalho, o desempenho do \textit{framework} Angular JS, que tornou o desenvolvimento muito mais ágil, uma vez que sua biblioteca traz várias funções pré moldadas.

\par Com a utilização destas tecnologias foram desenvolvidas funcionalidades como a possibilidade de criar uma rede de parcerias, realizando buscas por possíveis parceiros, sendo estes sugeridos pelo sistema com base em fatores comuns ou ainda buscando diretamente pelo nome. Também foi implementada a opção de buscar serviços, avaliá-los e comparar sua qualidade. Por meio da implementação dessas funcionalidades, observou-se que esse conjunto  atenderia não somente a justificativa inicial deste trabalho, como também daria a oportunidade de criação para novas ferramentas.

\par Visto que o tempo hábil para o desenvolvimento deste trabalho foi limitado, algumas funcionalidades ficaram por serem desenvolvidas. Um exemplo seria a opção de troca de mensagens entre os prestadores de serviço e contratantes. Esta funcionalidade permitiria que ambos pudessem se comunicar de forma rápida, a fim de agilizar o processo de contratação e definição das tarefas a ser realizadas. Com um tempo maior de estudos e testes, este ambiente poderia ter uma versão \textit{mobile}, o que facilitaria sua visualização em dispositivos móveis. 

\par Uma outra melhoria prevista é a de criar planos de contas. Estes planos visam separar em níveis os usuários cadastrados no sistema. A ideia é oferecer benefícios aos usuários que possuírem um nível maior que os demais. Estes benefícios serão obtidos mediante a alguma forma de pagamento ainda não definida.

\par Posterior ao desenvolvimento deste trabalho notou-se que esta aplicação causaria impacto social positivo, uma vez que oferece uma alternativa ao modo tradicional de se procurar por profissionais temporários. Com a utilização deste ambiente, há uma otimização do tempo de busca, pois o contato deles estão centralizados em um único lugar, sendo que eles são apresentados levando em consideração o nível de avaliação realizado por pessoas que fazem parte do círculo de parcerias do usuário que está realizando a busca. Outro impacto percebido foi que ao utilizar o sistema, o prestador de serviços teria uma maneira mais segura e confiável de divulgar os trabalhos exercidos.  

\par Por meio deste trabalho observou-se também a possibilidade de melhoria contínua, pois o mesmo poderá ser utilizado como base para futuros acadêmicos que tenham como objetivo realizar o estudo do banco de dados orientado a grafos, aplicado à solução de algum problema social. Este trabalho também estará disponível, juntamente com todos os códigos fontes, o que permite que novas funcionalidades sejam desenvolvidas para ele. Por se tratar de uma solução de baixo custo, a maioria da população poderá ter acesso a essa ferramenta de busca, o que irá disseminá-la.

\par Assim conclui-se que o desenvolvimento deste trabalho foi de grande valor, pois ofereceu auxílio a um problema tão comum nos dias atuais, além de proporcionar uma base maior de conhecimento aos seus pesquisadores se tratando da utilização de banco de dados orientado a grafos, tecnologia não pertencente a grade curricular.


\postextual %Início dos Elementos Pós-Textuais

%\citeoption{abnt-etal-cite=2}
\citeoption{ABNT-final}

%\bibliography{nuapatex-options, biblio} % insere as REFERENCIAS (arquivo biblio.bib)
\bibliography{biblio}			            % insere as REFERENCIAS (arquivo biblio.bib)
\addcontentsline{toc}{chapter}{REFERÊNCIAS} % adiciona o título no sumário

%\begin{apendicesenv}

%\apendicesname{APÊNDICES}
% Imprime uma página indicando o início dos apêndices
\partapendices*

\addcontentsline{toc}{chapter}{APÊNDICES}

\setcounter{figure}{0}
\setcounter{quadro}{0}
\chapter*{Apêndice 1: Iconix}
\label{ap1:iconix}

Neste Apêndice serão apresentados os artefatos gerados ao longo do processo de desenvolvimento deste trabalho.

\section*{Fluxos de eventos}

Após a criação do diagrama de casos de uso, foram gerados os fluxos de eventos para cada caso de uso, como serão apresentados a seguir. O primeiro fluxo de eventos gerado é referente ao caso de uso ''Aceitar parceria'', conforme apresenta o Quadro~\ref{ap1:quad:fluxo_evento_aceitar_parceria}.

\captionsetup[quadro]{list=no}
\begin{quadro}[h!]
	\input{./fluxos/fluxo-evento-aceitar-parceria}
	\caption[Fluxo de eventos para o caso de uso ''Aceitar parceria''.]
	{Fluxo de eventos para o caso de uso ''Aceitar parceria''. \textbf{Fonte:} Elaborado pelos autores}
	\label{ap1:quad:fluxo_evento_aceitar_parceria}
\end{quadro}

O próximo fluxo de eventos gerado é referente ao caso de uso ''Localizar parceiro'', conforme apresenta o Quadro~\ref{ap1:quad:fluxo_evento_localizar_parceiro}.

\captionsetup[quadro]{list=no}
\begin{quadro}[h!]
	\input{./fluxos/fluxo-evento-localizar-parceiro}
	\caption[Fluxo de eventos para o caso de uso ''Localizar parceiro''.]
	{Fluxo de eventos para o caso de uso ''Localizar parceiro''. \textbf{Fonte:} Elaborado pelos autores}
	\label{ap1:quad:fluxo_evento_localizar_parceiro}
\end{quadro}

O próximo fluxo de eventos gerado é referente ao caso de uso ''Adicionar parceiro'', conforme apresenta o Quadro~\ref{ap1:quad:fluxo_evento_adicionar_parceiro}.

\newpage
\captionsetup[quadro]{list=no}
\begin{quadro}[h!]
	\input{./fluxos/fluxo-evento-adicionar-parceiro}
	\caption[Fluxo de eventos para o caso de uso ''Adicionar parceiro''.]
	{Fluxo de eventos para o caso de uso ''Adicionar parceiro''. \textbf{Fonte:} Elaborado elos autores}
	\label{ap1:quad:fluxo_evento_adicionar_parceiro}
\end{quadro}

O próximo fluxo de eventos gerado é referente ao caso de uso ''Localizar mão de obra'', conforme apresenta o Quadro~\ref{ap1:quad:fluxo_evento_localizar_mao_de_obra}.

\newpage
\captionsetup[quadro]{list=no}
\begin{quadro}[h!]
	\input{./fluxos/fluxo-evento-localizar-mao-de-obra}
	\caption[Fluxo de eventos para o caso de uso ''Localizar mão de obra'']
	{Fluxo de eventos para o caso de uso ''Localizar mão de obra''. \textbf{Fonte:} Elaborado pelos autores}
	\label{ap1:quad:fluxo_evento_localizar_mao_de_obra}
\end{quadro}

O próximo fluxo de eventos gerado é referente ao caso de uso ''Avaliar mão de obra'', conforme apresenta o Quadro~\ref{ap1:quad:fluxo_evento_avaliar_mao_de_obra}.

\newpage
\captionsetup[quadro]{list=no}
\begin{quadro}[h!]
	\input{./fluxos/fluxo-evento-avaliar-mao-de-obra}
	\caption[Fluxo de eventos para o caso de uso ''Avaliar mão de obra''.]
	{Fluxo de eventos para o caso de uso ''Avaliar mão de obra''. \textbf{Fonte:} Elaborado pelos autores}
	\label{ap1:quad:fluxo_evento_avaliar_mao_de_obra}
\end{quadro}

O próximo fluxo de eventos gerado é referente ao caso de uso ''Gerenciar serviços'', conforme apresenta o Quadro~\ref{ap1:quad:fluxo_evento_gerenciar_servicos}.

\newpage
\captionsetup[quadro]{list=no}
\begin{quadro}[h!]
	\input{./fluxos/fluxo-evento-gerenciar-servicos}
	\caption[Fluxo de eventos para o caso de uso ''Gerenciar serviços''.]
	{Fluxo de eventos para o caso de uso ''Gerenciar serviços''. \textbf{Fonte:} Elaborado pelos autores}
	\label{ap1:quad:fluxo_evento_gerenciar_servicos}
\end{quadro}

O último fluxo de eventos gerado é referente ao caso de uso ''Criar conta'', conforme apresenta o Quadro~\ref{ap1:quad:fluxo_evento_criar_conta}.

\newpage
\begin{quadro}[h!]
	\input{./fluxos/fluxo-evento-criar-conta}
	\caption[Fluxo de eventos para o caso de uso ''Criar conta''.]
	{Fluxo de eventos para o caso de uso ''Criar conta''. \textbf{Fonte:} Elaborado pelos autores}
	\label{ap1:quad:fluxo_evento_criar_conta}
\end{quadro}

Após a criação dos fluxos de evento, foram criados os diagramas de robustez que serão apresentados a seguir.

\section*{Diagramas de Robustez}

Os diagramas de robustez foram criados baseado-se no diagrama de caso de uso, somado aos fluxos de eventos demonstrados na seção anterior. Seguindo a mesma ordem dos fluxos de eventos, o primeiro diagrama de robustez apresentado na Figura~\ref{fig:ap1:diagrama_robustez_aceitar_parceria} é referente ao caso de uso ''Aceitar parceria''.

\newpage
\captionsetup[figure]{list=no}
\begin{figure}[h!]
	\centerline{\includegraphics[scale=0.38]{./imagens/apendices/diagrama-robustez-aceitar-parceria.png}}
	\caption[Diagrama de robustez referente ao caso de uso ''Aceitar parceria''.]
	{Diagrama de robustez referente ao caso de uso ''Aceitar parceria''. \textbf{Fonte:} Elaborado pelos autores.}
	\label{fig:ap1:diagrama_robustez_aceitar_parceria}
\end{figure}

O próximo diagrama de robustez apresentado na Figura~\ref{fig:ap1:diagrama_robustez_localizar_parceiro} é referente ao caso de uso ''Localizar parceiro''.

\captionsetup[figure]{list=no}
\begin{figure}[h!]
	\centerline{\includegraphics[scale=0.37]{./imagens/apendices/diagrama-robustez-localizar-parceiros.png}}
	\caption[Diagrama de robustez referente ao caso de uso ''Localizar parceiro''.]
	{Diagrama de robustez referente ao caso de uso ''Localizar parceiro''. \textbf{Fonte:} Elaborado pelos autores.}
	\label{fig:ap1:diagrama_robustez_localizar_parceiro}
\end{figure}

O próximo diagrama de robustez apresentado na Figura~\ref{fig:ap1:diagrama_robustez_adicionar_parceiro} é referente ao caso de uso ''Adicionar parceiro''.

\newpage
\captionsetup[figure]{list=no}
\begin{figure}[h!]
	\centerline{\includegraphics[scale=0.35]{./imagens/apendices/diagrama-robustez-adicionar-parceiro.png}}
	\caption[Diagrama de robustez referente ao caso de uso ''Adicionar parceiro''.]
	{Diagrama de robustez referente ao caso de uso ''Adicionar parceiro''. \textbf{Fonte:} Elaborado pelos autores.}
	\label{fig:ap1:diagrama_robustez_adicionar_parceiro}
\end{figure}

O próximo diagrama de robustez apresentado na Figura~\ref{fig:ap1:diagrama_robustez_localizar_mao_de_obra} é referente ao caso de uso ''Localizar mão de obra''.

\captionsetup[figure]{list=no}
\begin{figure}[h!]
	\centerline{\includegraphics[scale=0.35]{./imagens/apendices/diagrama-robustez-localizar-mao-de-obra.png}}
	\caption[Diagrama de robustez referente ao caso de uso ''Localizar mão de obra''.]
	{Diagrama de robustez referente ao caso de uso ''Localizar mão de obra''. \textbf{Fonte:} Elaborado pelos autores.}
	\label{fig:ap1:diagrama_robustez_localizar_mao_de_obra}
\end{figure}

O próximo diagrama de robustez apresentado na Figura~\ref{fig:ap1:diagrama_robustez_avaliar_mao_de_obra} é referente ao caso de uso ''Avaliar mão de obra''.

\newpage
\captionsetup[figure]{list=no}
\begin{figure}[h!]
	\centerline{\includegraphics[scale=0.43]{./imagens/apendices/diagrama-robustez-avaliar-mao-de-obra.png}}
	\caption[Diagrama de robustez referente ao caso de uso ''Avaliar mão de obra''.]
	{Diagrama de robustez referente ao caso de uso ''Avaliar mão de obra''. \textbf{Fonte:} Elaborado pelos autores.}
	\label{fig:ap1:diagrama_robustez_avaliar_mao_de_obra}
\end{figure}

O próximo diagrama de robustez apresentado na Figura~\ref{fig:ap1:diagrama_robustez_gerenciar_servicos} é referente ao caso de uso ''Gerenciar serviços''.

\captionsetup[figure]{list=no}
\begin{figure}[h!]
	\centerline{\includegraphics[scale=0.38]{./imagens/apendices/diagrama-robustez-adicionar-servicos.png}}
	\caption[Diagrama de robustez referente ao caso de uso ''Gerenciar serviços''.]
	{Diagrama de robustez referente ao caso de uso ''Gerenciar serviços''. \textbf{Fonte:} Elaborado pelos autores.}
	\label{fig:ap1:diagrama_robustez_gerenciar_servicos}
\end{figure}

O último diagrama de robustez apresentado na Figura~\ref{fig:ap1:diagrama_robustez_criar_conta} é referente ao caso de uso ''Criar conta''.

\newpage
\captionsetup[figure]{list=no}
\begin{figure}[h!]
	\centerline{\includegraphics[scale=0.4]{./imagens/apendices/diagrama-robustez-criar-conta.png}}
	\caption[Diagrama de robustez referente ao caso de uso ''Criar conta''.]
	{Diagrama de robustez referente ao caso de uso ''Criar conta''. \textbf{Fonte:} Elaborado pelos autores.}
	\label{fig:ap1:diagrama_robustez_criar_conta}
\end{figure}

Após a criação dos diagramas de robustez, o diagrama de modelo de domínio foi atualizado, adicionando os atributos identificados pelos diagramas de caso de robustez, passando-se a trabalhar nos diagramas de sequência, que seão apresentados a seguir.

\section*{Diagramas de Sequência}

Os diagramas de sequência foram criados baseado-se no diagramas apresentados anteriormente. Seguindo a mesma ordem anteriormente definida, o primeiro diagrama de sequência apresentado na Figura~\ref{fig:ap1:diagrama_sequencia_aceitar_parceria} é referente ao caso de uso ''Aceitar parceria''.

\newpage
\captionsetup[figure]{list=no}
\begin{figure}[h!]
	\centerline{\includegraphics[angle=90,scale=0.42]{./imagens/apendices/diagrama-sequencia-aceitar-parceria.png}}
	\caption[Diagrama de sequência referente ao caso de uso ''Aceitar parceria''.]
	{Diagrama de sequência referente ao caso de uso ''Aceitar parceria''. \textbf{Fonte:} Elaborado pelos autores.}
	\label{fig:ap1:diagrama_sequencia_aceitar_parceria}
\end{figure}

O próximo diagrama de sequência apresentado na Figura~\ref{fig:ap1:diagrama_sequencia_localizar_parceiro} é referente ao caso de uso ''Localizar parceiro''.

\newpage
\captionsetup[figure]{list=no}
\begin{figure}[h!]
	\centerline{\includegraphics[angle=90,scale=0.42]{./imagens/apendices/diagrama-sequencia-localizar-parceiros.png}}
	\caption[Diagrama de sequência referente ao caso de uso ''Localizar parceiro''.]
	{Diagrama de sequência referente ao caso de uso ''Localizar parceiro''. \textbf{Fonte:} Elaborado pelos autores.}
	\label{fig:ap1:diagrama_sequencia_localizar_parceiro}
\end{figure}

O próximo diagrama de sequência apresentado na Figura~\ref{fig:ap1:diagrama_sequencia_adicionar_parceiro} é referente ao caso de uso ''Adicionar parceiro''.

\begin{landscape}
\newpage
\captionsetup[figure]{list=no}
\begin{figure}[h!]
	\centerline{\includegraphics[scale=0.3]{./imagens/apendices/diagrama-sequencia-adicionar-parceiros.png}}
	\caption[Diagrama de sequência referente ao caso de uso ''Adicionar parceiro''.]
	{Diagrama de sequência referente ao caso de uso ''Adicionar parceiro''. \textbf{Fonte:} Elaborado pelos autores.}
	\label{fig:ap1:diagrama_sequencia_adicionar_parceiro}
\end{figure}

O próximo diagrama de sequência apresentado na Figura~\ref{fig:ap1:diagrama_sequencia_localizar_mao_de_obra} é referente ao caso de uso ''Localizar mão de obra''.

\newpage
\captionsetup[figure]{list=no}
\begin{figure}[h!]
	\centerline{\includegraphics[scale=0.4]{./imagens/apendices/diagrama-sequencia-localizar-mao-de-obra.png}}
	\caption[Diagrama de sequência referente ao caso de uso ''Localizar mão de obra''.]
	{Diagrama de sequência referente ao caso de uso ''Localizar mão de obra''. \textbf{Fonte:} Elaborado pelos autores.}
	\label{fig:ap1:diagrama_sequencia_localizar_mao_de_obra}
\end{figure}

O próximo diagrama de sequência apresentado na Figura~\ref{fig:ap1:diagrama_sequencia_avaliar_mao_de_obra} é referente ao caso de uso ''Avaliar mão de obra''.

\newpage
\captionsetup[figure]{list=no}
\begin{figure}[h!]
	\centerline{\includegraphics[scale=0.2]{./imagens/apendices/diagrama-sequencia-avaliar-mao-de-obra.png}}
	\caption[Diagrama de sequência referente ao caso de uso ''Avaliar mão de obra''.]
	{Diagrama de sequência referente ao caso de uso ''Avaliar mão de obra''. \textbf{Fonte:} Elaborado pelos autores.}
	\label{fig:ap1:diagrama_sequencia_avaliar_mao_de_obra}
\end{figure}

O próximo diagrama de sequência apresentado na Figura~\ref{fig:ap1:diagrama_sequencia_gerenciar_servicos} é referente ao caso de uso ''Gerenciar serviços''.

\newpage
\captionsetup[figure]{list=no}
\begin{figure}[h!]
	\centerline{\includegraphics[scale=0.4]{./imagens/apendices/diagrama-sequencia-gerenciar-servicos.png}}
	\caption[Diagrama de sequência referente ao caso de uso ''Gerenciar serviços''.]
	{Diagrama de sequência referente ao caso de uso ''Gerenciar serviços''. \textbf{Fonte:} Elaborado pelos autores.}
	\label{fig:ap1:diagrama_sequencia_gerenciar_servicos}
\end{figure}

O último diagrama de sequência apresentado na Figura~\ref{fig:ap1:diagrama_sequencia_criar_conta} é referente ao caso de uso ''Criar conta''.

\captionsetup[figure]{list=no}
\begin{figure}[h!]
	\centerline{\includegraphics[scale=0.35]{./imagens/apendices/diagrama-sequencia-criar-conta.png}}
	\caption[Diagrama de sequência referente ao caso de uso ''Criar conta''.]
	{Diagrama de sequência referente ao caso de uso ''Criar conta''. \textbf{Fonte:} Elaborado pelos autores.}
	\label{fig:ap1:diagrama_sequencia_criar_conta}
\end{figure}
\end{landscape}

Após a criação dos diagramas de sequência, o diagrama de modelo de domínio foi atualizado, adicionando os atributos identificados pelos diagramas de sequência, gerando o diagrama de classes final.

\chapter*{Apêndice 2: API Rest}
\label{apendice:api_rest}

Nesse Apêndice será apresentado o contrato de serviços providos pelo \textit{Web Service} desenvolvido neste trabalho. Os serviços foram divididos em categorias, a fim de facilitar a compreensão dos contratos.

\section*{Estados e capitais}

\begin{lstlisting} [style=custom_XML,title={Contrato de serviço referente a estados e capitais. \textbf{Fonte:} Elaborado pelos autores.}, label=list:contrato_estados] 	
<application xmlns="http://wadl.dev.java.net/2009/02">
	<doc xml:lang="en" title="http://localhost:8080"/>
	<resources base="http://localhost:8080">
		<resource path="WebService/uf" id="uf">
			<doc xml:lang="en" title="uf"/>
			<method name="GET" id="uf">
				<doc xml:lang="en" title="uf"/>
				<request/>
				<response>
					<representation mediaType="application/json"/>
				</response>
			</method>
		</resource>
	<resources>
</application>
\end{lstlisting}


\section*{Cidades}

\begin{lstlisting} [style=custom_XML,title={Contrato de serviço referente a cidades. \textbf{Fonte:} Elaborado pelos autores.}, label=list:contrato_cidade] 	
<application xmlns="http://wadl.dev.java.net/2009/02">
	<doc xml:lang="en" title="http://localhost:8080"/>
	<resources base="http://localhost:8080">
		<resource path="WebService/city/cities/{state}" id="city">
			<doc xml:lang="en" title="city"/>
			<param name="state" type="xs:string" required="true" 
				default="" style="template" 
				xmlns:xs="http://www.w3.org/2001/XMLSchema"/>
			<method name="GET" id="city-cities">
				<doc xml:lang="en" title="city-cities"/>
				<request/>
				<response>
					<representation mediaType="application/json"/>
				</response>
			</method>
		</resource>
	<resources>
</application>
\end{lstlisting}


\section*{Sessão}

\begin{lstlisting} [style=custom_XML,title={Contrato de serviço referente a sessão do usuário autenticado. \textbf{Fonte:} Elaborado pelos autores.}, label=list:contrato_sessao] 	
<application xmlns="http://wadl.dev.java.net/2009/02">
	<doc xml:lang="en" title="http://localhost:8080"/>
	<resources base="http://localhost:8080">
		<resource path="WebService/session/userinfo/{token}" id="session">
			<doc xml:lang="en" title="session"/>
			<param name="token" type="xs:string" required="false" 
			default="" style="template" 
			xmlns:xs="http://www.w3.org/2001/XMLSchema"/>
			<method name="POST" id="session-login">
				<doc xml:lang="en" title="session-login"/>
				<request>
					<representation mediaType="application/json"/>
				</request>
				<response>
					<representation mediaType="application/json"/>
				</response>
			</method>
			<method name="GET" id="session-userinfo">
				<doc xml:lang="en" title="session-userinfo"/>
				<request/>
				<response>
					<representation mediaType="application/json"/>
				</response>
			</method>
		</resource>
	</resources>
</application>
\end{lstlisting}

\section*{Serviços}

\begin{lstlisting} [style=custom_XML,title={Contrato de serviço referente a serviços. \textbf{Fonte:} Elaborado pelos autores.}, label=list:contrato_servicos] 	
<application xmlns="http://wadl.dev.java.net/2009/02">
	<doc xml:lang="en" title="http://localhost:8080"/>
	<resources base="http://localhost:8080">
		<resource path="WebService/services/service/{name}" id="services">
			<doc xml:lang="en" title="services"/>
			<param name="name" type="xs:string" required="true"
			 default="" style="template" 
			 xmlns:xs="http://www.w3.org/2001/XMLSchema"/>
			<method name="GET" id="services-service">
				<doc xml:lang="en" title="services-service"/>
				<request/>
				<response>
					<representation mediaType="application/json"/>
				</response>
			</method>
		</resource>
	</resources>
</application>
\end{lstlisting}

\section*{Provedores de serviços}

\begin{lstlisting} [style=custom_XML,title={Contrato de serviço referente a provedores de serviços. \textbf{Fonte:} Elaborado pelos autores.}, label=list:contrato_provedores_servicos] 	
<application xmlns="http://wadl.dev.java.net/2009/02">
	<doc xml:lang="en" title="http://localhost:8080"/>
	<resource path="WebService/serviceprovider/removeservice" id="serviceprovider">
		<doc xml:lang="en" title="serviceprovider"/>
		<method name="GET" id="serviceprovider-byservice">
			<doc xml:lang="en" title="serviceprovider-byservice"/>
			<request/>
			<response>
				<representation mediaType="application/json"/>
			</response>
		</method>
		<method name="GET" id="serviceprovider-ratingInMyNetworkPartners">
			<doc xml:lang="en" title="serviceprovider-ratingInMyNetworkPartners"/>
			<request/>
			<response>
				<representation mediaType="application/json"/>
			</response>
		</method>
		<method name="GET" id="serviceprovider-ratingInMyCompany">
			<doc xml:lang="en" title="serviceprovider-ratingInMyCompany"/>
			<request/>
			<response>
				<representation mediaType="application/json"/>
			</response>
		</method>
		<method name="GET" id="serviceprovider-ratingInMyCity">
			<doc xml:lang="en" title="serviceprovider-ratingInMyCity"/>
			<request/>
			<response>
				<representation mediaType="application/json"/>
			</response>
		</method>
		<method name="GET" id="serviceprovider-data">
			<doc xml:lang="en" title="serviceprovider-data"/>
			<request/>
			<response>
				<representation mediaType="application/json"/>
			</response>
		</method>
		<method name="GET" id="serviceprovider-myservices">
			<doc xml:lang="en" title="serviceprovider-myservices"/>
			<request/>
			<response>
				<representation mediaType="application/json"/>
			</response>
		</method>
		<method name="POST" id="serviceprovider-addservice">
			<doc xml:lang="en" title="serviceprovider-addservice"/>
			<request>
				<representation mediaType="application/json"/>
			</request>
			<response>
				<representation mediaType="application/json"/>
			</response>
		</method>
		<method name="POST" id="serviceprovider-removeservice">
			<doc xml:lang="en" title="serviceprovider-removeservice"/>
			<request>
				<representation mediaType="application/json"/>
			</request>
			<response>
				<representation mediaType="application/json"/>
			</response>
		</method>
	</resource>
</application>
\end{lstlisting}

\section*{Avaliações de serviços}

\begin{lstlisting} [style=custom_XML,title={Contrato de serviço referente a avaliações de serviços. \textbf{Fonte:} Elaborado pelos autores.}, label=list:contrato_avaliacao_servicos] 	
<application xmlns="http://wadl.dev.java.net/2009/02">
	<doc xml:lang="en" title="http://localhost:8080"/>
	<resource path="WebService/rating/mylastestratings/{token}" id="rating">
		<doc xml:lang="en" title="rating"/>
		<param name="token" type="xs:string" required="true" 
		default="" style="template" 
		xmlns:xs="http://www.w3.org/2001/XMLSchema"/>
		<method name="POST" id="rating-save">
			<doc xml:lang="en" title="rating-save"/>
			<request>
				<representation mediaType="application/json"/>
			</request>
			<response>
				<representation mediaType="application/json"/>
			</response>
		</method>
		<method name="GET" id="rating-mylastestratings">
			<doc xml:lang="en" title="rating-mylastestratings"/>
			<request/>
			<response>
				<representation mediaType="application/json"/>
			</response>
		</method>
	</resource>	
</application>
\end{lstlisting}

\section*{Informações pessoais}

\begin{lstlisting} [style=custom_XML,title={Contrato de serviço referente a informações pessoais. \textbf{Fonte:} Elaborado pelos autores.}, label=list:contrato_informacoes_pessoais] 	
<application xmlns="http://wadl.dev.java.net/2009/02">
	<resource path="WebService/person/persondata/{partner}" id="person">
		<doc xml:lang="en" title="person"/>
		<param name="partner" type="xs:string" required="true"
		 default="" style="template" 
		 xmlns:xs="http://www.w3.org/2001/XMLSchema"/>
		<param name="token" type="xs:string" required="true"
		 default="" style="query" 
		 xmlns:xs="http://www.w3.org/2001/XMLSchema"/>
		<method name="POST" id="person-createaccount-personaldata">
			<doc xml:lang="en" title="person-createaccount-personaldata"/>
			<request>
				<representation mediaType="application/json"/>
			</request>
			<response>
				<representation mediaType="application/json"/>
			</response>
		</method>
		<method name="POST" id="person-createaccount-workdata">
			<doc xml:lang="en" title="person-createaccount-workdata"/>
			<request>
				<representation mediaType="application/json"/>
			</request>
			<response>
				<representation mediaType="application/json"/>
			</response>
		</method>
		<method name="GET" id="person-persondata">
			<doc xml:lang="en" title="person-persondata"/>
			<request/>
			<response>
				<representation mediaType="application/json"/>
			</response>
		</method>
	</resource>
</application>
\end{lstlisting}


\section*{Parceiros}

\begin{lstlisting} [style=custom_XML,title={Contrato de serviço referente a informações de paceiros. \textbf{Fonte:} Elaborado pelos autores.}, label=list:contrato_informacoes_de_parceiros] 	
<application xmlns="http://wadl.dev.java.net/2009/02">
	<resource path="WebService/partner/commonspartner/{partner}" id="partner">
		<doc xml:lang="en" title="partner"/>
		<param name="partner" type="xs:string" required="true" 
		default="" style="template" 
		xmlns:xs="http://www.w3.org/2001/XMLSchema"/>
		<param name="token" type="xs:string" required="true" 
		default="" style="query" 
		xmlns:xs="http://www.w3.org/2001/XMLSchema"/>
		<method name="GET" id="partner-possiblepartners">
			<doc xml:lang="en" title="partner-possiblepartners"/>
			<request/>
			<response>
				<representation mediaType="application/json"/>
			</response>
		</method>
		<method name="GET" id="partner-allpartners">
			<doc xml:lang="en" title="partner-allpartners"/>
			<request/>
			<response>
				<representation mediaType="application/json"/>
			</response>
		</method>
		<method name="POST" id="partner-add">
			<doc xml:lang="en" title="partner-add"/>
			<request>
				<representation mediaType="application/json"/>
			</request>
			<response>
				<representation mediaType="application/json"/>
			</response>
		</method>
		<method name="POST" id="partner-cancel">
			<doc xml:lang="en" title="partner-cancel"/>
			<request>
				<representation mediaType="application/json"/>
			</request>
			<response>
				<representation mediaType="application/json"/>
			</response>
		</method>
		<method name="GET" id="partner-allpartnerrequest">
			<doc xml:lang="en" title="partner-allpartnerrequest"/>
			<request/>
			<response>
				<representation mediaType="application/json"/>
			</response>
		</method>
		<method name="POST" id="partner-acceptpartnerrequest">
			<doc xml:lang="en" title="partner-acceptpartnerrequest"/>
			<request>
				<representation mediaType="application/json"/>
			</request>
			<response>
				<representation mediaType="application/json"/>
			</response>
		</method>
		<method name="POST" id="partner-rejectpartnerrequest">
			<doc xml:lang="en" title="partner-rejectpartnerrequest"/>
			<request>
				<representation mediaType="application/json"/>
			</request>
			<response>
				<representation mediaType="application/json"/>
			</response>
		</method>
		<method name="POST" id="partner-searchnewpartners">
			<doc xml:lang="en" title="partner-searchnewpartners"/>
			<request>
				<representation mediaType="application/json"/>
			</request>
			<response>
				<representation mediaType="application/json"/>
			</response>
		</method>
		<method name="POST" id="partner-searchnewpartnersonlybyname">
			<doc xml:lang="en" title="partner-searchnewpartnersonlybyname"/>
			<request>
				<representation mediaType="application/json"/>
			</request>
			<response>
				<representation mediaType="application/json"/>
			</response>
		</method>
		<method name="GET" id="partner-ismypartner">
			<doc xml:lang="en" title="partner-ismypartner"/>
			<request/>
			<response>
				<representation mediaType="application/json"/>
			</response>
		</method>
		<method name="GET" id="partner-commonspartner">
			<doc xml:lang="en" title="partner-commonspartner"/>
			<request/>
			<response>
				<representation mediaType="application/json"/>
			</response>
		</method>
	</resource>
</application>
\end{lstlisting}


\section*{Últimas atualizações}

\begin{lstlisting} [style=custom_XML,title={Contrato de serviço referente as últimas atualizações. \textbf{Fonte:} Elaborado pelos autores.}, label=list:contrato_ultimas_atualizacoes] 	
<application xmlns="http://wadl.dev.java.net/2009/02">
	<resource path="WebService/feed/lastestratings/{token}" id="feed">
		<doc xml:lang="en" title="feed"/>
		<param name="token" type="xs:string" required="false" 
		default="" style="template" 
		xmlns:xs="http://www.w3.org/2001/XMLSchema"/>
		<method name="GET" id="feed-lastestpartnership">
			<doc xml:lang="en" title="feed-lastestpartnership"/>
			<request/>
			<response>
				<representation mediaType="application/json"/>
			</response>
		</method>
		<method name="GET" id="feed-lastestratings">
			<doc xml:lang="en" title="feed-lastestratings"/>
			<request/>
			<response>
				<representation mediaType="application/json"/>
			</response>
		</method>
	</resource>
</application>
\end{lstlisting}

\section*{Gráficos}

\begin{lstlisting} [style=custom_XML,title={Contrato de serviço referente aos gráficos. \textbf{Fonte:} Elaborado pelos autores.}, label=list:contrato_graficos] 	
<application xmlns="http://wadl.dev.java.net/2009/02">
	<resource path="WebService/report/lastEvaluateInMyCity" id="report">
		<doc xml:lang="en" title="report"/>
		<param name="token" type="xs:string" required="true"
		 style="query" xmlns:xs="http://www.w3.org/2001/XMLSchema"/>
		<param name="service" type="xs:string" required="true"
		 style="query" xmlns:xs="http://www.w3.org/2001/XMLSchema"/>
		<param name="limit" type="xs:string" required="true"
		 style="query" xmlns:xs="http://www.w3.org/2001/XMLSchema"/>
		<method name="GET" id="report-lastEvaluateOfServiceProvider">
			<doc xml:lang="en" title="report-lastEvaluateOfServiceProvider"/>
			<request/>
			<response status="200">
				<representation mediaType="application/json"/>
			</response>
		</method>
		<method name="GET" id="report-lastEvaluateOfServiceInNetwork">
			<doc xml:lang="en" title="report-lastEvaluateOfServiceInNetwork"/>
			<request/>
			<response status="200">
				<representation mediaType="application/json"/>
			</response>
		</method>
		<method name="GET" id="report-lastEvaluate">
			<doc xml:lang="en" title="report-lastEvaluate"/>
			<request/>
			<response status="200">
				<representation mediaType="application/json"/>
			</response>
		</method>
		<method name="GET" id="report-lastEvaluateInMyCity">
			<doc xml:lang="en" title="report-lastEvaluateInMyCity"/>
			<request/>
			<response status="200">
				<representation mediaType="application/json"/>
			</response>
		</method>
	</resource>
</application>
\end{lstlisting}

Com este contrato definido foi possível dar continuidade no processo de desenvolvimento da aplicação.
\chapter*{Apêndice 3: Controlador de versão}
\label{ap3:github}

Neste Apêndice serão apresentados os passos necessários para a criação do repositório utilizado no sistema de controle de versão, junto com a instalação da ferramenta utilizada para trabalhar com este controlador de versão e sua configuração.

Para criar um repositório no GitHub (ferramenta de controle de versão) utilizado neste trabalho, deve-se acessar a  \textit{url} \texttt{http://github.com}, por meio de um navegador de internet e clicar no botão \textit{"Sign In"}, caso possua conta, caso contrário clique no botão \textit{"Sign up"} e crie sua conta. A Figura~\ref{fig:ap3:pagina_inicial_github} apresenta a página inicial do GitHub.

\captionsetup[figure]{list=no}
\begin{figure}[h!]
	\centerline{\includegraphics[scale=0.5]{./imagens/apendices/pagina-inicial-github.png}}
	\caption[Página inicial do GitHub.]
	{Página inicial do GitHub. \textbf{Fonte:} Elaborado pelos autores.}
	\label{fig:ap3:pagina_inicial_github}
\end{figure}

Para prosseguir com o processo de criação do repositório (com a conta já criada), deve-se clicar no botão \textit{"Sign In"} e realizar o \textit{login}. Após a conclusão desses passos a página inicial, contendo a lista de repositórios do usuário será apresentada conforme a Figura~\ref{fig:ap3:pagina_home_github} apresenta.

\newpage
\captionsetup[figure]{list=no}
\begin{figure}[h!]
	\centerline{\includegraphics[scale=0.5]{./imagens/apendices/pagina-home-github.png}}
	\caption[Página com a lista de repositórios do usuário no GitHub.]
	{Página com a lista de repositórios do usuário no GitHub. \textbf{Fonte:} Elaborado pelos autores.}
	\label{fig:ap3:pagina_home_github}
\end{figure}


Na página inicial deve-se clicar no botão \textit{"+ New Repository"} para que a página de criação do novo repositório seja apresentada conforme a Figura~\ref{fig:ap3:pagina_criacao_repository_github} apresenta.
\newpage
\captionsetup[figure]{list=no}
\begin{figure}[h!]
	\centerline{\includegraphics[scale=0.5]{./imagens/apendices/pagina-criacao-repositorio-github.png}}
	\caption[Página de criação de repositório no GitHub.]
	{Página de criação de repositório no GitHub. \textbf{Fonte:} Elaborado pelos autores.}
	\label{fig:ap3:pagina_criacao_repository_github}
\end{figure}

Nessa página deve-se informar os dados referentes ao repositório e, após preencher o formulário clique no botão \textit{"Create repository"}, a fim de concluir o processo de criação do repositório no GitHub.

Após criado o repositório, foi necessário adicionar os autores deste trabalho como colaboradores para que ambos pudessem modificar arquivos. Para fazer esta configuração é necessário clicar no menu \textit{"Settings"} da página inicial do repositório conforme apresenta a Figura~\ref{fig:ap3:pagina_inicial_repositororio_github}.

\newpage
\captionsetup[figure]{list=no}
\begin{figure}[h!]
	\centerline{\includegraphics[scale=0.5]{./imagens/apendices/pagina-inicial-repositorio-github.png}}
	\caption[Página do repositório no GitHub.]
	{Página do repositório no GitHub. \textbf{Fonte:} Elaborado pelos autores.}
	\label{fig:ap3:pagina_inicial_repositororio_github}
\end{figure}

Após clicar no menu \textit{"Settings"} é necessário navegar até a opção \textit{"Collaborators"} e informar o email ou o nome de usuário do colaborador conforme apresenta a Figura~\ref{fig:ap3:pagina_adicionar_colaborador_repositorio_github}.

\captionsetup[figure]{list=no}
\begin{figure}[h!]
	\centerline{\includegraphics[scale=0.5]{./imagens/apendices/pagina-adicionar-colaborador-ao-repositorio.png}}
	\caption[Página para adicionar um contribuidor ao repositório no GitHub.]
	{Página para adicionar um contribuidor ao repositório no GitHub. \textbf{Fonte:} Elaborado pelos autores.}
	\label{fig:ap3:pagina_adicionar_colaborador_repositorio_github}
\end{figure}

Após informar o dado do usuário e clicar no botão \textit{"Add collaborator"} o repositório estará pronto para ser utilizado. Para facilitar o manuseio de arquivos e suas respectivas versões, neste controlador de versão foi utlizada a ferramenta gráfica disponibilizada pelo GitHub a fim de facilitar a utilização deste sistema de controle de versão. Para realizar o download desta ferramenta, acesse a \textit{url} \texttt{https://desktop.github.com} por meio de um navegador de internet e clique no botão de download como apresenta a Figura~\ref{fig:ap3:pagina_download_github_para_windows}.

\captionsetup[figure]{list=no}
\begin{figure}[h!]
	\centerline{\includegraphics[scale=0.4]{./imagens/apendices/pagina-download-github.png}}
	\caption[Página de \textit{download} da ferramenta para gerenciamento de repositórios do GitHub.]
	{Página de \textit{download} da ferramenta para gerenciamento de repositórios do GitHub. \textbf{Fonte:} Elaborado pelos autores.}
	\label{fig:ap3:pagina_download_github_para_windows}
\end{figure}

Após realizar o download, deve-se executar o arquivo obtido por meio do processo de \textit{download} anteriormente mencionado. Após executá-lo, o processo de instalação da ferramenta irá iniciar, como apresenta a Figura~\ref{fig:ap3:instalacao_github_para_windows}.

\captionsetup[figure]{list=no}
\begin{figure}[h!]
	\centerline{\includegraphics[scale=0.5]{./imagens/apendices/instalacao-github-step1.png}}
	\caption[Instalação da ferramenta para gerenciamento de repositórios do GitHub.]
	{Instalação da ferramenta para gerenciamento de repositórios do GitHub. \textbf{Fonte:} Elaborado pelos autores.}
	\label{fig:ap3:instalacao_github_para_windows}
\end{figure}

Após concluir a instalação da ferramenta execute-a para configurar o repositório local no computador de trabalho. Com a aplicação em execução clique no botão "+" e em seguida, no menu \textit{"Clone"}, após esses passos localize o repositório desejado e clique em \textit{"Clone Repository"} como apresenta a Figura~\ref{fig:ap3:clonar_repositorio_github}.

\newpage
\captionsetup[figure]{list=no}
\begin{figure}[h!]
	\centerline{\includegraphics[scale=0.4]{./imagens/apendices/clonar-repositorio-github.png}}
	\caption[Processo para clonar repositório do GitHub.]
	{Processo para clonar repositório do GitHub. \textbf{Fonte:} Elaborado pelos autores.}
	\label{fig:ap3:clonar_repositorio_github}
\end{figure}

Após a realização dos passos descritos neste apêndice o repositório no GitHub estará totalmente configurado.
\chapter*{Apêndice 4: Eclipse e Tomcat}
\label{apendice:eclipse_tomcat}

Neste Apêndice serão apresentados os passos necessários para a instalação e configuração do Tomcat e do Eclipse (IDE de desenvolvimento utilizado neste trabalho).

\section*{Eclipse}

Para utilizar a IDE de desenvolvimento Eclipse, é necessário fazer o download dele por meio da \textit{url} \texttt{https://eclipse.org/downloads/}. Ao acessar esta \textit{url} a página de \textit{download} será apresentada ao usuário conforme a Figura~\ref{fig:ap2:pagina_download_eclipse}.

\captionsetup[figure]{list=no}
\begin{figure}[h!]
	\centerline{\includegraphics[scale=0.4]{./imagens/apendices/pagina-download-eclipse.png}}
	\caption[Página de \textit{download} do Eclipse.]
	{Página de \textit{download} do Eclipse. \textbf{Fonte:} Elaborado pelos autores.}
	\label{fig:ap2:pagina_download_eclipse}
\end{figure}

Nesta página o usuário deve selecionar qual a versão do sistema operacional ele utiliza para realizar o \textit{download} da versão correta.

Após realizado o download, o usuário deve descompactar o arquivo obtido e armazená-lo na pasta que desejar. Para iniciar o eclipse o usuário deve executar o arquivo eclipse.bat para sistemas operacionais Windows ou eclipse.sh para sistemas baseados em Unix como o Linux ou o OS X.

Após a executação deste arquivo o Eclipse, solicitará o diretório para criação do diretório de trabalho que ele irá utilizar, nesse momento o usuário deve informar um diretório válido. A Figura~\ref{fig:ap2:eclipse_selecionar_workspace} apresenta esta configuração.

\captionsetup[figure]{list=no}
\begin{figure}[h!]
	\centerline{\includegraphics[scale=0.5]{./imagens/apendices/eclipse-selecionar-workspace.png}}
	\caption[Tela de definição de diretório de trabalho do Eclipse.]
	{Tela de definição de diretório de trabalho do Eclipse. \textbf{Fonte:} Elaborado pelos autores.}
	\label{fig:ap2:eclipse_selecionar_workspace}
\end{figure}

Após realizar os procedimentos descritos nesta seção, a tela inicial do eclipse será apresentada ao usuário, que, nesse instante estará apto a utilizá-lo e realizar as devidas configurações.

\section*{Tomcat}

Para utilizar o Tomcat foi necessario fazer o \textit{download} dele por meio da \textit{url} \\ \texttt{https://tomcat.apache.org}, após acessá-la, selecione a versão que deseja e clique sob ela. A Figura~\ref{fig:ap2:pagina_inicial_apache_tomcat}.

\newpage
\captionsetup[figure]{list=no}
\begin{figure}[h!]
	\centerline{\includegraphics[scale=0.35]{./imagens/apendices/pagina-inicial-apache-tomcat.png}}
	\caption[Página inicial do Tomcat.]
	{Página inicial do Tomcat. \textbf{Fonte:} Elaborado pelos autores.}
	\label{fig:ap2:pagina_inicial_apache_tomcat}
\end{figure}

Após a seleção da versão do Tomcat a página específica da versão selecionada será apresentada ao usuário conforme a Figura~\ref{fig:ap2:pagina_download_apache_tomcat}. Clique na opção desejada para realizar o \textit{download}. Neste trabalho foi utilizada a versão compactada do \textit{Core}.

\captionsetup[figure]{list=no}
\begin{figure}[h!]
	\centerline{\includegraphics[scale=0.4]{./imagens/apendices/pagina-download-tomcat-7.png}}
	\caption[Página para \textit{download} do Tomcat.]
	{Página para \textit{download} do Tomcat. \textbf{Fonte:} Elaborado pelos autores.}
	\label{fig:ap2:pagina_download_apache_tomcat}
\end{figure}

Após o processo de \textit{dowload} concluído, deve-se descompactar o arquivo obtido em um diretório.

\subsection*{Configuração do Tomcat}

Para utilizar o Tomcat em conjunto com o Eclipse, foi necessário realizar algumas configurações que serão apresentadas a seguir.

Com o Eclipse sendo executado, abra a perspectiva \textit{Server} como demonstra a Figura~\ref{fig:ap2:perspectiva_server_no_server_eclipse} e clique no \textit{link} apresentado.

\captionsetup[figure]{list=no}
\begin{figure}[h!]
	\centerline{\includegraphics[scale=0.5]{./imagens/apendices/perspectiva-server-sem-servidor.png}}
	\caption[Perspectiva \textit{Server} do Eclipse.]
	{Perspectiva \textit{Server} do Eclipse. \textbf{Fonte:} Elaborado pelos autores.}
	\label{fig:ap2:perspectiva_server_no_server_eclipse}
\end{figure}

Uma nova tela será apresentada solicitando ao usuário que selecione a versão do servidor que deseja adicionar como mostra a Figura~\ref{fig:ap2:selecioanr_versao_server_adicionar_eclipse}, após esta definição clique no botão \textit{"Next"}.

\captionsetup[figure]{list=no}
\begin{figure}[h!]
	\centerline{\includegraphics[scale=0.4]{./imagens/apendices/criar-configuracao-tomcat-no-eclipse.png}}
	\caption[Definir a versão do novo servidor no Eclipse.]
	{Definir a versão do novo servidor no Eclipse. \textbf{Fonte:} Elaborado pelos autores.}
	\label{fig:ap2:selecioanr_versao_server_adicionar_eclipse}
\end{figure}

Após selecionar a versão do Tomcat, deve-se informar na próxima tela, como apresenta a Figura~\ref{fig:ap2:definir_diretorio_tomcat_no_eclipse}, as informações a respeito do diretório cujo Tomcat foi extraído e, sob qual versão do Java ele será executado, com essas informações definidas clique no botão \textit{"Finish"}.

\newpage
\captionsetup[figure]{list=no}
\begin{figure}[h!]
	\centerline{\includegraphics[scale=0.4]{./imagens/apendices/definir-pasta-home-tomcat.png}}
	\caption[Definir as configurações do Tomcat no Eclipse.]
	{Definir as configurações do Tomcat no Eclipse. \textbf{Fonte:} Elaborado pelos autores.}
	\label{fig:ap2:definir_diretorio_tomcat_no_eclipse}
\end{figure}

Após realizado todos os procedimentos descritos neste Apêndice, a IDE Eclipse em conjunto com o Tomcat estarão prontos para serem utilizados.

\chapter*{Apêndice 5: Banco de dados Neo4j}
\label{apendice:neo4j}

Neste Apêndice serão apresentados os passos para a instalação do banco de dados Neo4j.

\par Para realizar o \textit{download} do instalador do banco de dados Neo4j, deve-se acessar a seguinte URL, por meio de um  navegador de internet: \texttt{http://neo4j.com/download} e selecionar a opção desejada. Neste trabalho como já descrito foi utilizada a versão \textit{Community}. A Figura~\ref{fig:ap3:download_neo4j} apresenta a página de \textit{download} do Neo4j.

\captionsetup[figure]{list=no}
\begin{figure}[h!]
	\centerline{\includegraphics[scale=0.4]{./imagens/apendices/download-neo4j.png}}
	\caption[Página de \textit{download} do Neo4j.]
	{Página de \textit{download} do Neo4j. \textbf{Fonte:} http://neo4j.com/download}
	\label{fig:ap3:download_neo4j}
\end{figure}

\par Após concluído o \textit{download}, deve-se executar o arquivo. O processo de instalação se inicia e a primeira tela apresentada ao usuário é a tela contendo uma mensagem de boas vindas, conforme demonstra a Figura~\ref{fig:ap3:boas_vindas_neo4j}. Nesta tela, deve-se clicar no botão \textit{Next} para prosseguir com o processo de instalação.

\newpage
\captionsetup[figure]{list=no}
\begin{figure}[h!]
	\centerline{\includegraphics[scale=0.4]{./imagens/apendices/neo4j-install-step1.png}}
	\caption[Tela de boas vindas da instalação do Neo4j.]
	{Tela de boas vindas da instalação do Neo4j. \textbf{Fonte:} Elaborado pelos autores.}
	\label{fig:ap3:boas_vindas_neo4j}
\end{figure}

\par A próxima tela apresentada ao usuário diz respeito ao contrato de uso do \textit{software}, como mostra a Figura~\ref{fig:ap3:contrato_neo4j}. Após lê-lo, deve-se aceitar os termos do contrato e clicar em \textit{Next}.

\captionsetup[figure]{list=no}
\begin{figure}[h!]
	\centerline{\includegraphics[scale=0.4]{./imagens/apendices/neo4j-install-step2.png}}
	\caption[Tela do contrato de uso do Neo4j.]
	{Tela do contrato de uso do Neo4j. \textbf{Fonte:} Elaborado pelos autores.}
	\label{fig:ap3:contrato_neo4j}
\end{figure}

\par Na próxima tela, conforme a Figura~\ref{fig:ap3:definicao_diretorio_neo4j} demonstra, é definido o diretório de instalação do Neo4j. Por padrão este diretório é o mesmo das demais aplicações no \textit{Windows}, podendo ser alterado conforme a necessidade. Após definir o diretório de instalação deve-se clicar no botão \textit{Next}.

\captionsetup[figure]{list=no}
\begin{figure}[h!]
	\centerline{\includegraphics[scale=0.4]{./imagens/apendices/neo4j-install-step3.png}}
	\caption[Tela para definição do diretório de instalação do Neo4j.]
	{Tela para definição do diretório de instalação do Neo4j. \textbf{Fonte:} Elaborado pelos autores.}
	\label{fig:ap3:definicao_diretorio_neo4j}
\end{figure}

\par Após as definições anteriores, uma tela é apresentada questionando o usuário a respeito da criação de atalhos na área de trabalho, como é demonstrado na Figura~\ref{fig:ap3:criacao_atalho_neo4j}. Posterior à definição dos atalhos do Neo4j, deve-se clicar no botão \textit{Next}.

\captionsetup[figure]{list=no}
\begin{figure}[h!]
	\centerline{\includegraphics[scale=0.4]{./imagens/apendices/neo4j-install-step4.png}}
	\caption[Tela para criação de atalhos do Neo4j.]
	{Tela para criação de atalhos do Neo4j. \textbf{Fonte:} Elaborado pelos autores.}
	\label{fig:ap3:criacao_atalho_neo4j}
\end{figure}

\par Após realizar os procedimentos descritos para a instalação do Neo4j a tela final de instalação será apresentada, informando-o a respeito do resultado da instalação conforme demonstra a Figura~\ref{fig:ap3:tela_final_neo4j}. Clique no botão \textit{Finish} para finalizar o processo de instalação.
Após todos os passos realizados com sucesso, o Neo4j estará disponível.

\captionsetup[figure]{list=no}
\begin{figure}[h!]
	\centerline{\includegraphics[scale=0.4]{./imagens/apendices/neo4j-install-step5.png}}
	\caption[Tela final de instalação do Neo4j.]
	{Tela final de instalação do Neo4j. \textbf{Fonte:} Elaborado pelos autores.}
	\label{fig:ap3:tela_final_neo4j}
	
\end{figure}

\par A Figura~\ref{fig:ap3:tela_inicial_neo4j} apresenta a tela inicial do Neo4j após sua instalação. Para iniciar a utilização desse banco de dados, clique no botão \textit{"Start"}.

\captionsetup[figure]{list=no}
\begin{figure}[h!]
	\centerline{\includegraphics[scale=0.60]{./imagens/apendices/neo4j.jpg}}
	\caption[Tela de inicialização do Neo4j.]
	{Tela de inicialização do Neo4j. \textbf{Fonte:} Elaborado pelos autores.}
	\label{fig:ap3:tela_inicial_neo4j}
\end{figure}

\par Após realizar todos os procedimentos descritos neste apêndice o banco de dados Neo4j estará pronto para ser utilizado.

%\captionsetup[figure]{list=no}
%\begin{figure}[h!]
% \centerline{\includegraphics[scale=0.3]{./imagens/apendice_img1.png}}
% \caption[Outra imagem ainda.]
%           {Outra imagem ainda. \textbf{Fonte:} Elaborado pelos autores}
%  \label{fig:ap2:identificador}
%\end{figure}

%\section*{Segunda seção do apêndice 2}


%\par Continuando \ldots na figura Figura~\ref{fig:xml_exemplo} é mostrado um exemplo de XML.

%\begin{figure}[h!]
%\begin{lstlisting}[style=custom_XML]
%<project>
%...
% <dependencies>
%  ...
%  <dependency>
%   <groupId>org.neo4j</groupId>
%   <artifactId>neo4j</artifactId>
%   <version>1.9.4</version>
%  </dependency>
%  ...
% </dependencies>
% ...
%</project>
%\end{lstlisting}  
% \caption[Exemplo de código XML.]
%           {Exemplo de código XML. \textbf{Fonte:} Elaborado pelos autores}
%  \label{fig:xml_exemplo}
%\end{figure}


\chapter*{Apêndice 6: Questionário}
\label{apendice:questionario}

Nesse Apêndice será apresentado o questionário disponibilizado na internet e que foi utilizado para identificar a viabilidade desta pesquisa para a região de Pouso Alegre e sul de Minas Gerais.

Esse questionário teve como título ''Busca por mão de obra qualificada''. A fim de facilitar o tratamento e manipulação das respostas obtidas, foram utilizadas perguntas que seguirão o padrão de multipla escolhas. As perguntas utilizadas nessa pesquisa em conjunto respectivas opções são apresentadas a seguir:

\begin{enumerate}
	\item Qual das situações abaixo é a sua situação atual?
	\begin{enumerate}[label=(\alph*)]
		\item Moro com meus familiares;
		\item Moro em uma república com amigos;
		\item Moro sozinho.
	\end{enumerate}
	
	\item Como é a sua rotina de trabalho?
	\begin{enumerate}[label=(\alph*)]
		\item Fico o dia todo fora;
		\item Fico um período do dia fora;
		\item Trabalho em casa.
	\end{enumerate}
	
	\item Referente aos trabalhos domésticos e rotineiros, como você os realiza?
	\begin{enumerate}[label=(\alph*)]
		\item Eu mesmo faço os trabalhos no tempo vago;
		\item Contrato um profissional temporário para cuidar do trabalho;
		\item Raramente os realizo.
	\end{enumerate}

	\item Encontra dificuldade para encontrar mão de obra temporária? Tais como babá, encanador, pedreiro, etc.
	\begin{enumerate}[label=(\alph*)]
		\item Sim, encontro dificuldades;
		\item Não, encontro facilmente;
		\item Não utilizo.
	\end{enumerate}
	
	\item Quando você precisa destes tipos de serviços, qual das opções abaixo você considera ser o principal motivo da dificuldade de encontrá-los:
	\begin{enumerate}[label=(\alph*)]
		\item Este tipo de trabalho está escasso;
		\item Preciso de alguém de confiança;
		\item O custo está muito alto.
	\end{enumerate}
	
	\item Gostaria de um aplicativo para ajudar a encontrar este tipo de profissional?
	\begin{enumerate}[label=(\alph*)]
		\item Sim, seria muito útil;
		\item Não, não ajudaria;
		\item Talvez possa ser útil.
	\end{enumerate}	
\end{enumerate}


Com este questionário foi possível verificar a viabilidade deste trabalho no ambito social na região do sul de Minas Gerais, uma vez que foi obtido mais de cento e vinte respostas e cerca de 75\% demonstrou interesse nesse tipo de aplicação. 


\end{apendicesenv}
%%\anexoname{ANEXOS}
\begin{anexosenv}
	\partanexos*
	%\chapter*{ANEXO I}

\end{anexosenv}

\addcontentsline{toc}{chapter}{ANEXOS}

\chapter*{ANEXO I}



\printindex

\end{document}